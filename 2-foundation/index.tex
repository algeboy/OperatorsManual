
\cleardoublepage

\includegraphics[width=\textwidth]{curry-feys.png}

\chapter{Substitution and Operators}


There are a lot ways we may encounter number of which grammar is a powerful 
but variable method.  For instance we already know of these
\begin{quote}
    \code{<<Tally>>::= _ | I <Tally>}\\
    \code{<<Nat>>::= 0 | S <Nat>}\\
    \code{<<Peano>>::= 0 | <Peano>+}
\end{quote}
To look past these cosmetic differences we often turn to a 
notation that lists the before-after and ignores specific choices of 
grammar.
To summarize there are two rules that have emerged as the only means to introduce 
natural numbers. From nothing we get the natural number $0:\mathbb{N}$.
From any natural number $k:\mathbb{N}$ we obtain another $S(k):\mathbb{N}$.
It is customary to denoted all the hypothesis required for a deduction in a column or 
several columns above a horizontal line over the conclusion.  Hence,
\begin{gather}
    \tag{Introduction$_{\mathbb{N}}$}
    \overline{0:\mathbb{N}}
    \qquad
    \frac{k:\mathbb{N}}{S(k):\mathbb{N}}
\end{gather}
This notation, known as \emph{Gentzen's sequent calculus} will eventually
eclipse our use of grammars since it captures just what is necessary in terms of
data and is indifferent to small grammatical choices.

\section{Free substitution}\index{variable}\index{substitution}

It is a general reality that attention to domains is an af\-ter\-thought to an
algebra problem, not a forgoing assumption.  In fact if we explore inputs
outside the domain we might discover they are not so bad.  Many of them would
simply lead to complex numbers, and we might arrange for that eventuality.  
What we really have is an understanding of functions as a process to operate 
on data.  Later  we add aspects such as domains and codomains to study them in restricted ways.

So what is a function if we do not have domains and codomains with which to start?

Let us take seriously the idea of simple substitution: replacing $x$ for something else.
Start with something familiar like substituting $4$ for $x$, but to make 
it sporting use Roman numerals.
\[
    f(IV) = \frac{\sqrt{e^{IV}-1}}{IV^2-IV-6}.
\]
This is hideous but at the same time it remains appropriate, we can interpret 
$f$ as applying to $4$ however we choose to represent it.  The same happens with 
the concept of the variable $x$, we can rename it.
\[
    f(t) = \frac{\sqrt{e^{t}-1}}{t^2-t-6}.
\]
The way we know how is wrapped up in the fact that we know 
in principle how to replace $x$ by any symbol, say $x\leftrightarrows \clubsuit$
\[
    f(\clubsuit) = \frac{\sqrt{e^{\clubsuit}-1}}{\clubsuit^2-\clubsuit-6}.
\]
If we need to make sense of expressions like these we might call $\clubsuit$ 
a ``place holder'' for some yet undeclared explanation.  Yet on the otherhand 
I think it best to think of this more similar to  $f(-1)$, which, if used 
would lead to the temporary absurdity of a negative inside a square-root.
It would be unknown perhaps but not disqualifying.

This type of substitution is completely sound.  To make this clear 
we again have simplify the reading of a math formula by searilizing 
the symbols as we did in Section~\ref{sec:cfg}.
In this way everything we write can be discussed as a string $M_1 M_2\cdots M_{\ell}$
of smaller strings $M_i$, even if in reality the layout is more interesting.
This decomposition into smaller strings is not generally unique.  

\begin{remark}{Free choices}
    There are no agreed to conventions that say to read a fraction top-to-bottom
    instead of bottom-to-top, or to read a product of monomial left-to-right
    verses right-to-left. And yet we get the same answers.  Why?\\

    For all the efforts on orders of operations and conventions, there is simply 
    too much variability in how we breakup formulas into strings to ever judge this 
    as unique.  So something else is at play to make all these choices immaterial.
    Notice if a function is truly just the output for every input then this problem 
    does not exist---there are no robbers just their crime scene.
\end{remark}

With this notion of string, let us begin with the general case where everything 
can be replaced.  This means the atoms of the decomposition are restricted to an
alphabet we call ``variables''. That is all it means to be a variable:
\begin{center}
    \textbf{you are a
variable if you are in the alphabet of variables. }
\end{center}

\begin{definition}[Pure substitution]
    Given strings $M$ and $N$, and a variable $x$, to replace $x$ in $M$ by $N$ ,
    denoted here as $M[x\leftrightarrows N]$ by follow these rules.
    \begin{description}
        \item[Free.match] $x[x\leftrightarrows N]\defeq N$.
        \item[Free.other] $M[x\leftrightarrows N]\defeq M$ if $M$ is in the variables alphabet (and 
        because we already will have intercepted the case $M=x$ in the above case we know $M\neq x$).
        
        \item[Free.recurse] $(LM)[x\leftrightarrows N]\defeq L[x\leftrightarrows N]M[x\leftrightarrows N]$
    \end{description}
\end{definition}

A word on notation.  Some authors us $M[N/x]$ or $[N/x]M$ or $M[x/N]$ in place
of $M[x\leftrightarrows N]$.  That notation runs into confusion in algebra
which has other intentions for `/'. 

\begin{remark}{Replace variables don't assign them}
    There is a widely used abbreviation $x\defeq 2$ used 
    in place of $x[x\leftrightarrows 2]$ popular especially to programs.  
    It is common to encounter arguments shaped like the following.  Given:
    \begin{align*}
        M & \defeq x+3 & N & \defeq 2x
    \end{align*}
    ``Assign $x \defeq 2$ to find...''
    \begin{align*}
        M  & = 2+3 =5 & N & = 2\cdot 2 =4
    \end{align*}
    Strictly speaking, $x$ is in the variable alphabet and $2$ in the constant 
    alphabet so no amount of ``assignment'' can turn one into the other.
    The more accurate description is the following:
    \begin{align*}
        % M & \defeq x+3 & N &\defeq 2x\\
        M[x\leftrightarrows 2] & = 2+3=5 & N[x\leftrightarrows 2] & = 2\cdot 2=4.
    \end{align*}
    Confusing replacement with assignment leads eventually to confused 
    results.  For example, if later we reassign $x\defeq 7$ then $M=10$ not 5.
    To unwind this requires that we sequence all uses of $M$ and think of $M$ 
    ``at time/step 1, 2, ...''.  Retaining the substitution notation is 
    on the other hand always accurate.
\end{remark}



% This type of substitution got us into trouble, after all $(K_c(x)=x)[c\leftrightarrows x]$
% would become $x$.



% It will shock no one that given a formula
% \[
%     f(x) = \frac{x}{\sqrt{x^{2}-\frac{5}{3}x}}
% \]
% it is possible to replace $x$ by any symbol I like.  
% Here I swapped $x\leftrightarrows 2$:
% \[
%     f(2) = \frac{2}{\sqrt{2^2-\frac{5}{3}2}}=\frac{2\sqrt{3}}{\sqrt{2}}.
% \]
% To be cute I next used the Roman Numeral $x\leftrightarrows III$
% \[
%     f(III) = \frac{III}{\sqrt{III^2-\frac{5}{3}III}} = \frac{III}{II}.
% \]
% While it is unusual to work with Roman numerals this was in fact 
% valid.  To give it meaning we had to consult some interpretation 
% of Roman numerals as numbers and calculate in that notation. 
% As that worked what about $x\leftrightarrows \clubsuit$:
% \[
%     f(\clubsuit) = \frac{\clubsuit}{\sqrt{\clubsuit^2-\frac{5}{3}\clubsuit}}.
% \]
% At this point we might see this as a step too far.  Perhaps we might observe 
% that $\clubsuit$ is not in the domain.

% To call this ``evaluating $f$'' goes too far.  We truly are erasing 
% one symbol and drawing in another.  To emphasize this, suppose 
% we had not been given ``$f(x)=...$'' but instead this:
% \begin{align*}
%     f(x) \includegraphics[width=0.5cm]{sheep.jpg} x\sqrt{x^2-x}
%     \qquad 
%     f(\clubsuit) \includegraphics[width=0.5cm]{sheep.jpg}\clubsuit \sqrt{\clubsuit^2-\clubsuit}
% \end{align*}
% The substitution worked just as well and our only concern now is will the sheep eat 
%  the clovers $\clubsuit$.  

% This silly illustration points out that $f(x)=...$ 
% is tempting us to put too many assumptions on what the symbols mean.
% For example you might have thought there was an implied domain of decimal numbers.
% After all square-roots often turn up for decimals.  And yet that cannot be 
% enough motivation because all of us will, if we are honest most of use will on occasion 
% attempt to evaluates such a function at a number that wont make sense long term.


% , and if treated 
% just as symbols our minds may no longer lead us to a misunderstanding.  The locations
% of a fixed symbol are what matter, not the symbol itself.

% Given that location matter, notice that formulas use an array of locations:
% left-right $LM$, e.g.\ $(x+2)(x+3)$; up-down $\overset{L}{M}$, e.g.\ $\frac{x+2}{x+3}$
% on the diagonals $L^M$, e.g.$(x+2)^{(x+3)}$, in three dimensions, e.g. a tensor product $u\otimes v\otimes w$
%     \begin{center}
%         \includegraphics[width=2in,page=26]{Tensor-Product-Def-3D.pdf}
%     \end{center}
% The trouble is that when exploring substitution we want to give a step-by-step 
% description, not including nuances of locations.  So, actual location not-withstanding,
% when we decompose a formula we shall narrow 
% our notation to inline, so $N=LM$ and think of the result as a ``string'' reading 
% left-to-right.

Often in our applications it makes sense to leave some symbols fixed.
For example $0,1,2,3,\ldots$ or the number $\pi$ might not vary in an application.
When this is the case we make a separate alphabet of constants and change substitution 
rules around that alphabet.
\begin{definition}[Applied substitution]
    Given strings $M$ and $N$ with variables or constants, 
    and a variable $x$, to replace $x$ in $M$ by $N$ 
    follow the rules of pure substitution but add a  base case:
    \begin{description}
        \item[Constant] $c[x\leftrightarrows N]\defeq c$ when $c$ is a constants alphabet. 
    \end{description}
\end{definition}

Towards our earlier point in Chapter~\ref{chp:what-is-algebra}, pure algebra has no constants.

\section{Bound substitution}\index{bound}
With everything said so far we seem no closer to unravelling the paradox 
of the trapped variable.  Following our rules for free substitution and 
Walrus naming, $K_c(x)\defeq c$ so 
\begin{align*}
    \tag{judgemental equality}
    (K_c(x))[c\leftrightarrows x] 
    & = c[c\leftrightarrows x]\\
    \tag{Free.match}
    & = x.
\end{align*}
The problem however was not in the symbols, but in what we interpreted them 
to mean.  Recall we tempted ourselves to treat both $I$ and $K$ as functions.
The paradox was that the identify \emph{function} should not be a  
constant \emph{function}.  Our sterile substitution simply shifted our notation 
of one function to the notation of the other leaving comparably equal strings 
but meaning vastly different things.  To put a finger right on the problem, 
it was our use of $f(x)\defeq...$ notation for functions that caused the problem.
Functions are more than merely naming some string.

Lets start by replacing the Walrus in our functions.
\begin{align*}
    I(x) & \includegraphics[width=0.5cm]{sheep.jpg} x & 
    K_c(x) & \includegraphics[width=0.5cm]{sheep.jpg} c.
\end{align*}
Now when we substitute we see the situation with fresh perspective.
We cannot seem to substitute on just the left-hand side of equals as 
there is no equals.  There are now long strings.  To substitute 
we need to involve the whole string.
\begin{align*}
    \bigl(I(x) \includegraphics[width=0.5cm]{sheep.jpg} x\bigr)[x\leftrightarrows \clubsuit] &
    \quad = \quad I(\clubsuit) \includegraphics[width=0.5cm]{sheep.jpg} \clubsuit\\
    \bigl(K_c(x) \includegraphics[width=0.5cm]{sheep.jpg} x\bigr)[x\leftrightarrows \clubsuit] &
    \quad = \quad K_c(\clubsuit) \includegraphics[width=0.5cm]{sheep.jpg} c\\
    \bigl(K_c(x) \includegraphics[width=0.5cm]{sheep.jpg} x\bigr)[c\leftrightarrows \clubsuit] &
    \quad = \quad K_{\clubsuit}(x) \includegraphics[width=0.5cm]{sheep.jpg} \clubsuit.
\end{align*}
Written this way we are not tempted to think of functions at all.
We are just as likely to be discussing which string gave our sheep the 
most clovers to eat.

Since Walrus did not lead us to an effective concept of functions, we need 
a new notation that does.  You already know it, just right $x\mapsto x$
for $I$ and $(x,c)\mapsto c$ for $K$.  In fact the way we thought of 
constant functions was as individuals for each $c$ so $x\mapsto c$ and 
$K$ was the function $c\mapsto (x\mapsto c)$ which for each $c$ built a 
function that was constantly returning $c$.  The idea was introduced 
by Alanzo Church.  Church used the notation $\lambda x.x$ and
$\lambda x.c$ because it fit in with other notation such as $\forall x.P$ and
$\exists x.Q$ and so on.  In his honor we describe these functions as 
``Churches'', just kidding, we call them \emph{lambdas} or  
\emph{anonymous functions}.


The variable set apart as the input is said to
be \emph{bound} to the function, it now has a specific role to play similar to
how $\forall x.P$  and $\exists x.Q$ bind $x$ into a special role as well.
Variables that do not have special roles are called \emph{free}.  Note that the
special role of $x$ in $x\mapsto M$ is limited to $M$.  We can say $M$ is the
\emph{scope}.  For this reason programs typically refer to bound variables a
\emph{local}.

Now let us extend substitution into lambdas.
\begin{definition}
Given strings $M<n$ and variables $x,y,\ldots$,
\begin{description}
    \item[Bound.match] $(x\mapsto M)[x\leftrightarrows N]\quad \defeq\quad (x\mapsto N)$
    \item[Bound.trapped]
    if $x$ is a free variable in $M$ and $y$ is a free variable in $N$ then 
    \[ 
        (y\mapsto M)[x\leftrightarrows N]
        \quad \defeq\quad (y'\mapsto (M[y\leftrightarrows y'])[x\leftrightarrows N])
    \]
    where 
    $y'$ is a variable distinct from any in $M$ and $N$ (which exists 
    as there are unbounded numbers of variables);

    \item[Bound.other] $(y\mapsto M)[x\leftrightarrows N]\defeq (y\mapsto M[x\leftrightarrows N])$.
\end{description}
\end{definition}

The first rule Bound.match may at first seem odd.  Aren't we trying to place $x$?
Yes but when we write $x\mapsto M$ we are declaring $x$ as a local variable.  
It is completely meaningless what it is called outside the scope of $M$.
It is the same thing we come to expect when we do things like this:
\begin{align*}
    \sum_{i=1}^{10} i^2 \qquad \prod_{i\in I}X_i 
\end{align*}
or in code 
\begin{center}
\begin{Pcode}[]
def sum(ns)= {
  x = 0
  for n in ns 
    x = x + n
  x  
}

x = [2,3,4]
sum(x)  // the x outside is not the x inside sum
\end{Pcode}
\end{center}

\section{Operations are well-defined}
The most important aspect of functions to preserve is the reliability of evaluation.
\begin{align*}
    \tag{Well-defined}
    x=\acute{x} & \Rightarrow f(x)=f(\acute{x}).
\end{align*}
Mathematicians call this the ``well-defined'' property but it 
is a special case of a much older principle known as Leibniz' law 
and used to define equality.  We will come to that later.

There are of course witty counter-examples.
\begin{align*}
    f\left(\frac{a}{b}\right)\defeq a+b.
\end{align*}
Then using $1/2=2/4$ we find that
\begin{align*}
    1/2 = 2/4 & \Rightarrow 1+2=f(1/2)=f(2/4)=2+4.
\end{align*}
In the exercises you can unwind how this is mostly the 
usual mistake of using nominal judgement ($\defeq$) 
being misinterpreted as operator abstraction ($\mapsto$).
There is no function here at all.

Yet, we ought not dismiss the potential for error hidden 
just below the surface.  After all, what makes us 
confident in functions like $f(x)=x+2$?  If pressed we 
base our judgement on $+$ being well-defined and that 
gets to the heart of whether we trust operators in algebra.

The first answer is that we must all doubt some aspects 
of algebraic operations.  Not only do we make mistakes when 
calculating, we make mistakes when designing operators.  
A negative sign wrong, a bug in the code, a missing 2.  
But on such matters either it gets fixed or no one really needed 
it to begin with.  Lets set that sort of error to the side.

A real mathematical problem is that operators make new strings 
from old, and this can make longer strings.  For example, 
because an operator has no domain by rights it can be applied to 
itself ``f(f)'', only it gets written as $self:x\mapsto xx$.  Let us 
apply this to itself.
\begin{align*}
    self(n) & = (x\mapsto xx)(12345) & \leadsto (xx)[x\leftrightarrows 12345] = 1234512345.\\
\end{align*}

The answer comes in two parts.  First a bit of disappointment.
There is no reason to expect an operation to come to an end.
Not only do we write code with bugs, at a deeper level some 
questions lead to more questions and the computation 
evolves without bound.  Worse still, Turing taught us that 
no program exists to detect if another program will halt.
Thus operators in algebra suffer a fate the perhaps they may 
never finish.  You may think this fate is reserved for esoteric 
logical puzzles, but in examples are encountered routinely.

First we should address when to stop an evaluation.
\begin{definition}
    \begin{itemize}
        \item If $G$ and $F$ are strings and there are strings 
        $G_1,\ldots,G_{\ell}$ such that 
        \[ G=G_1 \leadsto G_2 \leadsto\cdots\leadsto G_{\ell}=F\]
        then write $G\leadsto F$.

        \item if $G\leadsto H$ and $F\leadsto H$ then write $G\betaeq H$.
        In particular $G\betaeq F$ if after renaming variables and doing some reductions 
        they agree.

        \item A formula $F$ is \emph{reduced} when it has no terms where 
        $(x\mapsto M)N$.  
    
    \end{itemize}
    
\end{definition}

\subsection{Confluence and Normal Forms}
\begin{theorem}[Church-Rosser]
    If $L\leadsto_{\beta} M$ and $L\leadsto N$ then there is $O$ such that 
    \[M\leadsto_{\beta} O\qquad N\leadsto_{\beta} O.\] 
    In particular, if $F\leadsto S,R$ where both $R$ and $S$ are reduced,
    then $R\betaeq S$ (that is equal up to possibly renaming a variable).
\end{theorem}
\begin{proof}[Sketch of proof]
Suppose that both $L\leadsto M$ and $L\leadsto N$ use a single $\beta$-reduction.
If they are the same reduction then $M=N$ and are done.  Otherwise 
the two reductions occur in different places in the string $L$.  So we might 
think that we could apply both reductions in parallel to arrive at a string $O$
for which $M\leadsto O$ and $N\leadsto O$, for example:
\begin{align*}
    L = \ldots (x\mapsto A)B\ldots (y\mapsto C)D\ldots 
\end{align*}
can reduce to 
\begin{align*}
    M & = \ldots A[x\leftrightarrows B]\ldots (y\mapsto C)D\ldots \\
    N & =  \ldots (x\mapsto A)B\ldots C[y\leftrightarrows D]\ldots 
\end{align*}
and then both of these to 
\begin{align*}
O =  \ldots A[x\leftrightarrows B]\ldots C[y\leftrightarrows D]\ldots 
\end{align*}
Unfortunately, if we make a reduction on terms that are nested 
the substitutions may evolve with a different number of total moves.
So instead we must take care to select sequences reductions in $M$ and $N$ carefully 
paired so that when applied in parallel they indeed arrive at the same term.  
The type of parallel reductions that do so were introduced by Tait (1965) and
Martin-L\"of (1971). Following this adaptation the proof follows from induction
on the reduction poset.
\begin{center}
\begin{tikzpicture}
    \node (L) at (0,0) {$L$};
    
    \node (M1) at (-1,-1) {$M_1$};
    \node (N1) at (1,-1.5) {$N_1$};

    \node (M2) at (-2,-2.5) {$M_2$};
    \node (O1) at (0,-2) {$O_1$};    
    \node (N2) at (2,-2) {$N_2$};

    \node (M3) at (-3,-3) {$M_3$};
    \node (O21) at (-1,-3.5) {$O_{21}$};    
    \node (O12) at (1,-3) {$O_{12}$};    
    \node (N3) at (3,-3) {$N_3$};

    \node (M) at (-4,-4) {$M$};        
    \node (N) at (4,-4) {$N$};        
    \node (O) at (0,-7) {$O$};

    \draw[-] (L) -- (M1) -- (M2) -- (M3);
    \draw[-] (L) -- (N1) -- (N2) -- (N3);
    \draw[-] (M1) -- (O1) -- (O12);
    \draw[-] (N1) -- (O1) -- (O21);
    \draw[-] (M2) -- (O21);
    \draw[-] (N2) -- (O12);
    \draw[dotted] (M3) -- (M);
    \draw[dotted] (N3) -- (N);
    \draw[dotted] (O12) -- (2,-4);
    \draw[dotted] (O21) -- (-2,-4);
    \draw[dotted] (M) -- (O);
    \draw[dotted] (N) -- (O);

\end{tikzpicture}
\end{center}


A complete proof can be found in Hindley-Seldin Appendix A.
\end{proof}

This is the first of many \emph{normal-form} theorems in algebra.
Normal forms are unique representations given after rewriting a formula.


\begin{definition}
    A operator is a sentence in $\lambda$-calculus.
    Evaluating an operator is to apply $\beta$-reductions.
\end{definition}

\begin{corollary}
    If a operator evaluates in finite time then whatever process it uses 
    gives the same answer.
\end{corollary}



It is the job of a programming language to implement a version 
of substitution that follows these rules.  Once done the notation will take 
on the usual character of the programming language but often the notation 
comes close to the mathematical notations.  Here are some popular variations to try.
\begin{center}
    \code{lambda x.x+2}
    \hspace{1cm}
    \code{x => x+2}
    \hspace{1cm}
    \code{x |-> x+2}\\
    \code{func(x)=x+2}
    \hspace{1cm}
    \code{(x)-> {return x+2}}
\end{center} 


So our traditional $f(x)\defeq M$ notation would no be hinting at 
$f:x\mapsto M$ in our notation, and the $f$ here would be naming 
the specific example $x\mapsto M$, which is often helpful.  Yet 
we should not take this correspondence too far since we have already 
seen the ways in which substituting for $x$ in $f(x)\defeq M$ notation 
goes astray.

The first rule Bound.match may at first seem odd.  Aren't we trying to place $x$?
Yes but when we write $x\mapsto M$ we are declaring $x$ as a local variable.  
It is completely meaningless what it is called outside the scope of $M$.
It is the same thing we come to expect when we do things like this:


In some situations the role of bound/local variables is further 
restricted to roles of a decidedly special meaning, for example, as indices that 
run through a range.
\begin{align*}
    \sum_{i=1}^{10} i^2 \qquad \prod_{i\in I}X_i 
\end{align*}
or in code 
\begin{center}
\begin{Pcode}[]
def sum(ns)= {
  x = 0
  for n in ns 
    x = x + n
  x  
}

x = [2,3,4]
sum(x)  // the x outside is not the x inside sum
\end{Pcode}
\end{center}



% 
\section{Basic Grammar}
Admittedly  $\Box+\Box$, $\Box\cdot \Box$, and $-\Box$ indicate where to place 
information but they do not clarify what can be placed in each spot. 
We can add some clarity by clarifying the grammar with more meaningful tags.
For example, 
\begin{center}
    \code{<Matrix> ::= <Matrix> + <Matrix>}
\end{center}
would clarify that for this $+$ the intension was to add two matrices and 
the result will be another.   If we want to be certain that the dimensions 
match we can add this to the grammar.
\begin{quote}
    \code{<Matrix(2,3)> ::= <Matrix(2,3)> + <Matrix(2,3)>}\\
    \code{<Matrix(2,4)> ::= <Matrix(2,3)>  <Matrix(3,4)>}
\end{quote}
This approach becomes somewhat tedious as it depends so visibly on 
constants what will change between applications.  Later we revisit 
this problem with a few better options.


There are two implicit assumptions in what 
we have written.  First, while this definition is recursive we only intend 
that we should place a $+$ between two matrices that already exist.  In this 
way the grammar looks only back in time eventually settling on some constant
matrices.  Otherwise we could end up with some sort of infinite loop that 
never draws to a close.  Recursion that looks back in time to a start point is 
called \emph{primitive recursion}.  The second unexplained assumption is what 
qualifies as a constant matrix, a base case, to start the process off. 
The zero matrix for example would be one option, as would the matrices 
$E_{ij}$ that have $0$ in all position except row $i$ and column $j$ where 
the number is $1$.  If we include rescaling as an option then through 
linear combinations we could specify any matrix in this way.

Later we shall be more formal with grammars but we close we a few more 
demonstrations.
\begin{center}
\begin{Gcode}
<List> ::= cat <List> <List>
<List> ::= <List> + <List>
<A or B> ::= if <Boolean> then <A> else <B>
\end{Gcode}
\end{center}
When we wish to indicate that symbols $x$ have met the requirement to be 
treated as a type say ``matrix'', or ``list'', or ``Boolean'' we 
write $x:Matrix$, $x:List$, $x:Boolean$ accordingly.  We are also 
lenient with the use of popular shorthand such as $\mathbb{N}$ for natural 
numbers.  So $n:\mathbb{N}$ indicates that $n$ is a natural number.
Here are some related demonstrations.
\begin{quote}
    \code{(cat [1,2,3] [4,5,6]):List}.\\
    \code{([1,2,3] + [4,5,6]):\text{List}}.\\
    $\displaystyle 
        \begin{bmatrix} 1 & 0 & 8 \\ 2 & 7 & -1\end{bmatrix}
    + \begin{bmatrix} -1 & 0 \\ 0 & 1 \end{bmatrix}:\mathbb{R}^{2\times 3}$.
\end{quote}


\section*{Conclusion}

The process of replacing one set of symbols for another based on fixed rules 
is known in algebra circles as \emph{rewriting}\index{rewriting}.  Rewriting 
occurs everywhere from the evaluation of operators like we are explore now, 
to multiplying solvable groups, to solving algebraic geometry problems through 
Gr\"obner-Shirshov bases.  The strategies of rewriting evolve with every context 
but most have an arc similar to the one we pursue in this chapter.  Start by 
considering the order in which you rewrite.  Make it a partial order or at 
least a directed graph.  Search for confluence: places where branches meet back
together.  When enough confluence exists look for a reason for finite rewriting 
to reach a normal form---a unique lowest point, subject to the order we introduce.
Finally, when normal form exist the work begins to get that normal form efficiently.
Above all notice how many possible points of failure there are in effective 
rewriting.  As a general intuition treat rewriting as undecideable, but if decideable 
then exponentially hard, but if efficient then it was never rewriting---it was 
linear algebra in bad notation. 

\section{Basic Grammar}
Admittedly  $\Box+\Box$, $\Box\cdot \Box$, and $-\Box$ indicate where to place 
information but they do not clarify what can be placed in each spot. 
We can add some clarity by clarifying the grammar with more meaningful tags.
For example, 
\begin{center}
    \code{<Matrix> ::= <Matrix> + <Matrix>}
\end{center}
would clarify that for this $+$ the intension was to add two matrices and 
the result will be another.   If we want to be certain that the dimensions 
match we can add this to the grammar.
\begin{quote}
    \code{<Matrix(2,3)> ::= <Matrix(2,3)> + <Matrix(2,3)>}\\
    \code{<Matrix(2,4)> ::= <Matrix(2,3)>  <Matrix(3,4)>}
\end{quote}
This approach becomes somewhat tedious as it depends so visibly on 
constants what will change between applications.  Later we revisit 
this problem with a few better options.


There are two implicit assumptions in what 
we have written.  First, while this definition is recursive we only intend 
that we should place a $+$ between two matrices that already exist.  In this 
way the grammar looks only back in time eventually settling on some constant
matrices.  Otherwise we could end up with some sort of infinite loop that 
never draws to a close.  Recursion that looks back in time to a start point is 
called \emph{primitive recursion}.  The second unexplained assumption is what 
qualifies as a constant matrix, a base case, to start the process off. 
The zero matrix for example would be one option, as would the matrices 
$E_{ij}$ that have $0$ in all position except row $i$ and column $j$ where 
the number is $1$.  If we include rescaling as an option then through 
linear combinations we could specify any matrix in this way.

Later we shall be more formal with grammars but we close we a few more 
demonstrations.
\begin{center}
\begin{Gcode}
<List> ::= cat <List> <List>
<List> ::= <List> + <List>
<A or B> ::= if <Boolean> then <A> else <B>
\end{Gcode}
\end{center}
When we wish to indicate that symbols $x$ have met the requirement to be 
treated as a type say ``matrix'', or ``list'', or ``Boolean'' we 
write $x:Matrix$, $x:List$, $x:Boolean$ accordingly.  We are also 
lenient with the use of popular shorthand such as $\mathbb{N}$ for natural 
numbers.  So $n:\mathbb{N}$ indicates that $n$ is a natural number.
Here are some related demonstrations.
\begin{quote}
    \code{(cat [1,2,3] [4,5,6]):List}.\\
    \code{([1,2,3] + [4,5,6]):\text{List}}.\\
    $\displaystyle 
        \begin{bmatrix} 1 & 0 & 8 \\ 2 & 7 & -1\end{bmatrix}
    + \begin{bmatrix} -1 & 0 \\ 0 & 1 \end{bmatrix}:\mathbb{R}^{2\times 3}$.
\end{quote}

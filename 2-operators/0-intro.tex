One day  $+$ means to add natural numbers, the next day 
polynomials, later matrices.  
You can even add colors ``Yellow=Blue+Green''. When you program 
you learn to add strings
\begin{center}
\begin{notebookin}
print "Algebra " + " is " + " computation"
\end{notebookin}
\begin{notebookout}
Algebra is computation
\end{notebookout}
\end{center}
If we focus on our 
speech we find more expansive uses:
``add the flour, water, yeast, and salt'' or  
``count each household, then add''.

These suggest that addition is a stand in, a variable.  It is not however 
an ordinary variable.  We don't replace $+$ sign with a numbers.  
Addition depends on some inputs, the total 
number being its \emph{valence}.    It also depends on a grammar.
It may be \emph{infix} $2+3$ as we commonly 
use with arithmetic.  It can be \emph{prefix} $+,2,3$ as in the recipe instructions,
or it could be \emph{postfix} $2,3,+$; as in the census illustration.  
The variations in grammar are like any language 
where there could be a dialect that evolves the operator's grammar and lexicon.
 HP calculators were postfix for some time to match engineering requirements.
A program to add two lists could get away with the following linguistic drift:
\begin{center}
\begin{notebookin}
cat [3,1,4] [1,5,9]
[3,1,4] + [1,5,9]
\end{notebookin}
\begin{notebookout}[2]
[3,1,4,1,5,9]
[4,6,13]
\end{notebookout}
\end{center}
Using \texttt{cat} reminded us to concatenate and avoided confusion with the
later $+$ concept.  It was the better choice. Challenge yourself to see both as
addition and you will find addition everywhere. 

Addition  may be with two or more terms.  It also may be 
sensitive to order: ``add the food coloring to the vinegar, then add the baking
powder'' is a different to ``add the vinegar and baking powder then the
food coloring''.  To make a manageable theory, algebra needs fewer variations that 
build towards this complexity.


\section{Operators to bring closure}
Sometimes we need operators to keep the data within some context.
For example we can multiply with rectangular matrices but their shape changes.
If we really plan a multiplication of $(2\times 3)$-matrices we need 
something unusual, like this:
\begin{align*}
    /A,B,C/ & = AB^{\dagger}C
\end{align*}
where $B^{\dagger}$ is the transpose.  This also has a partner.  If we have
$(3\times 2)$-matrices then the following product turns them into $(2\times 3)$-matrices.
\begin{align*}
    \backslash A,B,C\backslash & = A^{\dagger} B C^{\dagger}
\end{align*}
Playing these products off of one another leads to a fuller understanding of left-right 
behaviors of rectangular matrices.
Such products lead to the general 
concept of \emph{pair algebras}, e.g. pairing $\mathbb{R}^{2\times 3}$ 
with $\mathbb{R}^{3\times 2}$, which were introduced by Ottmar Loos in the 1970's.
\index{pair algebra}  These have been useful in sorting out exceptional geometric 
phenomena and obstacles that arise when $2=0$.


Another serious product comes up 
in symmetric matrices.  Notice if $A=A^{\dagger}$ and $B=B^{\dagger}$
then $(AB)^{\dagger}=B^{\dagger}A^{\dagger}=BA$ which is not in general 
the same thing as $AB$.  This means that the theory of symmetric matrices 
appears not to behave well under multiplication and that hampers attempts 
to study geometry and particle physics.

The solution comes in the form of other products, most importantly, 
the  \emph{Jordan Triple product} are the solutions to the equation
\begin{align*}
    2\{A,B,C\} & = ABC+CBA
\end{align*}
This is part of an whole family of Jordan products including solutions to
\begin{align*}
    2(A\bullet B) & = AB+BA\\
    2\langle A_1,\ldots,A_{\ell}\rangle & = A_1\cdots A_{\ell}+A_{\ell}\cdots A_1.
\end{align*}
Why the 2?  It is cosmetic but for example $I_n\bullet A=A=A\bullet I_n$ in this way. 
Notice in all these case if $A_i=A_i^{\dagger}$ then $\langle A_1,\ldots,A_{\ell}\rangle=
\langle A_1,\ldots,A_{\ell}\rangle^{\dagger}$. 
Pascual Jordan invented his products while formalizing 
Heisenberg's matrix model of quantum mechanics in the 1920's, and these were 
carefully investigated by Albert, von Neumann, Jacobson, culminating in the 
field's medal wining work of Efim Zelmanov.


A related product is to consider 
skew-symmetric matrices where $A_i^{\dagger}=-A_i$.  Then we would consider using the following 
alternating variation on the Jordan product known as a \emph{Lie bracket}
\begin{align*}
    [A,B]_+ & = AB-BA
\end{align*}
These operators are needed so that when we begin with a special type of
element---(skew) symmetric matrices, the product is again (skew) symmetric.
In the usual jargon we say that these matrices are \emph{closed} to the operator.
Sophus Lie studied the calculus of derivatives for its algebraic 
qualities in the 1880's.  These came to considerable fame with the rise of algebraic 
geometry pushed along by Emile Cartan, Andre Weil, Dynkin, and the Abel prize winner Jacques Tits. 

When we can divide by 2, the Jordan and Lie perspectives become related.
\begin{align*}
    2AB = 2(A\bullet B)+[A,B].
\end{align*}
\section{Substitution rules}\index{variable}\index{substitution}
The purpose of a variable is to replace it. 

As a start, it matters first to admit that functions do not have 
domains.  Surprised? In actuality domain is a concept added to the language of some 
functions to help us make fewer mistakes.  (There will be functions that simply 
can have no domain no matter how clever we are.)  

For example, it is a bit disingenuous 
to assume that a student exploring 
\[
    f(x) = \frac{\sqrt{1-e^x}}{x^2-x-6}
\]
with a graphing calculator will have already understood to 
avoid $x=(-\infty,0],3$.  It is a general reality that attention to 
domains is an afterthought to an algebra problem,
not a forgoing assumption.  

With examples like this in mind we would like operators that 
can salvage functions when they break, and allow us to make proper 
sense of evaluating $f$ ``outside of its domain.''


It will shock no one that given a formula
\[
    f(x) = \frac{x}{\sqrt{x^{2}-\frac{5}{3}x}}
\]
it is possible to replace $x$ by any symbol I like.  
Here I swapped $x\leftrightarrows 2$:
\[
    f(2) = \frac{2}{\sqrt{2^2-\frac{5}{3}2}}=\frac{2\sqrt{3}}{\sqrt{2}}.
\]
To be cute I next used the Roman Numeral $x\leftrightarrows III$
\[
    f(III) = \frac{III}{\sqrt{III^2-\frac{5}{3}III}} = \frac{III}{II}.
\]
While it is unusual to work with Roman numerals this was in fact 
valid.  To give it meaning we had to consult some interpretation 
of Roman numerals as numbers and calculate in that notation. 
As that worked what about $x\leftrightarrows \clubsuit$:
\[
    f(\clubsuit) = \frac{\clubsuit}{\sqrt{\clubsuit^2-\frac{5}{3}\clubsuit}}.
\]
At this point we might see this as a step too far.  Perhaps we might observe 
that $\clubsuit$ is not in the domain.

To call this ``evaluating $f$'' goes too far.  We truly are erasing 
one symbol and drawing in another.  To emphasize this, suppose 
we had not been given ``$f(x)=...$'' but instead this:
\begin{align*}
    f(x) \includegraphics[width=0.5cm]{sheep.jpg} x\sqrt{x^2-x}
    \qquad 
    f(\clubsuit) \includegraphics[width=0.5cm]{sheep.jpg}\clubsuit \sqrt{\clubsuit^2-\clubsuit}
\end{align*}
The substitution worked just as well and our only concern now is will the sheep eat 
 the clovers $\clubsuit$.  

This silly illustration points out that $f(x)=...$ 
is tempting us to put too many assumptions on what the symbols mean.
For example you might have thought there was an implied domain of decimal numbers.
After all square-roots often turn up for decimals.  And yet that cannot be 
enough motivation because all of us will, if we are honest most of use will on occasion 
attempt to evaluates such a function at a number that wont make sense long term.


, and if treated 
just as symbols our minds may no longer lead us to a misunderstanding.  The locations
of a fixed symbol are what matter, not the symbol itself.

Given that location matter, notice that formulas use an array of locations:
left-right $LM$, e.g.\ $(x+2)(x+3)$; up-down $\overset{L}{M}$, e.g.\ $\frac{x+2}{x+3}$
on the diagonals $L^M$, e.g.$(x+2)^{(x+3)}$, in three dimensions, e.g. a tensor product $u\otimes v\otimes w$
    \begin{center}
        \includegraphics[width=2in,page=26]{Tensor-Product-Def-3D.pdf}
    \end{center}
The trouble is that when exploring substitution we want to give a step-by-step 
description, not including nuances of locations.  So, actual location not-withstanding,
when we decompose a formula we shall narrow 
our notation to inline, so $N=LM$ and think of the result as a ``string'' reading 
left-to-right.

With this notion of string the atoms of the decomposition are restricted to an
alphabet we call ``variables''. That is all it means to be a variable: you are a
variable if you are in the alphabet of variables.  

\begin{definition}[Pure substitution]
    Given strings $M$ and $N$, and a variable $x$, to replace $x$ in $M$ by $N$ ,
    denoted here as $M[x\leftrightarrows N]$ by follow these rules.
    \begin{description}
        \item[Free.match] $x[x\leftrightarrows N]\defeq N$.
        \item[Free.other] $M[x\leftrightarrows N]\defeq M$ if $M$ is in the variables alphabet (and 
        because we already will have intercepted the case $M=x$ in the above case we know $M\neq x$).
        
        \item[Free.recurse] $(LM)[x\leftrightarrows N]\defeq L[x\leftrightarrows N]M[x\leftrightarrows N]$
    \end{description}
\end{definition}

Often in our applications it makes sense to leave some symbols fixed.
For example $0,1,2,3,\ldots$ or the number $\pi$ might not vary in an application.
When this is the case we make a separate alphabet of constants and change substitution 
rules around that alphabet.
\begin{definition}[Applied substitution]
    Given strings $M$ and $N$ with variables or constants, 
    and a variable $x$, to replace $x$ in $M$ by $N$ 
    follow the rules of pure substitution but add the following  base case:
    \begin{description}
        \item[Constant] $c[x\leftrightarrows N]\defeq c$ when $c$ is a constants alphabet. 
    \end{description}
\end{definition}

Towards our earlier point in Chapter~\ref{chp:what-is-algebra}, pure algebra has no constants.

\begin{remark}{Replace variables don't assign them}
    Think of Walrus $\defeq$ as  naming.
    The $\defeq$ is used to define the symbols on the left, what is 
    known as \emph{assignment} or \emph{judgemental equality}.  Notice none of 
    these assigns a value to a variable, say $x$, rather it assigns a value to the variable 
    decorated by instructions, $x[x\leftrightarrow N]$.  
    
    It is common to encounter arguments shaped like the following.  Given:
    \begin{align*}
        M & \defeq x+3 & N & \defeq 2x
    \end{align*}
    ``Assign $x \defeq 2$ to find...''
    \begin{align*}
        M  & = 2+3 =5 & N & = 2\cdot 2 =4
    \end{align*}
    Strictly speaking, $x$ is in the variable alphabet and $2$ in the constant 
    alphabet so no amount of ``assignment'' can turn one into the other.
    The more accurate description is the following:
    \begin{align*}
        % M & \defeq x+3 & N &\defeq 2x\\
        M[x\leftrightarrows 2] & = 2+3=5 & N[x\leftrightarrows 2] & = 2\cdot 2=4.
    \end{align*}
    % This confusion leads to programs that behave erratically. For example $M$
    % may at some points in time be $x+3$ but later become $5$. In general it
    % serves us to recognize the subtle difference between ``assigning a
    % variable'' (which is a ultimately now well-defined) 
    % and ``replacing a variable'' which always makes sense and is predictable.
\end{remark}
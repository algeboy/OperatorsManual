\section{Operators measuring defects}
Algebraist spend a lot of time worried about misbehaving operators 
causing them to generate new operators that spot the flaws.  This is another 
source of higher valence operators. If 
we can add, subtract, and multiply then we can make the following operators as well.
\begin{align*}
    [a,b] & = ab-ba \tag{Commutator}\\
    (a,b,c) & = a(bc)-(ab)c \tag{Associator}
\end{align*}
This commutator turns out to be the same as Lie's bracket and that coincidence 
has often been a subject to exploit.  Even so the goals are quite different. 
In Lie's case we need a product that respect skew-symmetry, we essentially forget 
the original matrix product and use just that Lie bracket.  Meanwhile when we 
approach this product as a commutator the entire purpose is to study the original 
product and gain insights by looking at the behavior of its commutator.

Commutative algebra requires $[a,b]=0$ while associative algebra needs $(a,b,c)=0$.
For example matrices fail to be commutative algebra but are associative.
Replace the role of multiplication of matrices with $[a,b]$ and ask for it's 
associate, i.e.
\begin{align*}
    (a,b,c)_{[,]} & = [a,[b,c]]-[[a,b],c]
\end{align*}
and we no longer get associative nor commutative algebra.  These are structures 
known as Lie algebras.  While not associative, because they are based originally 
on matrix products that are associative we can stumble eventually upon a 
graceful alternative
\begin{align*}
    0 & = [a,a] \tag{Alternating}\\
    0 & = [a,[b,c]]+[b,[c,a]]+[c,[a,b]].    
    \tag{Jacobi}
\end{align*}
So the problem is not getting worse, at least we wont be needing 
to look into some  valence 4 operators as defects.  Sabinin algebra 
studies how defects in operators pile up or die off.



Keep in mind requiring that $[a,b]=0$ or $(a,b,c)=0$ is an equation. 
Like any equation it has limited solutions.  By that reasoning, 
most of algebra wont behave commutative nor associative.

There are also variations on these.  If we have multiplication $\bullet$, 
an identity and inverses we can look for commuting products:
\begin{align*}
    [a,b]_{\bullet} & =(ab)^{-1}(ab)
    &
    {_{\bullet} [a,b]} & =(ab)(ab)^{-1}.
\end{align*}
These are especially useful in the  theory of equivalence, commonly subsumed 
in the topic of group theory and higher category theory.
Sometimes we have only left/right inverses so the same concept shifts to 
use these one-sided operators.
\begin{align*}
    [a,b]_{\bullet} & =(ab)\backslash (ab)
    &
    {_{\bullet} [a,b]} & =(ab)/(ab).
\end{align*}
These products emerge in the study of loops and combinatorics.

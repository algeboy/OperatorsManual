
\section{Operators that unify}
Stranger ternary products showup in places where we wish we had easier binary products 
to explain things.  For example in geometry, part of the success of algebra and 
geometry was the realization that lines are described by the equations $y=mx+b$.
Yet that requires a number of hard to meet conditions on geometry beyond the obvious 
point-line intersection rules.  So when it comes to very general geometries 
it was not know how to describe lines algebraically as $y=mx+b$ because 
appropriate choices of $+$ and $\cdot$ were not known.
So Marshall Hall decided we could simply invent a trivalent (ternary) operator
\[
    -\otimes-\oplus -
\]
This is merely suggestive notation.  It allows us to write an equation 
that looks like our line equation:
\[
    y=m\otimes x\oplus b
\] 
yet this is not an amalgum of two operators just one single trivalent operator.
With this trivalent product, Hall was able to associate every 
projective plane to coordinates in some ``ternary ring''.  Once you have such a ternary ring you 
can go to work to see if it might actually decompose into two binary operations of multiplicaton 
and addition, e.g.\ by locating a ``one'' and a ``zero'' where $x=1\otimes x\oplus 0$.  Then 
you reverse the process and define $m\cdot x\defeq m\otimes x\oplus 0$ and $x+b\defeq 1\otimes x\oplus b$
to get a more familiar ring-like structure.

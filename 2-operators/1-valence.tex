
\subsection{Valence}
\index{variadic}
We can add any finite list of $\sum_{i} n_i$ and 
programs back this up with commands like \code{sum(ns)} 
(the convention in programs is that a sequence $n_*$ is transcribed as 
the plural \code{ns}).  Likewise we can concatenate any 
number of strings.  In reality though, we have a limited work force:
ourselves and our machines. We therefore end up adding a bounded number at once,
often just 2.  So while we can entertain addition as having 
\emph{variadic} (variable valence), it is a practical reality that 
we build up arbitrary valance by composing several operators of 
fixed valence.

\index{bivalent!opeator}\index{binary operator|see{bivalent operator}}
Addition from here on out will be bivalent (also called binary), meaning 
that it requires 2 inputs.  We typically prefer infix grammar $\Box +\Box$.
Since we are evolving, we may as well permit multiplication as a bivalent operator
symbol, changing the signature to $\Box \cdot \Box$, i.e. $2\cdot 4$; or
$\Box\Box$, e.g. $xy$.   Avoid $\Box\times \Box$, we need that symbol elsewhere.
These days composition $\Box\circ\Box$ is written as multiplication; so, you can
use that symbol however you like.  Addition is held to high standards in algebra
(that it will evolve into linear algebra).  So when you are considering a binary
operation with few if any good properties, use a multiplication inspired
notation instead.   




\index{univalent!opeator}\index{unary operator|see{univalent operator}} Valence
1, also called \emph{univalent} or \emph{unary}, operators include the negative
sign $-\Box$ to create $-2$ as well as the transpose $A^{\dagger}$ of a matrix
$A$. Notice in the case of negative an integer remained an integer, but in the
case of transpose a $(2\times 3)$-matrix becomes a $(3\times 2)$-matrix.
Operators can change the type of data we explore.
 
Programming languages add several others univalent operators
such as \code{++i, --i} which are said to \emph{increment} 
or \emph{decrement} the counter i (change it by $\pm 1$).

Programs also exploit a trivalent (ternary) operator:
\begin{center}
\begin{Pcode}[]
if (...) then (...) else (...)
\end{Pcode}
\end{center}
The words, while helpful, are unimportant and some languages
replace it with symbols emphasizing it is an operator:
\begin{center}
    \pcode{_?_:_}
\end{center}
% Here for example is division with remainder of positive integers
% \begin{center}
% \begin{Pcode}[]
%     div(m,n)=(m>=n)?(div(m-n,n)+(1,0)):(0,m)
% \end{Pcode}
% \end{center}



\section{Operators fixing defects}
Some operators are invented to work around unfriendly behavior.
Imagine a function that tries its best to return a value but 
hits an error or runs short on resources (time or memory) before 
it can locate the right result.  Should this no longer be a function?

For example, it is a bit disingenuous 
to assume that a student exploring 
\[
    f(x) = \frac{\sqrt{1-e^x}}{x^2-x-6}
\]
with a graphing calculator will have already understood to 
avoid $x=(-\infty,0],3$.  It is a general reality that attention to 
domains is an afterthought to an algebra problem,
not a forgoing assumption.  

With examples like this in mind we would like operators that 
can salvage functions when they break, and allow us to make proper 
sense of evaluating $f$ ``outside of its domain.''



\subsection{Partial functions and maybe operators}
As we just showed substitution can be preformed without domains and codomain, but 
domains can help us avoid errors.  For example, in our formula $f$ we might detect 
from the symbols that we expect to get a decimal number out.  And to get this 
we should restrict attention to decimal numbers as inputs. 
The problem of course is that the formula hides some further issues that will 
eventually eliminate more inputs, $(-\infty,0)$ and $3$.  Yet it may be 
worthwhile exploring $f$ as if it is almost a function $\mathbb{R}\to \mathbb{R}$,
we call this a \emph{partial function}, and denote it 
\[ 
    f:\mathbb{R}\dashrightarrow \mathbb{R}
\] 
Now what happens when we substitute by values that do not give real numbers?  We 
can think of this as short-circuiting to a default failed value say $\bot$.  So 
we can just adjoint $\{\bot\}$ to $\mathbb{R}$ and extend our partial function $f$
\begin{align*}
    \mathbb{R}^? & \defeq \mathbb{R}\sqcup \{\bot \}\\
    f^?(x) & = \begin{cases}
                f(x)  & x>0, x\neq 3\\
                \bot & \text{else}
    \end{cases}             
\end{align*}
Now we arrive at an honest function with domain and codomain:
\[
    (f:\mathbb{R}\dashrightarrow \mathbb{R})\mapsto f^?:\mathbb{R}\to \mathbb{R}^?.
\]
The operator $?$ turns partial functions into functions.
Other repairing operators might add clarity to the repairs, for example, 
detailing if the term is undefined, imaginary but not real, $\pm\infty$ and so on.

Programs use such repair operators all the time because basically it is impossible to 
know ahead of time the actual domain of a function.  Such operators go by the 
names of \code{Maybe}, \code{Option}, \code{Either} and some others.
We explore this in  Chapter~\ref{chp:grammar}.

%\usepackage[dvipsnames]{xcolor}
%\usetikzlibrary{external}
%\tikzexternalize[prefix=figures/] % activate and define figures/ as cache folder
\usepackage{tikz-qtree}
\usetikzlibrary{calc}
\usetikzlibrary{hobby}
\usetikzlibrary{positioning, shapes}

\newcommand{\elastic}{-\dimexpr\pgfmatrixcolumnsep+0.6em\relax}

\tikzset{pics/.cd,
opencube/.style args={#1/#2/#3}{code={
\coordinate (O) at (0,0,0);
\coordinate (A) at (0,#2,0);
\coordinate (B) at (0,#2,#3);
\coordinate (C) at (0,0,#3);
\coordinate (D) at (#1,0,0);
\coordinate (E) at (#1,#2,0);
\coordinate (F) at (#1,#2,#3);
\coordinate (G) at (#1,0,#3);
%% Background
\draw[black,dotted] (O) -- (A);
\draw[black,dotted] (O) -- (C);
\draw[black,dotted] (O) -- (D);
% Forground
\draw[black,dashed] (A) -- (E) -- (F) -- (B) -- cycle;
\draw[black,dashed] (E) -- (D) -- (G) -- (C) -- (B);
\draw[black,dashed] (F) -- (G);

%\draw[black,dashed, blue] (O) -- (A) -- (E) -- (D) -- cycle;
%\draw[black,dashed] (O) -- (A) -- (B) -- (C) -- cycle;
%\draw[black,dashed] (D) -- (E) -- (F) -- (G) -- cycle;
%\draw[black,dashed] (C) -- (B) -- (F) -- (G) -- cycle;
%\draw[black,dashed] (A) -- (B) -- (F) -- (E) -- cycle;

}}}

%%%%%%%%%%%%%%%%%%%%%%%%%%%%%%%%%%%%%%%%%%%%%%%%%%%%%%%%%%%%%%%%%%%%%%%%%%%
%% Print a shaded cube of row/column/width  / contents
%%%%%%%%%%%%%%%%%%%%%%%%%%%%%%%%%%%%%%%%%%%%%%%%%%%%%%%%%%%%%%%%%%%%%%%%%%%
\tikzset{pics/.cd,
linecube/.style args={#1/#2/#3/#4}{code={
\coordinate (OO) at (0,0,0);
\coordinate (AA) at (0,#2,0);
\coordinate (BB) at (0,#2,#3);
\coordinate (CC) at (0,0,#3);
\coordinate (DD) at (#1,0,0);
\coordinate (EE) at (#1,#2,0);
\coordinate (FF) at (#1,#2,#3);
\coordinate (GG) at (#1,0,#3);
%% Background
\draw[black,dashed] (OO) -- (AA);
\draw[black,dashed] (OO) -- (CC);
\draw[black,dashed] (OO) -- (DD);

\node at (0.5*#1,0.5*#2,0.5*#3) {#4};

% Foreground
\draw[black] (AA) -- (EE) -- (FF) -- (BB) -- cycle;
\draw[black] (EE) -- (DD) -- (GG) -- (CC) -- (BB);
\draw[black] (FF) -- (GG);

%\draw[black,dashed, blue] (O) -- (A) -- (E) -- (D) -- cycle;
%\draw[black,dashed] (O) -- (A) -- (B) -- (C) -- cycle;
%\draw[black,dashed] (D) -- (E) -- (F) -- (G) -- cycle;
%\draw[black,dashed] (C) -- (B) -- (F) -- (G) -- cycle;
%\draw[black,dashed] (A) -- (B) -- (F) -- (E) -- cycle;

}}}

%%%%%%%%%%%%%%%%%%%%%%%%%%%%%%%%%%%%%%%%%%%%%%%%%%%%%%%%%%%%%%%%%%%%%%%%%%%
%% Print a shaded cube of row/column/width / color / contents
%%%%%%%%%%%%%%%%%%%%%%%%%%%%%%%%%%%%%%%%%%%%%%%%%%%%%%%%%%%%%%%%%%%%%%%%%%%
\tikzset{pics/.cd,
shadedcube/.style args={#1/#2/#3/#4/#5}{code={
\coordinate (O) at (0,0,0);
\coordinate (A) at (0,#2,0);
\coordinate (B) at (0,#2,#3);
\coordinate (C) at (0,0,#3);
\coordinate (D) at (#1,0,0);
\coordinate (E) at (#1,#2,0);
\coordinate (F) at (#1,#2,#3);
\coordinate (G) at (#1,0,#3);
\draw[black,fill=#4!80] (O) -- (C) -- (G) -- (D) -- cycle;
\draw[black,fill=#4!30] (O) -- (A) -- (E) -- (D) -- cycle;
\draw[black,fill=#4!10] (O) -- (A) -- (B) -- (C) -- cycle;
\draw[black,fill=#4!20,opacity=0.8] (D) -- (E) -- (F) -- (G) -- cycle;
\draw[black,fill=#4!20,opacity=0.6] (C) -- (B) -- (F) -- (G) -- cycle;
\draw[black,fill=#4!20,opacity=0.8] (A) -- (B) -- (F) -- (E) -- cycle;
\node at (0.5*#1,0.5*#2,0.5*#3) {#5};
}}}


%%%%%%%%%%%%%%%%%%%%%%%%%%%%%%%%%%%%%%%%%%%%%%%%%%%%%%%%%%%%%%%%%%%%%%%%%%%
%% Print a cube of subcubes, row/column/width / subrows / subcols / subwid / color
%%%%%%%%%%%%%%%%%%%%%%%%%%%%%%%%%%%%%%%%%%%%%%%%%%%%%%%%%%%%%%%%%%%%%%%%%%%
\tikzset{pics/.cd,
gridcube/.style args={#1/#2/#3/#4/#5/#6/#7}{code={
\coordinate (O) at (0,0,0);
\coordinate (A) at (0,#2,0);
\coordinate (B) at (0,#2,#3);
\coordinate (C) at (0,0,#3);
\coordinate (D) at (#1,0,0);
\coordinate (E) at (#1,#2,0);
\coordinate (F) at (#1,#2,#3);
\coordinate (G) at (#1,0,#3);

% Foreground
\draw[fill=#7!20] (A) -- (E) -- (F) -- (B) -- cycle;
\draw[fill=#7!40] (E) -- (F) -- (G) -- (D) -- cycle;
\draw[fill=#7!30] (B) -- (F) -- (G) -- (C) -- cycle;
%\draw[black] (E) -- (D) -- (G) -- (C) -- (B);
%\draw[black] (F) -- (G);

% lines
\foreach \ll in {1,...,#4} {
    \draw[black] (#1/#4*\ll,#2,0) -- (#1/#4*\ll,#2,#3) -- (#1/#4*\ll,0,#3);
}
\foreach \ll in {1,...,#5} {
    \draw[black] (0,#2/#5*\ll,#3) -- (#1, #2/#5*\ll,#3) -- (#1,#2/#5*\ll,0);
}
\foreach \ll in {1,...,#6} {
    \draw[black] (0,#2,#3/#6*\ll) -- (#1,#2,#3/#6*\ll) -- (#1,0,#3/#6*\ll);
}

}}}

\tikzset{pics/.cd,
%%% y/z/color/label
lwing/.style args={#1/#2/#3/#4}{code={
\coordinate (O) at ( 0, 0, 0);
\coordinate (A) at ( 0, 0,#2);
\coordinate (B) at ( 0,#1,#2);
\coordinate (C) at ( 0,#1, 0);
\draw[black,fill=#3] (O) -- (A) -- (B) -- (C) -- cycle;
\node at (0,0.5*#1,0.5*#2) {#4};
}}}

%%% FLAT COLOR RECTANGLE WITH ENTRY
\tikzset{pics/.cd,
mwing/.style args={#1/#2/#3/#4/#5}{code={
\coordinate (O) at ( 0, 0, 0);
\coordinate (A) at (#1, 0, 0);
\coordinate (B) at (#1,#2, 0);
\coordinate (C) at ( 0,#2, 0);
\draw[black,fill=#5!20] (O) -- (A) -- (B) -- (C) -- cycle;
\node at (0.5*#1,0.5*#2,0) {#4};
}}}

\tikzset{
    pics/.cd,
    % width/lenght/xticks/yticks/bgcolor
    matrix/.style args={#1/#2/#3/#4/#5}{
        code={
            \coordinate (O) at ( 0, 0, 0);
            \coordinate (A) at (#1, 0, 0);
            \coordinate (B) at (#1,#2, 0);
            \coordinate (C) at ( 0,#2, 0);
            \fill[color=#5] (O) -- (A) -- (B) -- (C) -- cycle;
            \foreach \x in {0,...,#3} {
                \draw[black] (#1/#3*\x,0,0) -- ++(0,#2,0);
            }
            \foreach \y in {0,...,#4} {
                \draw[black] (0,#2/#4*\y,0) -- ++(#1,0,0);
            }
        }
    }
}

\tikzset{
    pics/.cd,
    % width/lenght/xticks/yticks/bgcolor
    rmatrix/.style args={#1/#2/#3/#4/#5}{
        code={
            \coordinate (O) at ( 0, 0, 0);
            \coordinate (A) at (#1, 0, 0);
            \coordinate (B) at (#1, 0, #2);
            \coordinate (C) at ( 0, 0, #2);
            \fill[color=#5] (O) -- (A) -- (B) -- (C) -- cycle;
            \foreach \x in {0,...,#3} {
                \draw[black] (#1/#3*\x,0,0) -- ++(0,0,#2);
            }
            \foreach \z in {0,...,#4} {
                \draw[black] (0,0,0#2/#4*\z) -- ++(#1,0,0);
            }
        }
    }
}

\tikzset{
    pics/.cd,
    % width/lenght/xticks/yticks/bgcolor
    lmatrix/.style args={#1/#2/#3/#4/#5}{
        code={
            \coordinate (O) at ( 0, 0, 0);
            \coordinate (A) at ( 0, 0,#2);
            \coordinate (B) at ( 0,#1,#2);
            \coordinate (C) at ( 0,#1, 0);
            \fill[color=#5] (O) -- (A) -- (B) -- (C) -- cycle;
            \foreach \z in {0,...,#4} {
                \draw[black] (0,0,#2/#4*\z) -- ++(0,#1,0);
            }
            \foreach \y in {0,...,#3} {
                \draw[black] (0,#1/#3*\y,0) -- ++(0,0,#2);
            }
        }
    }
}

% \tikzset{pics/.cd,
% mwing/.style args={#1/#2/#3/#4}{code={
% \coordinate (O) at ( 0, 0, 0);
% \coordinate (A) at (#1, 0, 0);
% \coordinate (B) at (#1,#2, 0);
% \coordinate (C) at ( 0,#2, 0);
% \draw[black,fill=green!20] (O) -- (A) -- (B) -- (C) -- cycle;
% \node at (0.5*#1,0.5*#2,0) {#4};
% }}}

\tikzset{pics/.cd,
%% x/z/color/label
rwing/.style args={#1/#2/#3/#4}{code={
\coordinate (O) at ( 0, 0, 0);
\coordinate (A) at (#1, 0, 0);
\coordinate (B) at (#1, 0,#2);
\coordinate (C) at ( 0, 0,#2);
\draw[black,fill=#3] (O) -- (A) -- (B) -- (C) -- cycle;
\node at (0.5*#1,0,0.5*#2) {#4};
}}}

\tikzset{pics/.cd,
    abacus2/.style args={#1/#2}{
        code={
            \draw[red] (0,0) -- (1.5,0);
            \draw[red] (0,0.25) -- (1.5,0.25);
            \foreach \u in {1,...,#1} {
                \filldraw[red] (0.1*\u,0) circle (1pt);
            }
            \foreach \u in {1,...,#2} {
                \filldraw[red] (0.1*\u,0.25) circle (1pt);
            }
        }
    } 
}

\tikzset{pics/.cd,
    abacus3/.style args={#1/#2/#3/#4}{
        code={
            \draw[#4] (0,0) -- (1.5,0);
            \draw[#4] (0,0.25) -- (1.5,0.25);
            \draw[#4] (0,0.5) -- (1.5,0.5);
            \foreach \u in {1,...,#1} {
                \filldraw[#4] (0.1*\u,0) circle (1pt);
            }
            \foreach \u in {1,...,#2} {
                \filldraw[#4] (0.1*\u,0.25) circle (1pt);
            }
            \foreach \u in {1,...,#3} {
                \filldraw[#4] (0.1*\u,0.5) circle (1pt);
            }
        }
    } 
}

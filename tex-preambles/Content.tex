%% GENERAL ADD ONS %%%%%%%%%%%%%%%%%%%%%%%%%%%%%%%%%%%%%%%%%%%%%%%%%%%%%%%%%%%%

\usepackage[dvipsnames]{xcolor}  %% Must come before Tikz etc.
\usepackage[many]{tcolorbox}
\usepackage{dcounter}
\usepackage{enumerate}
\usepackage{graphicx}
\usepackage[
    hyperindex=true,
    colorlinks=true,
    linkcolor=teal,
    citecolor=ForestGreen]{hyperref}
\usepackage{float}     %% given captions to listings and types.
\usepackage{caption}
\usepackage{subcaption}
% \usepackage{aeguill}  %% IF NEEDED INCLUDE IN Content.tex near top
\usepackage[light]{kpfonts} % Nice Adobe Font
\usepackage{pifont} % \xmark
    \newcommand{\cmark}{\ding{51}}%
    \newcommand{\xmark}{\ding{55}}%

%% TIKZ %%%%%%%%%%%%%%%%%%%%%%%%%%%%%%%%%%%%%%%%%%%%%%%%%%%%%%%%%%%%%%%%%%%%%%%
\usepackage{tikz}
\usepackage{tikz-cd}
    \usetikzlibrary{positioning,arrows,backgrounds}
%%%%%%%%%%%%%%%%%%%%%%%%%%%%%%%%%%%%%%%%%%%%%%%%%%%%%%%%%%%%%%%%%%%%%%%%%%%%%%%

%% MATH %%%%%%%%%%%%%%%%%%%%%%%%%%%%%%%%%%%%%%%%%%%%%%%%%%%%%%%%%%%%%%%%%%%%%%%

\usepackage{amsmath,amsthm,amssymb}

\newcommand{\bigforall}{\mbox{\Large $\mathsurround0pt\forall$}}
\newcommand{\bigexists}{\mbox{\Large $\mathsurround0pt\exists$}}


\usepackage{bussproofs}  %% proof diagrams
\usepackage{wasysym} % \Leftcircle
\usepackage{mathtools}  % left/right harpoons
% \usepackage{lscape}  %% Landscape pages.
% \usepackage{longtable}  %% tables that span 2 pages

%%%%%%%%%%%%%%%%%%%%%%%%%%%%%%%%%%%%%%%%%%%%%%%%%%%%%%%%%%%%%%%%%%%%%%%%%%%%%%%

\usepackage{pdfpages}
% \usepackage{zref-abspos} % Used to know where digress/focus starts/ends

%% STYLE CHAPTERS %%%%%%%%%%%%%%%%%%%%%%%%%%%%%%%%%%%%%%%%%%%%%%%%%%%%%%%%%%%%%
\makechapterstyle{invhangnum}{%
  \setlength\beforechapskip{0pt}%
  \renewcommand*\chapterheadstart{\vspace{\beforechapskip}}%
  \setlength\afterchapskip{2\onelineskip plus .2\onelineskip minus 0.2\onelineskip}%  
  \renewcommand\chaptitlefont{\Huge\bfseries}
  \setlength\midchapskip{-\baselineskip}%
  \renewcommand\chapnumfont{\Huge\bfseries}
  \renewcommand*{\printchaptername}{}
  \renewcommand*{\chapternamenum}{}
  \renewcommand*{\printchapternum}{%
      \raisebox{\dimexpr\midchapskip+\baselineskip\relax}[0pt][0pt]{%
        \makebox[0pt][l]{%
        \makebox[\dimexpr\textwidth+4em\relax][l]{%
          \parbox[t]{\textwidth}{\mbox{}}%
          \parbox[t]{4em}{\hfill\chapnumfont \thechapter}}}}}%
  \renewcommand*{\printchaptertitle}[1]{%
    \raisebox{\dimexpr\midchapskip+\baselineskip\relax}[0pt][0pt]{%
      \parbox[t]{\textwidth}{\raggedright\chaptitlefont ##1}}}%
}
\chapterstyle{invhangnum}
\newlength{\digresslen}
% \chapterstyle{hangnum}
% \hangsecnum

    \usepackage{eso-pic}   %% Used to add gray strip to side of pages.
    %%%%%%%%% Digression margins.
    % \newenvironment{}{}{}
    \newcommand{\digresshere}[1]{
        % \newdimen\digheight
        % \setbox0=\vbox{#1}
        % \digheight=\ht0 \advance\digheight by \dp0
        % \zsavepos{#1-digress}
        % \write\mywrite{#1: \zposy{#1-digress}}%
        % \AddToShipoutPictureBG{%
        % \AtPageLowerLeft{%
        
            % \ifodd\value{page}
            % \hspace*{\dimexpr\paperwidth-3em}%
            % \fi
            % \settoheight{\digresslen}{\vbox{#1}}
        \marginpar{\hspace{\marginparwidth}\textcolor{black!25}{\rule{3em}{1in}}}%
        % }%
        % }%
        % \newcommand{\focus}{\ClearShipoutPictureBG}
    }

    \newcommand{\digress}{
        % \zsavepos{#1-digress}
        % \write\mywrite{#1: \zposy{#1-digress}}%
    \AddToShipoutPictureBG{%
        \AtPageLowerLeft{%
        %\raisebox{.25\paperheight}{%
            \ifodd\value{page}
            \hspace*{\dimexpr\paperwidth-3em}%
            \fi
            \textcolor{black!25}{\rule{3em}{\paperheight}}%
        %}%
        }%
        }
    }
    \newcommand{\focus}{\ClearShipoutPictureBG}
    \newcommand{\codemargin}[1]{
        \marginpar{
            \colorbox{black!25}{
                \parbox{0.9\marginparwidth}{
                    {\small \textsf{#1}}
                }
            }
        }
    }
    %% TOC Formatting
    % Indentation of numbers
    \setlength{\cftsectionindent}{2em}      % default 1.5em

    % Distance between numbers and titles
    \setlength{\cftpartnumwidth}{2.25em}      % default 1.5em
    \setlength{\cftchapternumwidth}{2em}      % default 1.5em
    \setlength{\cftsectionnumwidth}{3em}      % default 2.3em





\usepackage{imakeidx}
    \makeindex
    \indexsetup{headers={Index}{Index}} % Left/Right headers for the index

    \newcommand{\term}[1]{\emph{#1}}%\index{#1}}
%%%%%%%%%%%%%%%%%%%%%%%%%%%%%%%%%%%%%%%%%%%%%%%%%%%%%%%%%%%%%%%%%%%%%%%%%%%%%%%

%% TITLE PAGE %%%%%%%%%%%%%%%%%%%%%%%%%%%%%%%%%%%%%%%%%%%%%%%%%%%%%%%%%%%%%%%%%
\usepackage{epigraph}
\usepackage{titling}
\usepackage{bookman}
    \renewcommand\epigraphflush{flushright}
    \renewcommand\epigraphsize{\normalsize}
    \setlength\epigraphwidth{0.7\textwidth}

    % CSU Green 92, 18, 94, 61
    \definecolor{titlepagecolor}{cmyk}{.92,.18,.94,.61}
    % CSU Gold 11, 6, 64, 13
    \definecolor{titlepagecolor2}{cmyk}{.11,.06,.64,.13}
    %\definecolor{titlepagecolor}{cmyk}{1,.60,0,.40}

    \DeclareFixedFont{\titlefont}{T1}{ppl}{b}{it}{0.5in}
    % The following code is borrowed from: https://tex.stackexchange.com/a/86310/10898

    %%% STOLEN FROM 
    %%%% https://tex.stackexchange.com/questions/85904/showcase-of-beautiful-title-page-done-in-tex
    \newcommand\titlepagedecoration{%
    \begin{tikzpicture}[remember picture,overlay]%,shorten >= -10pt]
    \coordinate (aux1) at ([yshift=-15pt]current page.north east);
    \coordinate (aux2) at ([yshift=-410pt]current page.north east);
    \coordinate (aux3) at ([xshift=8cm, yshift=15cm]current page.south west);
    \coordinate (aux4) at ([yshift=-150pt]current page.north east);

    \node at (aux3) {\begin{tikzpicture}
        \node (e) at (0,0) {$\epsilon$};
        \node (a) at (0:1) {a};
        \node (b) at (120:1) {b};
        \node (c) at (240:1) {c};
        \node (aa) at (-40:2) {aa};
        \node (ba) at (0:2) {ba};
        \node (ca) at (40:2) {ca};
        \node (ab) at (80:2) {ab};
        \node (bb) at (120:2) {bb};
        \node (cb) at (160:2) {cb};
        \node (ac) at (200:2) {ac};
        \node (bc) at (240:2) {bc};
        \node (cc) at (280:2) {cc};
    
        \draw[thick,->,BrickRed] (e) -- (a);
        \draw[thick,->,PineGreen] (e) -- (b);
        \draw[thick,->,RoyalBlue] (e) -- (c);
    
        \draw[thick,->,BrickRed] (a) -- (aa);
        \draw[thick,->,PineGreen] (a) -- (ba);
        \draw[thick,->,RoyalBlue] (a) -- (ca);
    
        \draw[thick,->,BrickRed] (b) -- (ab);
        \draw[thick,->,PineGreen] (b) -- (bb);
        \draw[thick,->,RoyalBlue] (b) -- (cb);
    
        \draw[thick,->,BrickRed] (c) -- (ac);
        \draw[thick,->,PineGreen] (c) -- (bc);
        \draw[thick,->,RoyalBlue] (c) -- (cc);
    \end{tikzpicture}};
% \begin{scope}[titlepagecolor!40,line width=12pt,rounded corners=12pt]
%     \draw
%     (aux1) -- coordinate (a)
%     ++(225:5) --
%     ++(-45:5.1) coordinate (b);
%     \draw[shorten <= -10pt]
%     (aux3) --
%     (a) --
%     (aux1);
%     \draw[opacity=0.6,titlepagecolor2,shorten <= -10pt]
%     (b) --
%     ++(225:2.2) --
%     ++(-45:2.2);
%     \end{scope}
%     \draw[titlepagecolor,line width=8pt,rounded corners=8pt,shorten <= -10pt]
%     (aux4) --
%     ++(225:0.8) --
%     ++(-45:0.8);
%     \begin{scope}[titlepagecolor!70,line width=6pt,rounded corners=8pt]
%     \draw[shorten <= -10pt]
%     (aux2) --
%     ++(225:3) coordinate[pos=0.45] (c) --
%     ++(-45:3.1);
%     \draw
%     (aux2) --
%     (c) --
%     ++(135:2.5) --
%     ++(45:2.5) --
%     ++(-45:2.5) coordinate[pos=0.3] (d);   
%     \draw 
%     (d) -- +(45:1);
%     \end{scope}
    \end{tikzpicture}%
    }


%%%%%%%%%%%%%%%%%%%%%%%%%%%%%%%%%%%%%%%%%%%%%%%%%%%%%%%%%%%%%%%%%%%%%%%%%%%%%%%

%% STYLE POPOUTS %%%%%%%%%%%%%%%%%%%%%%%%%%%%%%%%%%%%%%%%%%%%%%%%%%%%%%%%%%%%%%

\usepackage{tcolorbox}
    \newenvironment{guide}[1]{
        \medskip
        \begin{center}
        \begin{minipage}[t]{0.95\linewidth} %{\dimexpr0.33\textwidth-2\fboxrule-2\fboxsep\relax}
            \begin{tcolorbox}[colback=gray!5,colframe=green!40!black,title=#1]
    }{
            \end{tcolorbox}
        \end{minipage}
        \end{center}
        \medskip
    }%
    \newenvironment{mywarning}[1] {
        \medskip
        \begin{center}
        \begin{minipage}[t]{0.95\linewidth} %{\dimexpr0.33\textwidth-2\fboxrule-2\fboxsep\relax}
            \begin{tcolorbox}[colback=gray!5,colframe=red!40!black,title=#1]
    }{
            \end{tcolorbox}
        \end{minipage}
        \end{center}
        \medskip
    }%
    \newenvironment{open}{
        \medskip
        \begin{center}
        \begin{minipage}[t]{0.95\linewidth} %{\dimexpr0.33\textwidth-2\fboxrule-2\fboxsep\relax}
            \begin{tcolorbox}[colback=gray!5,colframe=blue!40!black,title=Open Problem]
    }{
            \end{tcolorbox}
        \end{minipage}
        \end{center}
        \medskip
    }%
%%%%%%%%%%%%%%%%%%%%%%%%%%%%%%%%%%%%%%%%%%%%%%%%%%%%%%%%%%%%%%%%%%%%%%%%%%%%%%%





%% FONTS %%%%%%%%%%%%%%%%%%%%%%%%%%%%%%%%%%%%%%%%%%%%%%%%%%%%%%%%%%%%%%%%%%%%%%
% \usepackage{fontspec}
% \usepackage[T1]{fontenc}
% \usepackage{kpfonts}
% \usepackage[T1]{fontenc}

%% AUTHOR INDEX %%%%%%%%%%%%%%%%%%%%%%%%%%%%%%%%%%%%%%%%%%%%%%%%%%%%%%%%%%%%%%%

\newcommand{\Church}{\index{Church, Alonzo (1903--1995)}}
\newcommand{\Codd}{\index{Codd, Edgar F. (1923--2003)}}
\newcommand{\Coquand}{\index{Coquand, Thierry (1961--)}}
\newcommand{\Curry}{\index{Curry, Haskell (1900--1982)}}
\newcommand{\Frege}{\index{Frege, Gottlob (1884--1925)}}
\newcommand{\Hamilton}{\index{Hamilton, Sir William Rowan (1805--1865)}}
\newcommand{\Hopper}{\index{Hopper, Grace (1906--1992)}}
\newcommand{\Leibniz}{\index{Leibniz, Gottfried (1646--1716)}}
\newcommand{\MartinLof}{\index{Martin-Lof@Martin-L\"of, Per (1942--)}}
\newcommand{\Russell}{\index{Russell, Bertrand (1872--1970)}}
\newcommand{\Tarski}{\index{Tarski, Alfred (1901--1983)}}
\newcommand{\Turing}{\index{Turing, Alan (1912--1954)}}
\newcommand{\Whitehead}{\index{Whithead, Alfred North (1861--1947)}}
\newcommand{\ZF}{\index{Zermelo, Ernst (1871--1953)}\index{Fraenkel, Abraham (1891--1965)}}




%% FINAL SETTINGS %%%%%%%%%%%%%%%%%%%%%%%%%%%%%%%%%%%%%%%%%%%%%%%%%%%%%%%%%%%%%
\numberwithin{figure}{chapter}
\numberwithin{table}{chapter}

%% EDITING MARKS %%%%%%%%%%%%%%%%%%%%%%%%%%%%%%%%%%%%%%%%%%%%%%%%%%%%%%%%%%%%%%
\usepackage[draft]{pdfcomment}
\usepackage{xparse}
    \DeclareDocumentCommand \EDITmargin { o m } {
        \IfNoValueTF {#1} {
            \pdfmargincomment[icon=Note]{#2}
        }{
            \pdfmargincomment[icon=NOte,author=#1]{#2}
        }
    }
    \DeclareDocumentCommand \EDITcomment { o m } {
        \IfNoValueTF {#1} {
            \pdfcomment[color=Blue!20]{#2}
        }{
            \pdfcomment[color=Blue!20,author=#1]{#2}
        }
    }
    \DeclareDocumentCommand \EDITalt { o m m } {
        \IfNoValueTF {#1} {
            \pdfmarkupcomment[markup=StrikeOut]{#2}{#3}
        }{
            \pdfmarkupcomment[markup=StrikeOut,icon=key, author=#1]{#2}{#3}
        }
    }
    \DeclareDocumentCommand \EDITtypo { o m o } {
        \IfNoValueTF {#3} {
            \pdfmarkupcomment[markup=Squiggly,author=#1]{#2}{}
        }{
            \pdfmarkupcomment[markup=Squiggly,author=#1]{#2}{#3}
        }
    }
    \DeclareDocumentCommand \EDIThighlight { o m m } {
        % \IfNoValueTF {#3} {
        %     \pdfmarkupcomment[markup=Highlight,author=#1,color=blue!20]{#2}{}
        % }{
            \pdfmarkupcomment[markup=Highlight,author=#1,color=blue!20]{#2}{#3}
        % }
    }


    
%%%%%%%%%%%%%%%%%%%%%%%%%%%%%%%%%%%%%%%%%%%%%%%%%%%%%%%%%%%%
%% Theorems
\usepackage{thmtools}
\declaretheoremstyle[
    shaded={
        rulecolor=black!30,
        rulewidth=1pt,    
        bgcolor=white
    }
]{thm}
\declaretheorem[numberwithin=chapter, style=thm]{theorem}
\declaretheorem[sibling=theorem, style=thm]{lemma}
\declaretheorem[sibling=theorem, style=thm]{proposition}
\declaretheorem[sibling=theorem, style=thm]{corollary}


\declaretheoremstyle[
    shaded={
        rulecolor=purple!30,
        rulewidth=1pt,    
        bgcolor=white
    }
]{eg}
% \declaretheorem[sibling=theorem, style=eg]{ex}
\declaretheorem[sibling=theorem, style=eg]{example}

\declaretheoremstyle[
    shaded={
        rulecolor=teal!20,
        rulewidth=1pt,    
        bgcolor=white
    }
]{def}
\declaretheorem[sibling=theorem, style=def]{definition}

\usepackage{tcolorbox}

%% Remark box
\newtcolorbox[auto counter,number within=chapter]{remark}[2][]{
  enhanced,
  breakable,
  fonttitle=\scshape,
  sharp corners,
  colframe=gray,
  colback=White,
  title={Remark~\thetcbcounter~#2},
  #1
}

% \declaretheoremstyle[
%     shaded={
%         rulecolor=ForestGreen,
%         rulewidth=1pt,    
%         bgcolor=white
%     }
% ]{rem}
% \declaretheorem[sibling=theorem, style=rem]{remark}

\declaretheorem[sibling=theorem, name=Engineering Note, style=rem]{implremark}
\declaretheorem[sibling=theorem, name=Historical Remark, style=rem]{histremark}

%-----------------------------------------------------
%       Standard theorem like environments.
%-----------------------------------------------------
%% \theoremstyle{plain} %% This is the default
\numberwithin{equation}{chapter}
%\newtheorem{theorem}[equation]{Theorem}
%\newtheorem*{theorem*}{Theorem}
%\newtheorem{lemma}[equation]{Lemma}
%\newtheorem{proposition}[equation]{Proposition}
\newtheorem{prob}{}[chapter]

%\newtheorem{lem}[equation]{Lemma}
%\newtheorem{corollary}[equation]{Corollary}
%\newtheorem*{coro*}{Corollary}
\newtheorem{quest}[equation]{Question}

\theoremstyle{remark}


\theoremstyle{definition}
%\newtheorem{definition}[equation]{Definition}
%\newtheorem{ex}{Example}[equation]
% \newtheorem*{remark*}{Remark}
% \newtheorem{remark}[equation]{Remark}
% \newtheorem{progrem}[equation]{Programing Remark}
% \newtheorem{histrem}[equation]{Historical Remark}



%%%%%%%%%%%%%%%%%%%%%%%%%%%%%%%%%%%%%%%%%%%%%%%%%%%%%%%%%%%

\newcommand{\plusplus}{{\tiny ++}}

\newcommand{\tsspace}{\mathsf{space}~}
\newcommand{\tsaxes}{\mathsf{axes}~}
\newcommand{\tsframe}{\mathsf{frame}~}
\newcommand{\tsbase}{\mathsf{base}~}
\newcommand{\tsinterp}{\mathsf{interp.}~}


\newcommand{\NamedTensorSpace}[6]{
\begin{array}{lllll}
    #1 & \defeq \mathsf{TensorSpace}\big(
        &\tsspace & #2,\\
&        &\tsaxes & #3,\\
&        &\tsframe & #4,\\
&        &\tsbase & #5,\\            
&        &\tsinterp & #6 ~~~\big)        
    \end{array}
}
\newcommand{\TensorSpace}[5]{
\begin{array}{lll}
\mathsf{TensorSpace}\big(
        &\tsspace & #1,\\
        &\tsaxes & #2,\\
        &\tsframe & #3,\\
        &\tsbase & #4,\\            
        &\tsinterp & #5 ~~~\big)        
    \end{array}
}
\newcommand{\InlineTensorSpace}[5]{
$\mathsf{TensorSpace}($ 
    $\tsspace #1$, 
    $\tsaxes #2$,
    $\tsframe #3$,
    $\tsbase #4$,
    $\tsinterp #5$)
}
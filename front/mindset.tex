
Algebra today is taught by example. First groups, then rings,
then fields, then modules, then bigger groups, then crazier non-associative
rings, then groups and rings with outside influences like topology, analysis,
and geometry. Today there are few general-practice algebraist.  There are
instead geometric group theorist, algebraic geometers, commutative ring
theorist, representation theorist, non-associative algebraist, computational
algebraist (that's me), and even those topics are too general for any one
theorist to master. Its acceptable because we still know one another and can ask
a specialist when we do not understand. 

The problem is that our algebra textbooks are getting thicker, some split into
multiple volumes. We have not covered an entire text for decades and few  
students can familiarize themselves with the entire book in one year.
One outcome has been to emphasize the good-old-days of successful theory from the
1800's and early 1900's.  Sylow's theorems, Wedderburn-Artin, Hilbert's
Nullstallensatz, Krull-Schmidt, Jordan-H\"older, Frobenius reciprocity. But a
lot a has happened in the century since.  Another approach has been for teachers to
pick their favorites making each course different form the next.  
Too many of us follow trends and fashions for this to be reliable (rebellion of 
a trends is a trend in its own right).

What if algebra were not taught by examples?  What if we taught its
underlying structural methodology and saw examples as reinforcing the theory
not being the theory.  What if instead of proving Krull-Schmidt for modules and
saying ``the same idea applies for groups and rings'', we just proved a theorem
that applied to modules, groups, and rings?  There is no harm in repetition and
useful examples.  I understand why we skip details of homomorphisms of modules
or Lie algebras because by the time students see these, they have seen enough
examples of that concept.  Yet, that behavior goes very wrong very quick.  Ask
your students (ask yourself) What is the kernel of a homomorphism of semigroups?
Monoids?  Homomorphism on $\mathbb{N}$?  Basic as these are, experience 
with groups and rings will not answer these riddles.
When we use our models instead of our theory 
we loose something.  

\begin{center}
    \textbf{Abstract algebra is not abstract enough.}
\end{center}
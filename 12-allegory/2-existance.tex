\section{Algebras exist}

It will surprise no one that a grammar that talks about math ends up having 
something to do with math.  But what about grammars that talk about 
programming languages, or contrived facsimiles of natural language?
Do these deserve to be included in the cannon of algebra?  Yes.

\begin{theorem}
    For every (unambiguous) context-free grammar $\sigma=(\Sigma,\mathcal{N},\mathcal{P},\mathcal{S})$
    with $\mathcal{S}=\mathcal{N}$, the multi-language it 
    accepts together with the production rules interprets a $\sigma$-algebra.
\end{theorem}
\begin{proof}
    For each tag $A:\mathcal{N}$, as $A:\mathcal{S}$ there is a 
    language $\mathcal{L}(A)$ of accepted strings.  Define 
    $\bar{A}\defeq \mathcal{L}(A)$.
    For each production $(A,w):\mathcal{P}$, define  
    \begin{align*}
        ops_{(A,w)}(s_1,\ldots,s_{\ell}) & \defeq s_1\cdots s_{\ell}
    \end{align*}
    As each term $s_i:\bar{w_i}$, $s_1\cdots s_{\ell}$ is a string matching
    the grammar of $w$ and accepted by this production rule into $\mathcal{L}(A)$.
    That is, 
    \begin{align*}
        \frac{
            s_1:\bar{w}_1,\ldots,s_{\ell}:\bar{w}_{\ell}\vdash s_1\cdots s_{\ell}:\mathcal{L}(A)
        }{
            (s_1,\ldots, s_{\ell})\mapsto s_1\cdots s_{\ell}
            :\prod_{i=1}^{\ell}\bar{w}_i\to \bar{A}.
        }
    \end{align*}
\end{proof}

\begin{theorem}
    For every type $X$, the formula language $F_{\sigma}\langle X\rangle$ is a $\sigma$-algebra.
\end{theorem}



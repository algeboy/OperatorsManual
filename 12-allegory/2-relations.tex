
\section{Relations}
Imagine you want to explain linear algebra.  You work up to a concept of 
$\mathbb{Z}$-module, each is a set $V$ with some operations $+$ and $\cdot$. You would likely
consider subspaces, $U\leq V$, which you dutifully show to be subsets that are
closed to $+$ and $\cdot$ and $0$. You show the person linear mappings $f:V\to
W$---those are linear maps that preserve $+$ and $\cdot$.  You might even dabble
with quotients, needing an equivalence relation $u\equiv v\pmod{K}$ for which
$+$ and $\cdot$ are well-defined.  There is a lot wrapped up in this already and
them come many more questions.  What about a subspace of a subspace.  What about
quotient of a subspace? What happens when you apply a linear map to a quotient
of as subspace? With something this well-known it is surprising how many
branches of abstraction we seem to need to follow to build and understanding.
But perhaps the fact that it all holds together is a sign of a simpler process
behind it all. Perhaps, congruence relations, linear maps, subspace and more are
really one unified concept.


Consider how we define the concepts.  Lets look first at just the addition.
\begin{align*}
    U\subset V\text{ is a subspace} & \Leftrightarrow (\forall u,\acute{u}\in U)(u+\acute{u}\in U)\\
    f:U\to V\text{ is linear} & \Leftrightarrow (\forall u,\acute{u}\in U)(f(u+\acute{u})=f(u)+f(\acute{u}))\\
\end{align*}






Fix a context-free grammar $\sigma$ which consists of terminal symbols 
we think of as constants, and non-terminal symbols we think of as operators
along with production rules.


Given an algebra $A$ modeling the theory,
an \emph{algebraic relation} is function $P:\prod_{i:I} A_i\to \Prop$ if 
\begin{align*}
    \begin{array}{cc}
        & P(a_{11}, \cdots, a_{1m})\\
        & \vdots \\
        \langle\cdots\rangle & P(a_{m1}, \cdots, a_{mn})\\
    \hline
       & P(\langle a_{*1}\rangle,\ldots \langle a_{*n}\rangle)
    \end{array}
\end{align*}

\subsection{Heteromorphisms}


\section{Substructure}
A function $P:\prod_{i:I}(A_i\to \Prop)$ describes a substructure.
\begin{center}
    \begin{tikzcd}
        \prod_{i:I}A_i \arrow[r, "\langle\cdots\rangle"] & A_0\\
        \prod_{i:I}\bigsqcup_{a_i:A_i}P(a_i) \arrow[u,hook]\arrow[r,"closed"] 
            & \bigsqcup_{a_0:A_0}P_0(a_0)\arrow[u, hook]
    \end{tikzcd}
\end{center}
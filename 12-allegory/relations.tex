\section{What do we want in a relationship?}

The method of algebra is to maintain balance.
$2=5$ is unbalanced, so to fix it we add whatever 
is needed, $x+2=5$ restores the balance.  Now 
that balance exists we want to keep it, but we'd 
also like to discover what $x$ really is.  
We add $-2$ to both sides
\begin{align*}
    (x+2)+(-2) & = 5+(-2)\\
    x+(2+(-2)) & = 3\\
    x+0 & = 3\\
    x & =3.
\end{align*}
This much we learned from Al Khawarismi, the inventer,
or at least chief cataloger, of al-jaber, the method of balance
or what today we call algebra.

So then the point of the equality relation is to preserve balance 
as we operate.  So many things we have built since then do so, 
though none quite as good as $=$.  Suppose we use an 
equivalence relation $EQ_k(m,n)\defeq (m\equiv n \pmod{k})$.
\begin{align*}
    \begin{array}{crlc}
        & m & \equiv  n & \pmod{541}\\
    \wedge    & \acute{m} & \equiv  \acute{n} & \pmod{541}\\
    \hline
    & m+\acute{m} & \equiv n+\acute{n} & \pmod{541} 
    \end{array}
    & & 
\begin{array}{cl}
        & EQ_{541}(m,n)\\
    \wedge & EQ_{541}(\acute{m},\acute{n})\\
    \hline 
        & EQ_{541}(m+\acute{m},n+\acute{n})
\end{array}
\end{align*}

We can use an ordering $O(a,b)\defeq a\leq b$ on 
decimal numbers.
\begin{align*}
    \begin{array}{crl}
        & a & \leq b\\
    \wedge    & \acute{a} & \leq \acute{b}\\
    \hline 
        & a+\acute{a} & \leq b+\acute{b}
    \end{array}    
    & & 
\begin{array}{cl}
        & O(a,b)\\
    \wedge & O(\acute{a},\acute{b})\\
    \hline 
        & O(a+\acute{a},b+\acute{b})
\end{array}
\end{align*}

Next let try the graph relation $G_f(u,v)\defeq (f(u)=v)$
and consider group homomorphisms.
\begin{align*}
\begin{array}{crl}
        & f(u) & = v\\
    \wedge & f(\acute{u}) & =\acute{v}\\
    \hline 
        & f(u\acute{u}) & = v\acute{v}
\end{array}
    & & 
\begin{array}{cl}
        & G_f(u,v)\\
    \wedge & G_f(\acute{u},\acute{v})\\
    \hline 
        & G_f(u+\acute{u},v+\acute{v})
\end{array}
\end{align*}
At this point you might draw a conclusion that many other have as 
well.  One useful relation for algebra is a binary relation $R(a,b)$
with the property that there is a function 
\begin{align*}
    cong:R(a,b)\times R(\acute{a},\acute{b})\to R(a\acute{a},b\acute{b}).
\end{align*}
The long-term minded may even realize we can relax the binary assumption 
which was focussed only on binary operations.  We could use an 
$m$-valent operator $\langle\cdots\rangle$ and an $n$-valent relation $R$.
\begin{align*}
    \begin{array}{cc}
        & R(a_{11},\ldots,a_{1n})\\
        & \cdots \\
    \wedge    & R(a_{m1},\ldots,a_{mn})\\
    \hline 
        & R(\langle a_{*1}\rangle,\ldots,\langle a_{*n}\rangle)
    \end{array}
\end{align*}
Though a powerful extension of any theory, this approach remains on the 
whole homogeneous and misses out even on natural common techniques form 
calculus.

The story overall is no complete without much more flexibility.
Continuing with the graph relation we find some differences
\begin{align*}
    \begin{array}{crl}
        & \delta(u) & =v\\
    \wedge &\delta(\acute{u}) & =\acute{v}\\
    \hline 
        & \delta(u*\acute{u}) & =v*\acute{u}+u*\acute{v}
    \end{array}
&&
    \begin{array}{cl}
        & G_{\delta}(u,v)\\
    \wedge & G_{\delta}(\acute{u},\acute{v})\\
    \hline 
        & G_{\delta}(u*\acute{u},v*\acute{u}+u*\acute{v})
    \end{array}
\end{align*}

What do we really need then?  The main idea is to be able to do 
algebra in one component of a relation and have a prediction of 
what happens in other components.  For that work uniformly, we need 
that there are polynomials to apply to the available constants to fill 
into each component of the relation.
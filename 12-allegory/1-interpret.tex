\section{Models interpreted in a language}
\index{models}
We know grammar lets us write down algebra and its formulas but so 
far we have built much of the algebra we see around us.  Formulas 
are not the stuff of numbers.  So the idea now is to evaluate the 
formulas on other data. The idea is to assign data types to a 
bunch of the tag names of the grammar and then use the production 
rules to identify the operations we need between the types.

So let us begin with a formal context-free grammar $\sigma$.
We assume its alphabet is $\Sigma$, its tag names are $N$, 
and its production rules are $P\subset N\times (\Sigma\sqcup N)^*$.
We will begin with the starting rules being $S=N$ and later remark on 
how to address the case when $S$ is a proper subset of $N$.

First we need to assign to every tag-name a type.  We can see this 
as a function.
\begin{align*}
    base:N\to \Type
\end{align*}
The person describing a model of the algebra is responsible for 
creating this assignment.  Next we need to extend this function 
to work on $N\sqcup \Sigma$ but in a some-what trivial manner 
meant mostly to name things not to demand new types.
\begin{align*}
    \overline{base}(x) & \defeq \begin{cases}
        base(n) & x=\iota_N(n)\\
        \{a\} & x=\iota_{\Sigma}(a)
    \end{cases}
\end{align*}
Next we can extend $\overline{base}$ to apply to all 
strings of $(N\sqcup \Sigma)^*$ by replacing juxtaposition 
in the string with the cartesian product of types.  Doing so 
we now see how to align each production rule with an appropriate 
operator.  We assume 
our algebra comes with an assignment
\begin{align*}
    ops:\prod_{(A,w):P}\prod_{t\in w}\overline{base}(t)\to base(A).
\end{align*}


\begin{example}
    Fix the grammar \code{<<Nat>> ::= 0 | S <Nat>}.
    Then define 
    \begin{align*}
        base & : \{Nat\}\to \Type &
        base(Nat) & \defeq \mathbb{C}[x]\\
        ops_0 & : \{0\}\to \mathbb{C}[x] &  ops_0(0) & \defeq 1\\
        ops_S & : \{S\}\times \mathbb{C}[x]\to \mathbb{C}[x] &  ops_S(S,a(x)) & \defeq a(x)x.
    \end{align*}
    Thus $\{1,x,xx,\ldots, x^n,\ldots\}$ is interpreted in the language 
    of natural numbers.
\end{example}

As a convention going forward instead of explicitly naming the function $base$
we can instead write $\bar{A}\defeq base(A)$ for any $A:N$.  We also notice that 
the constants do not affect the definition of any function and they can be 
omitted, but including them can have the effect of removing ambiguity and thus 
maintaining a bijection with the original production rules.


Now if we consider a more complicated grammar we may have several production rules 
meant to aid in explaining the language.  In fact it is an often sought after 
quality of grammars to add production rules to reduce the complexity of understanding 
what strings are accepted.  For context-free grammars one can achieve a form 
where every production rule is empty \code{<A>::=}, uni-valent \code{<A>::= t}
for a character $t:\Sigma$, or bi-valent \code{<A>::= <B><C>}. This is 
known as Chompsky Normal Form.\index{Chompsky Normal Form}
We achieve this by adding new production rules to break apart ones of higher 
valence.  For example
\begin{center}
\begin{Gcode}[]
<<A>> ::= [<A>,<A>,<A>]
\end{Gcode}
\end{center}
could become
\begin{center}
\begin{Gcode}[]
   <Third> ::= ,<A>
     <Mid> ::= <A><Third>
  <Second> ::= ,<Mid>
<Interior> ::= <A><Second>
    <Open> ::= <Interior>]
     <<A>> ::= [<Open>
\end{Gcode}
\end{center}
As a result of intermediate tags we cannot generally assume every tag 
is a start symbol.  These other tags are therefore used in the construction 
of the symbols we want.  So in an interpretation of model in a language 
we would like to think that the start symbols need actual data types 
and the further auxillary tags are merely there as helpers.  For example,
if we encounter the grammar \code{<<A>> ::= [<A>,<A>,<A>]}
we seem to have need of a single type $A$ and tri-valent function 
$[-,-,-]:A^3\to A$.  Yet when we are handed this same grammar in Chompsky 
Normal Form we suddenly loose track of the original tri-valent map 
in favor of a long list of bi-valent maps chained together.
\begin{align*}
    & \{,\}\times \bar{A}\to \overline{Third}\\
    & \bar{A}\times \overline{Third}\to \overline{Mid}\\
    & \{,\}\times \bar{Mid}\to \overline{Second}\\
    & \bar{A}\times \overline{Second}\to \overline{Interior}\\
    & \overline{Interior}\times \{]\}\to \overline{Open}\\
    & \{[\}\times \overline{Open}\to \bar{A}\\
\end{align*}
The single start symbol exposes that we cannot use each of the operations 
independently in the interpretation.  Instead we much chain them together as follows.
\begin{align*}
    & \{[\}\times 
        \overline{Open}
    \to \bar{A}\\
    & \{[\}\times 
        \overline{Interior}
    \times \{]\}\to \bar{A}\\
    & \{[\}\times \bar{A}\times 
        \overline{Second}
    \times \{]\}\to \bar{A}\\
    & \{[\}\times \bar{A}\times  \{,\}
        \times \overline{Mid}
    \times \{]\}\to \bar{A}\\
    & \{[\}\times \bar{A}\times  \{,\}\times \bar{A}\times 
        \overline{Third}
    \times \{]\}\to \bar{A}\\
    & \{[\}\times \bar{A}\times  \{,\}\times \bar{A}\times \{,\}\times \bar{A}
    \times \{]\}\to \bar{A}.
\end{align*}
So we see that we can match the original intent of a single tri-valent operator 
by interpreting all the productions as partial evaluations of the 
tri-valent operotor.

\begin{definition}
    An \emph{interpretation} of a formal context-free grammar $\sigma=(\Sigma,N,P,S)$ 
\end{definition}

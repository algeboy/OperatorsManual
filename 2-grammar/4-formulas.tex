\chapter{Formulas}
While natural numbers and strings certainly feature throughout mathematics, 
our claim about algebra is that it lets us study equations.  Equations 
with variables are the only really interesting cases.  So we need 
strings with variables, but also a mix of operations like $+$, $\cdot$ and 
so on.

\section{Introducing Formulas}
It is time to add more sorts of data.  For a change of scene let us look 
at Boolean (true/false) algebra.  Here we have just two ``numerical''
symbols true $\top$,
false $\bot$, but a number of operators: and $\wedge$, or $\vee$, and not $\neg$.
The language might be defined
like this.
\begin{Gcode}[]
<Var>  ::= x | x_<Nat>
<Bool> ::= <Var>
        | $\top$
        | $\bot$
        | $\neg$ <Bool> 
        | <Bool> $\vee$ <Bool> 
        | <Bool> $\wedge$ <Bool>.
\end{Gcode}
Some terms this grammar will except include 
\begin{center}
    $\top \wedge \bot :Bool$, 
    $\neg (x_1\wedge x_2\vee x_3):Bool$, \&
    $\neg(x_1\wedge \neg x):Bool$.
\end{center}
Notice $\top\wedge \bot$ is a Boolean, even though it is not what we might 
consider to be ``true''.   The type of a string is whether it obeys the grammar,
only later do we investigate if this string has any meaning to an applied task.
Notice we the a single letter $x$ is given a different type than $\top,\bot$
hinting at our eventual use of $x$ as a variable.  We even allow our variables 
to take along a natural number so that we can have as many variables as our situation 
should require.  Of course for know this is just a grammar.  It will take using 
the formula to reveal how $x$ behaves like a variable.  At present to be 
a variable only means it is of type ``variable''.



Notice we have clarified in the grammar how 
constants $\top,\bot$ can appear on their own, variables can too.  
Keeping these separate means we can easily swap out our meaning for either.
For example, some boolean systems only allow a fixed number of variables, say 
$x$ and $y$, we would simply adjust to use 
\begin{center}
\code{<Var> ::= x | y}
\end{center}
Also in some applications the constants of truth also have to change.
Intuitionistic logic admits the possibility that that there may be 
things unknowable, so we add to $\top,\bot$ a symbol $?$,
\begin{center}
\begin{Gcode}[]
<Const> ::= $\top$ | $\bot$ | ?
\end{Gcode}
\end{center}
Fuzzy logic considers tolerance and thresholds to accommodate real life uncertainty, 
e.g.\ ``If the bridge is made of 85\% pure steel, then it is 99\% likely to last 50 years.''
Here we may replace \code{<Const>} with a type for intervals [0,1].
In temporal logic, the value of true/false changes with time, so the 
constants have a time parameter.  Modal logic is when truth values 
switch between modes, e.g.\ ``In the day solar power runs the washer; at night 
batteries do.''.  So modes may feature in \code{<Const>}.


\begin{remark}{Different logic can be better}
    There are ways to relate one class of logic to another. 
    For example, G\"odel proved that intuitionistic logic embeds
    in the negative into classical logic.  So you 
    might come away from that saying ``classical logic can do 
    intuitionistic logic.''   As a caution against this perspective note that every integer is 
    a complex number, but we learn a great deal more about integers when we 
    study them with number theory then simply investigating them 
    through complex analysis. 
    % Resist the temptation to toss  
    % aside other logic as special cases of classical logic, less you end up missing 
    % the point.
\end{remark}


Another familiar algebra concerns our customary polynomials.  We will use 
the type \code{<Reals>} describe in a later exercise as the coefficients.
\begin{Gcode}[]
    <Var>  ::= x 
    <UniPolyReals> ::= <Var>
            | <Reals>
            | <UniPolyReals>+<UniPolyReals>
            | (<UniPolyReals>)(<UniPolyReals>)
\end{Gcode}
As is typically  we write $\mathbb{R}[x]$ in place 
of \code{UniPolyReals}.  So this language accepts formulas like these 
\begin{center}
    $1.5+3.4(x)(x):\mathbb{R}[x]$,
    $(x+2)(x-2):\mathbb{R}[x]$, \&
    $12.9:\mathbb{R}[x]$.
\end{center}
This is the minimal requirement for real polynomials.
For example, in this form $x^2+2$ would not be accepted but 
$(x)(x)+2$ would be, and $3x$ would be rejected but $(3)(x)$ would 
be allowed.  We can add extra production rules to accommodate these 
such as \code{<Var>^<Nat>} and \code{<Reals> <Var>} (see exercises).  However, 
these will add more cases to any subsequent elimination of these formulas,
and they introduce ambiguity.  For example, is $x^2$ the same as $(x)(x)$?
It should be but the grammar wont force this.

\begin{remark}
    In point of fact some algebra are not what is known as 
    \emph{power associative} in that $x^n$ is not well-defined 
    on account that we need to know the grouping.  For instance,
    $x(xx)$ need not equal $(xx)x$.

    When a program allows for shortcuts like $3x$ and $x^3$, it is 
    implicitly introducing rules into the algebra which may not be true 
    for all situations.
\end{remark}

Suppose we want to move to polynomials in $x$ and $y$.  Have factored out 
the variables as a separate alphabet we could modify the productions of 
\code{<Var>} leaving the rest unchanged, expect perhaps to rename 
\code{<UniPolyReals>} to \code{PolyReals}.
\begin{Gcode}[]
    <Var>  ::= x | y
    <PolyReals> ::= <Var>
            | <Reals>
            | <PolyReals>+<PolyReals>
            | (<PolyReals>)(<PolyReals>)
\end{Gcode}
We could go one further and alter the use of real numbers to be some other 
kind  of number.  We will call these numbers \code{<Ring>} for now but its 
meaning will be variable until we fixate on evaluation polynomials.
So we now arrive at the general meaning of a polynomial.
\begin{Gcode}[]
    <Poly> ::= <Var>
            | <Ring>
            | <Poly>+<Poly>
            | (<Poly>)(<Poly>)
\end{Gcode}



To formalize the process we have begun we are dealing first in defining types, 
Natural numbers, Strings, \& Booleans.  Now we are engaged in finding common rolls 
amongst the components of these types, e.g.\ constants and variables.
Operators too are set apart in how we use them in the grammar, they are another sort.  They 
need the company of other booleans, or other polynomials.
To designate this division of labor we introduce a new term \emph{sort}.\index{sort}
So we are making a hierarchy to organize our data:
\begin{center}
    terms have types;
    types have sorts.
\end{center}
As with any formalized term there will be examples that blur the boundaries and force 
to revisit the sorting.  Having sorts simply lets us express the difference as we 
find them, it does not in any meaningful way map out a reality.

\subsection{Eliminating formulas}

Elimination of a natural number was to work through the tower of successors, 
but with the introduction of operators we can now depend on many terms.  This 
is a form of induction with possibly many bases cases.  Our experience with 
evaluation polynomials offers us a great advantage in guessing how to use these steps.
We simply draw the parse tree and begin replacing variables with their assigned
quantities.  We also replace any operator symbols with the algorithm that performs 
that operation.

For Boolean algebras the operators may be given by a truth table.
\begin{align*}
    \phi \vee \tau \defeq \begin{cases}
        \top & \phi=\top\\
        \top & \phi=\bot, \tau=\top\\
        \bot & \phi=\bot, \tau=\bot
    \end{cases}
\end{align*}

\begin{quote}
% As my philosophy  colleague Professor Dustin Tucker says, 
\emph{A system studied long enough reveals its paradoxes.}\\
~\hfill-- Dustin Tucker
    % \emph{The power of algebra is that every symbol in an equation is a 
    % variable...}
\end{quote}
The point of a variable is to replace it.  If we believe equations 
in algebra are fully variables then this replacement idea ought
to be quite sound before we go further.  Let us check that we understand.
Suppose I define $I$ and $K$ as follows.
\begin{align*}
    I(x) & = x & 
    K_c(x) & = c.
\end{align*}
You may call $I$ the identity function and the $K$ constant functions,
a different one for each $c$.\footnote{Functions in this sense are so primitive 
they have no domains and codomains.  You can put anything into these functions.}
Try some substitutions, I tried $I(3)=3$, $I(\clubsuit)=\clubsuit$.
I found $K_3(2)=3$ and $K_3(\clubsuit)=3$ as well.  I even tried 
$K_{\clubsuit}(2)$ and got $\clubsuit$.  I changed $x$ for $y$, 
$I(y)=y$, and $d$ for $c$, $K_d(x)=d$. Next I substituted $x$ for $c$ and got
\begin{align*}
    K_x(x)=x=I(x).
\end{align*}
Now we have a true problem: a constant function should not equal 
an identity function.  This is the \emph{paradox of the trapped (or captured) variable}.

Paradoxes (para = distinct + dox = opinion) are places where logical 
reasoning leads to two seemingly opposing conclusions.  We could just 
as well call this an inconsistency and declare the topic dead, but usually 
mathematics reserves the word paradox for settings where we could avoid the 
inconsistency by revisiting some fundamental notion and choosing a narrower 
interpretation.  True inconsistency we reserve for cases where our only option 
is to throw out the system.  For example, we could attempt to withhold some options that thus prevent one of the two opinions to surface.  It is not a philosophically satisfying resolution perhaps,
which is why most paradox hacks lead to schisms. 

So what is the root cause of our paradox of the trapped variable?
The answer are bound (local) verses free variables. 

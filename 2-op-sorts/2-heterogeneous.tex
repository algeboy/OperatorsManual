
\section{Heterogeneous operators}
Multiplication however is often more subtle and 
could require a more explicit grammar.  Think of rescaling a vector, we need 
different types of data so our scheme changes.
\begin{center}
% \begin{minipage}{0.6\textwidth}
%     \centering
%     \begin{Gcode}[]
%     <Vec> ::= <Scale> <Vec> 
%     \end{Gcode}
    $\Box = \bigcirc \Box$
    % \end{minipage}    
\end{center}    
This is no longer one data type.  To put words to it we have here a \emph{heterogeneous}
operator whereas the additions we explored before were each \emph{homogeneous}.\index{heterogeneous}\index{homogeneous}

Addition is held to high standards in algebra
(that it will evolve into linear algebra).  So when you are considering a bivalent
operation with few if any good properties, use a multiplication inspired
notation instead.   


\index{bivalent!opeator}\index{binary operator|see{bivalent operator}}
Addition from here on out will be bivalent (also called binary), meaning 
that it requires 2 inputs.  We typically prefer infix grammar $\Box +\Box$.




\index{univalent!opeator}\index{unary operator|see{univalent operator}} Valence
1, also called \emph{univalent} or \emph{unary}, operators include the negative
sign $-\Box$ to create $-2$ as well as the transpose $A^{\dagger}$ of a matrix
$A$. Notice in the case of negative an integer remained an integer, but in the
case of transpose a $(2\times 3)$-matrix becomes a $(3\times 2)$-matrix.
Operators can change the type of data we explore.
 
Programming languages add several others univalent operators
such as \code{++i, --i} which are said to \emph{increment} 
or \emph{decrement} the counter i (change it by $\pm 1$).

Programs also exploit a trivalent (ternary) operator:
\begin{center}
\begin{Pcode}[]
if (...) then (...) else (...)
\end{Pcode}
\end{center}
The words, while helpful, are unimportant and some languages
replace it with symbols emphasizing it is an operator:
\begin{center}
    \pcode{_?_:_}
\end{center}
% Here for example is division with remainder of positive integers
% \begin{center}
% \begin{Pcode}[]
%     div(m,n)=(m>=n)?(div(m-n,n)+(1,0)):(0,m)
% \end{Pcode}
% \end{center}

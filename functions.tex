
\chapter{Equality}

One day  $+$ means to add natural numbers, the next day 
polynomials, or matrices.   You can even add colors ``Yellow=Blue+Green''.
You can add music, well at least you can play two songs at the same time 
and decide if you want to call it music.  In fact 

The + is there as a notation to explain 
there is a pattern at play that you have seen before.  All additions 
come in pairs for instance.  Its the signature of addition to write 
the pair as $x+y$.  Beyond that there are many reasonable assumptions 
you might make with the presence of $+$.  You probably suspect 
that any $+$ can be done in any order: Green+Blue also equals Yellow, 
2+3 is the same as 3+2.  And I bet you expect that parentheses don't 
matter either with a $+$.

To call them all addition suggests we see things in common and 
we can borrow experience from other numbers to guess at properties 
for the next system.  When we are successful in abstracting the 
right properties we can turn a guess into a proof.  For example,
the typical situation these days is that the symbol $+$ abstracts 
any process that turns a pair of numbers $x$ and $y$ into a new one,
written $x+y$, along  with the property that $x+y=y+x$ and $x+(y+z)=(x+y)+z$.
Important abstractions get clever names: $x+y=y+x$ is the 
\emph{commutative law}, like dance partners the variable move past each other but 
stay together.  The second abstraction is the \emph{associative law}, where 
the variables are stuck in a line but can talk to the one in front or behind them.
But you know this already.


That is because 
of a a trick: \emph{abstraction}, which means to study a subject 
by considering only some of its possible attributes.  For example, 

Yet if you open a text on algebra that isn't for high-schools
you find instead page after page of 
Of course every algebraist knows this but it 

since applications often need us to solve for one.
Its simply not as important to study $2^2+2+1=7$  
as it is to sol
Of course we could study 2=2 and 
The method however is 
to  of study is  Start with $x^2+1=0$.
What
It has no 
real solutions.


The first lesson of algebra is that every symbol can be variable.
For example imagine focussing on solving $x^2=3$.  


So focus on solving $x^2+x+1=0$.  


For example consider the equation $x^2+x+1=0$.  What 
When you see $x+3=5$


\chapter{Functions}

The first lesson of algebra is that everything you see is a variable. you can improve how we study equations 

A function takes an input and use some process to determine an output,
under the premise that this process is repeatable.  So reading the 
time off a watch is not a function because time keeps changing, 
but converting time from hours to minutes is a function because 
every hour has 60 minutes.  From that philosophical understanding 
two functions are obvious:
\begin{align*}
    I(x) & = x & 
    K_c(x) & = c.
\end{align*}
We call the first an \emph{identity} function and the second is a \emph{constant}
function.  This notation is misleading, technically $I$ is a symbol
which denotes some unknown process such that when applied to data $\clubsuit$,
its output, often denoted $I(\clubsuit)$, is given input data unchanged.
So $I(\clubsuit)=\clubsuit$ is a judgement we can make about an identity 
function not a program.

There is a bit of implicit information in our use 
of parenthesis and what they mean.  Traditionally $I$ and $K$ are 
the functions.  An input $\clubsuit$ passed to a function $I$ 
yields an output denoted $I(\clubsuit)$.  In the case of the identity 
function $I$ does nothing to modify the input so we can judge 
$I(\clubsuit)=\clubsuit$.  Since this applies for any input we 
can abstract over the inputs by replacing their role with a variable 
$x$ and write $I(x)=x$  It is not a definition of $I$ so much as 
a consequence.  Of course we could use the outcome rule to conjure 
a perspective algorithm to perform the function.  When we do this 
we often insert some extra language like ``define I(x)=x'' or 
use the Walrus notation $I(x)\defeq x$.  In programs words like 
define are abbreviated.  For example, the following two programs satisfy
the identity function rule without following the same process.
\begin{lstlisting}
    def doNothing(x) = x 
    def doNothingUseful(x) = compute 200! then return x
\end{lstlisting}
Much of the feasibility of algebra comes down to appreciating 
different processes that achieve the same overall calculations.


nothing in the statement $I(x)=x$ prevents the function 
from reading in an input, taking a trip to mars and returning, then 
outputting $x$.  So the description $I(x)=x$ is not so much a program 
but an fact

what we write.
Writing $I(x)=x$ is a judgement we can make after using the function $I$.
For example, the identity function applied to $3$, often written $I(3)$, yields 
3.  Indeed $I(3)=3$.  In fact if we want a function $J$ that never changes the input 
then all such $J$ will have the quality that $J(x)=x$ in the end.  Of course 
nothing prevents $J$ form first move the data around from place to place, 
Curiously 
this rule seems to So without 
even considering an algorithm to act Back tracking from the conclu

a function of $x$ meaning $x$ is the variable for which we 
will substitutes other quantities, $y(3)=3$ whereas $\acute{y}(3)=2$.
In fact $\acute{y}_2=2$ is part of a family of functions which we can describe 
as general \emph{constant functions} denoted 
\begin{align*}
    \acute{}
\end{align*} 

Composing means to send the output of one function to the input 
of another so we could do this with our identity and constant functions 



These are not the same functions of course.  

That they are functions is more like a rule than a fact.  When you 
need to formalize this a bit more you can start by saying you have an alphabet 
of symbols containing $I,K$.  You 


If you needed 
to formalize this a bit more you could say $I$ is a symbol and there is rule 
$That is,
$I(x)$ just means $

writing $I(x)=x$ is notation for ``$I$ applied to data $x$ returns $x$'' and 
because 
it would take some circular reasoning to explain what ``I(x)=x'' means if not 
to operate as function that changes nothing.  We might like to compose functions, 
for instance $IK_c$ or 
In fact this brings up a problem 
of what it 


\section{}
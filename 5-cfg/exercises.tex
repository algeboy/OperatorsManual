\section*{Exercises}
\begin{enumerate}
    \item Write down a grammar to accept natural numbers as digits.  Remember $07$ is not proper substitute for $7$.
    \item Make a grammar to describe rational numbers.
    \item Make a grammar to describe decimal numbers, call this \code{<Reals>}.  
    \item Augment your \code{<Reals>} to include constants for $\pi$ and $e$.
    \item Give a full grammar for real polynomials including conventional short-hand 
    such as $x$ for $x^1$ and $ax$ instead of $(a)(x)$.
    
    \item Suppose that we want to able to add a tally mark to either the left (L) or right (R).
    Define a grammar that allows.  Explain why this is not a model for the natural numbers.
        
    \item Turn the following function into code in both functional and procedural dialects.
    \begin{align*}
        f(n) & = \begin{cases}
                    0 & n=0\\
                    S0 & n=S(k)
        \end{cases}
         =\begin{cases} 0 & n=0 \\ 1 & \text{else}\end{cases}.
    \end{align*}
    
    \item Define multiplication of natural numbers from the Peano grammar given.
    
    \item Write a program that defines natural numbers, giving its usual operations of addition 
    and multiplication from their inductive definition.  It wont be fast, that is not the point,
    it is practicing to translate induction into programs.
\end{enumerate}
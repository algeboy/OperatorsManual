
\documentclass[12pt,twoside,letterpaper]{memoir}

\usepackage{import}
\import{../tex-preambles/}{Content.tex}
% %% GENERAL ADD ONS %%%%%%%%%%%%%%%%%%%%%%%%%%%%%%%%%%%%%%%%%%%%%%%%%%%%%%%%%%%%

\usepackage[dvipsnames]{xcolor}  %% Must come before Tikz etc.
\usepackage[many]{tcolorbox}
\usepackage{dcounter}
\usepackage{enumerate}
\usepackage{graphicx}
\usepackage[
    hyperindex=true,
    colorlinks=true,
    linkcolor=teal,
    citecolor=ForestGreen]{hyperref}
\usepackage{float}     %% given captions to listings and types.
\usepackage{caption}
\usepackage{subcaption}
% \usepackage{aeguill}  %% IF NEEDED INCLUDE IN Content.tex near top
\usepackage[light]{kpfonts} % Nice Adobe Font
\usepackage{pifont} % \xmark
    \newcommand{\cmark}{\ding{51}}%
    \newcommand{\xmark}{\ding{55}}%

%% TIKZ %%%%%%%%%%%%%%%%%%%%%%%%%%%%%%%%%%%%%%%%%%%%%%%%%%%%%%%%%%%%%%%%%%%%%%%
\usepackage{tikz}
\usepackage{tikz-cd}
    \usetikzlibrary{positioning,arrows,backgrounds}
%%%%%%%%%%%%%%%%%%%%%%%%%%%%%%%%%%%%%%%%%%%%%%%%%%%%%%%%%%%%%%%%%%%%%%%%%%%%%%%

%% MATH %%%%%%%%%%%%%%%%%%%%%%%%%%%%%%%%%%%%%%%%%%%%%%%%%%%%%%%%%%%%%%%%%%%%%%%

\usepackage{amsmath,amsthm,amssymb}

\newcommand{\bigforall}{\mbox{\Large $\mathsurround0pt\forall$}}
\newcommand{\bigexists}{\mbox{\Large $\mathsurround0pt\exists$}}


\usepackage{bussproofs}  %% proof diagrams
\usepackage{wasysym} % \Leftcircle
\usepackage{mathtools}  % left/right harpoons
% \usepackage{lscape}  %% Landscape pages.
% \usepackage{longtable}  %% tables that span 2 pages

%%%%%%%%%%%%%%%%%%%%%%%%%%%%%%%%%%%%%%%%%%%%%%%%%%%%%%%%%%%%%%%%%%%%%%%%%%%%%%%
% \usepackage{zref-abspos} % Used to know where digress/focus starts/ends

%% STYLE CHAPTERS %%%%%%%%%%%%%%%%%%%%%%%%%%%%%%%%%%%%%%%%%%%%%%%%%%%%%%%%%%%%%
\makechapterstyle{invhangnum}{%
  \setlength\beforechapskip{0pt}%
  \renewcommand*\chapterheadstart{\vspace{\beforechapskip}}%
  \setlength\afterchapskip{2\onelineskip plus .2\onelineskip minus 0.2\onelineskip}%  
  \renewcommand\chaptitlefont{\Huge\bfseries}
  \setlength\midchapskip{-\baselineskip}%
  \renewcommand\chapnumfont{\Huge\bfseries}
  \renewcommand*{\printchaptername}{}
  \renewcommand*{\chapternamenum}{}
  \renewcommand*{\printchapternum}{%
      \raisebox{\dimexpr\midchapskip+\baselineskip\relax}[0pt][0pt]{%
        \makebox[0pt][l]{%
        \makebox[\dimexpr\textwidth+4em\relax][l]{%
          \parbox[t]{\textwidth}{\mbox{}}%
          \parbox[t]{4em}{\hfill\chapnumfont \thechapter}}}}}%
  \renewcommand*{\printchaptertitle}[1]{%
    \raisebox{\dimexpr\midchapskip+\baselineskip\relax}[0pt][0pt]{%
      \parbox[t]{\textwidth}{\raggedright\chaptitlefont ##1}}}%
}
\chapterstyle{invhangnum}
\newlength{\digresslen}
% \chapterstyle{hangnum}
% \hangsecnum

    \usepackage{eso-pic}   %% Used to add gray strip to side of pages.
    %%%%%%%%% Digression margins.
    % \newenvironment{}{}{}
    \newcommand{\digresshere}[1]{
        % \newdimen\digheight
        % \setbox0=\vbox{#1}
        % \digheight=\ht0 \advance\digheight by \dp0
        % \zsavepos{#1-digress}
        % \write\mywrite{#1: \zposy{#1-digress}}%
        % \AddToShipoutPictureBG{%
        % \AtPageLowerLeft{%
        
            % \ifodd\value{page}
            % \hspace*{\dimexpr\paperwidth-3em}%
            % \fi
            % \settoheight{\digresslen}{\vbox{#1}}
        \marginpar{\hspace{\marginparwidth}\textcolor{black!25}{\rule{3em}{1in}}}%
        % }%
        % }%
        % \newcommand{\focus}{\ClearShipoutPictureBG}
    }

    \newcommand{\digress}{
        % \zsavepos{#1-digress}
        % \write\mywrite{#1: \zposy{#1-digress}}%
    \AddToShipoutPictureBG{%
        \AtPageLowerLeft{%
        %\raisebox{.25\paperheight}{%
            \ifodd\value{page}
            \hspace*{\dimexpr\paperwidth-3em}%
            \fi
            \textcolor{black!25}{\rule{3em}{\paperheight}}%
        %}%
        }%
        }
    }
    \newcommand{\focus}{\ClearShipoutPictureBG}
    \newcommand{\codemargin}[1]{
        \marginpar{
            \colorbox{black!25}{
                \parbox{0.9\marginparwidth}{
                    {\small \textsf{#1}}
                }
            }
        }
    }
    %% TOC Formatting
    % Indentation of numbers
    \setlength{\cftsectionindent}{2em}      % default 1.5em

    % Distance between numbers and titles
    \setlength{\cftpartnumwidth}{2.25em}      % default 1.5em
    \setlength{\cftchapternumwidth}{2em}      % default 1.5em
    \setlength{\cftsectionnumwidth}{3em}      % default 2.3em





\usepackage{imakeidx}
    \makeindex
    \indexsetup{headers={Index}{Index}} % Left/Right headers for the index

    \newcommand{\term}[1]{\emph{#1}}%\index{#1}}
%%%%%%%%%%%%%%%%%%%%%%%%%%%%%%%%%%%%%%%%%%%%%%%%%%%%%%%%%%%%%%%%%%%%%%%%%%%%%%%

%% TITLE PAGE %%%%%%%%%%%%%%%%%%%%%%%%%%%%%%%%%%%%%%%%%%%%%%%%%%%%%%%%%%%%%%%%%
\usepackage{epigraph}
\usepackage{titling}
\usepackage{bookman}
    \renewcommand\epigraphflush{flushright}
    \renewcommand\epigraphsize{\normalsize}
    \setlength\epigraphwidth{0.7\textwidth}

    % CSU Green 92, 18, 94, 61
    \definecolor{titlepagecolor}{cmyk}{.92,.18,.94,.61}
    % CSU Gold 11, 6, 64, 13
    \definecolor{titlepagecolor2}{cmyk}{.11,.06,.64,.13}
    %\definecolor{titlepagecolor}{cmyk}{1,.60,0,.40}

    \DeclareFixedFont{\titlefont}{T1}{ppl}{b}{it}{0.5in}
    % The following code is borrowed from: https://tex.stackexchange.com/a/86310/10898

    %%% STOLEN FROM 
    %%%% https://tex.stackexchange.com/questions/85904/showcase-of-beautiful-title-page-done-in-tex
    \newcommand\titlepagedecoration{%
    \begin{tikzpicture}[remember picture,overlay,shorten >= -10pt]

    \coordinate (aux1) at ([yshift=-15pt]current page.north east);
    \coordinate (aux2) at ([yshift=-410pt]current page.north east);
    \coordinate (aux3) at ([xshift=-4.5cm]current page.north east);
    \coordinate (aux4) at ([yshift=-150pt]current page.north east);

    \begin{scope}[titlepagecolor!40,line width=12pt,rounded corners=12pt]
    \draw
    (aux1) -- coordinate (a)
    ++(225:5) --
    ++(-45:5.1) coordinate (b);
    \draw[shorten <= -10pt]
    (aux3) --
    (a) --
    (aux1);
    \draw[opacity=0.6,titlepagecolor2,shorten <= -10pt]
    (b) --
    ++(225:2.2) --
    ++(-45:2.2);
    \end{scope}
    \draw[titlepagecolor,line width=8pt,rounded corners=8pt,shorten <= -10pt]
    (aux4) --
    ++(225:0.8) --
    ++(-45:0.8);
    \begin{scope}[titlepagecolor!70,line width=6pt,rounded corners=8pt]
    \draw[shorten <= -10pt]
    (aux2) --
    ++(225:3) coordinate[pos=0.45] (c) --
    ++(-45:3.1);
    \draw
    (aux2) --
    (c) --
    ++(135:2.5) --
    ++(45:2.5) --
    ++(-45:2.5) coordinate[pos=0.3] (d);   
    \draw 
    (d) -- +(45:1);
    \end{scope}
    \end{tikzpicture}%
    }


%%%%%%%%%%%%%%%%%%%%%%%%%%%%%%%%%%%%%%%%%%%%%%%%%%%%%%%%%%%%%%%%%%%%%%%%%%%%%%%

%% STYLE POPOUTS %%%%%%%%%%%%%%%%%%%%%%%%%%%%%%%%%%%%%%%%%%%%%%%%%%%%%%%%%%%%%%

\usepackage{tcolorbox}
    \newenvironment{guide}[1]{
        \medskip
        \begin{center}
        \begin{minipage}[t]{0.95\linewidth} %{\dimexpr0.33\textwidth-2\fboxrule-2\fboxsep\relax}
            \begin{tcolorbox}[colback=gray!5,colframe=green!40!black,title=#1]
    }{
            \end{tcolorbox}
        \end{minipage}
        \end{center}
        \medskip
    }%
    \newenvironment{mywarning}[1] {
        \medskip
        \begin{center}
        \begin{minipage}[t]{0.95\linewidth} %{\dimexpr0.33\textwidth-2\fboxrule-2\fboxsep\relax}
            \begin{tcolorbox}[colback=gray!5,colframe=red!40!black,title=#1]
    }{
            \end{tcolorbox}
        \end{minipage}
        \end{center}
        \medskip
    }%
    \newenvironment{open}{
        \medskip
        \begin{center}
        \begin{minipage}[t]{0.95\linewidth} %{\dimexpr0.33\textwidth-2\fboxrule-2\fboxsep\relax}
            \begin{tcolorbox}[colback=gray!5,colframe=blue!40!black,title=Open Problem]
    }{
            \end{tcolorbox}
        \end{minipage}
        \end{center}
        \medskip
    }%
%%%%%%%%%%%%%%%%%%%%%%%%%%%%%%%%%%%%%%%%%%%%%%%%%%%%%%%%%%%%%%%%%%%%%%%%%%%%%%%





%% FONTS %%%%%%%%%%%%%%%%%%%%%%%%%%%%%%%%%%%%%%%%%%%%%%%%%%%%%%%%%%%%%%%%%%%%%%
% \usepackage{fontspec}
% \usepackage[T1]{fontenc}
% \usepackage{kpfonts}
% \usepackage[T1]{fontenc}

%% AUTHOR INDEX %%%%%%%%%%%%%%%%%%%%%%%%%%%%%%%%%%%%%%%%%%%%%%%%%%%%%%%%%%%%%%%

\newcommand{\Church}{\index{Church, Alonzo (1903--1995)}}
\newcommand{\Codd}{\index{Codd, Edgar F. (1923--2003)}}
\newcommand{\Coquand}{\index{Coquand, Thierry (1961--)}}
\newcommand{\Curry}{\index{Curry, Haskell (1900--1982)}}
\newcommand{\Frege}{\index{Frege, Gottlob (1884--1925)}}
\newcommand{\Hamilton}{\index{Hamilton, Sir William Rowan (1805--1865)}}
\newcommand{\Hopper}{\index{Hopper, Grace (1906--1992)}}
\newcommand{\Leibniz}{\index{Leibniz, Gottfried (1646--1716)}}
\newcommand{\MartinLof}{\index{Martin-Lof@Martin-L\"of, Per (1942--)}}
\newcommand{\Russell}{\index{Russell, Bertrand (1872--1970)}}
\newcommand{\Tarski}{\index{Tarski, Alfred (1901--1983)}}
\newcommand{\Turing}{\index{Turing, Alan (1912--1954)}}
\newcommand{\Whitehead}{\index{Whithead, Alfred North (1861--1947)}}
\newcommand{\ZF}{\index{Zermelo, Ernst (1871--1953)}\index{Fraenkel, Abraham (1891--1965)}}




%% FINAL SETTINGS %%%%%%%%%%%%%%%%%%%%%%%%%%%%%%%%%%%%%%%%%%%%%%%%%%%%%%%%%%%%%
\numberwithin{figure}{chapter}
\numberwithin{table}{chapter}

%% EDITING MARKS %%%%%%%%%%%%%%%%%%%%%%%%%%%%%%%%%%%%%%%%%%%%%%%%%%%%%%%%%%%%%%
\usepackage[draft]{pdfcomment}
\usepackage{xparse}
    \DeclareDocumentCommand \EDITmargin { o m } {
        \IfNoValueTF {#1} {
            \pdfmargincomment[icon=Note]{#2}
        }{
            \pdfmargincomment[icon=NOte,author=#1]{#2}
        }
    }
    \DeclareDocumentCommand \EDITcomment { o m } {
        \IfNoValueTF {#1} {
            \pdfcomment[color=Blue!20]{#2}
        }{
            \pdfcomment[color=Blue!20,author=#1]{#2}
        }
    }
    \DeclareDocumentCommand \EDITalt { o m m } {
        \IfNoValueTF {#1} {
            \pdfmarkupcomment[markup=StrikeOut]{#2}{#3}
        }{
            \pdfmarkupcomment[markup=StrikeOut,icon=key, author=#1]{#2}{#3}
        }
    }
    \DeclareDocumentCommand \EDITtypo { o m o } {
        \IfNoValueTF {#3} {
            \pdfmarkupcomment[markup=Squiggly,author=#1]{#2}{}
        }{
            \pdfmarkupcomment[markup=Squiggly,author=#1]{#2}{#3}
        }
    }
    \DeclareDocumentCommand \EDIThighlight { o m m } {
        % \IfNoValueTF {#3} {
        %     \pdfmarkupcomment[markup=Highlight,author=#1,color=blue!20]{#2}{}
        % }{
            \pdfmarkupcomment[markup=Highlight,author=#1,color=blue!20]{#2}{#3}
        % }
    }


    
%%%%%%%%%%%%%%%%%%%%%%%%%%%%%%%%%%%%%%%%%%%%%%%%%%%%%%%%%%%%
%% Theorems
\usepackage{thmtools}
\declaretheoremstyle[
    shaded={
        rulecolor=black!30,
        rulewidth=1pt,    
        bgcolor=white
    }
]{thm}
\declaretheorem[numberwithin=chapter, style=thm]{theorem}
\declaretheorem[sibling=theorem, style=thm]{lemma}
\declaretheorem[sibling=theorem, style=thm]{proposition}
\declaretheorem[sibling=theorem, style=thm]{corollary}


\declaretheoremstyle[
    shaded={
        rulecolor=purple!30,
        rulewidth=1pt,    
        bgcolor=white
    }
]{eg}
% \declaretheorem[sibling=theorem, style=eg]{ex}
\declaretheorem[sibling=theorem, style=eg]{example}

\declaretheoremstyle[
    shaded={
        rulecolor=teal!20,
        rulewidth=1pt,    
        bgcolor=white
    }
]{def}
\declaretheorem[sibling=theorem, style=def]{definition}

\declaretheoremstyle[
    shaded={
        rulecolor=ForestGreen,
        rulewidth=1pt,    
        bgcolor=white
    }
]{rem}
\declaretheorem[sibling=theorem, style=rem]{remark}

\declaretheorem[sibling=theorem, name=Engineering Note, style=rem]{implremark}
\declaretheorem[sibling=theorem, name=Historical Remark, style=rem]{histremark}

%-----------------------------------------------------
%       Standard theorem like environments.
%-----------------------------------------------------
%% \theoremstyle{plain} %% This is the default
\numberwithin{equation}{chapter}
%\newtheorem{theorem}[equation]{Theorem}
%\newtheorem*{theorem*}{Theorem}
%\newtheorem{lemma}[equation]{Lemma}
%\newtheorem{proposition}[equation]{Proposition}
\newtheorem{prob}{}[chapter]

%\newtheorem{lem}[equation]{Lemma}
%\newtheorem{corollary}[equation]{Corollary}
%\newtheorem*{coro*}{Corollary}
\newtheorem{quest}[equation]{Question}

\theoremstyle{remark}


\theoremstyle{definition}
%\newtheorem{definition}[equation]{Definition}
%\newtheorem{ex}{Example}[equation]
% \newtheorem*{remark*}{Remark}
% \newtheorem{remark}[equation]{Remark}
% \newtheorem{progrem}[equation]{Programing Remark}
% \newtheorem{histrem}[equation]{Historical Remark}



%%%%%%%%%%%%%%%%%%%%%%%%%%%%%%%%%%%%%%%%%%%%%%%%%%%%%%%%%%%

\newcommand{\plusplus}{{\tiny ++}}

\newcommand{\tsspace}{\mathsf{space}~}
\newcommand{\tsaxes}{\mathsf{axes}~}
\newcommand{\tsframe}{\mathsf{frame}~}
\newcommand{\tsbase}{\mathsf{base}~}
\newcommand{\tsinterp}{\mathsf{interp.}~}


\newcommand{\NamedTensorSpace}[6]{
\begin{array}{lllll}
    #1 & \defeq \mathsf{TensorSpace}\big(
        &\tsspace & #2,\\
&        &\tsaxes & #3,\\
&        &\tsframe & #4,\\
&        &\tsbase & #5,\\            
&        &\tsinterp & #6 ~~~\big)        
    \end{array}
}
\newcommand{\TensorSpace}[5]{
\begin{array}{lll}
\mathsf{TensorSpace}\big(
        &\tsspace & #1,\\
        &\tsaxes & #2,\\
        &\tsframe & #3,\\
        &\tsbase & #4,\\            
        &\tsinterp & #5 ~~~\big)        
    \end{array}
}
\newcommand{\InlineTensorSpace}[5]{
$\mathsf{TensorSpace}($ 
    $\tsspace #1$, 
    $\tsaxes #2$,
    $\tsframe #3$,
    $\tsbase #4$,
    $\tsinterp #5$)
}
\import{../tex-preambles/}{tikz-sets.tex}
\import{../tex-preambles/}{math-preambles.tex}
\import{../tex-preambles/}{listings-preamble.tex}

\usepackage{dsfont} % for black-board bold 1.
\newcommand{\one}{\mathds{1}}
\begin{document}

\author{James B. Wilson}
\title{Categories as I came to know them}
\date{\today}
\maketitle

These days when I think of a category I see a monoid with holes and I think for 
some of you it might help to know this perspective as well.  So what do I mean by a monoid?

A \emph{monoid} is a multiplication $g\cdot h$ and an identity $1$ such that
\begin{align*}
    g\cdot (h\cdot k) & = (g\cdot h)\cdot k 
    & 
    1\cdot g & = g = g\cdot 1.
\end{align*}
These are signatures of a monoid not the required notation.
You may think of the natural numbers 
\begin{align*}
    n:\mathbb{N} = 0:\mathbb{N} 
    \text{ or a successor } S(k):\mathbb{N} 
    \text{ of an existing } k:\mathbb{N}.
\end{align*}
Successors are for counting---a tally, but I often count in groups and 
add the results so this gives over to the common notion of addition:
\begin{align*}
    m+n & = \begin{cases} m & n=0\\ S(m+k) & n=S(k) \end{cases}
\end{align*}
In this monoid the role of $\cdot$ is played by $+$ and $1$ by $0$.
Meanwhile if I switch to using multiplication of natural numbers 
\begin{align*}
    m\cdot n & = \begin{cases} 0 & n=0\\ m\cdot k +m & n=S(k) \end{cases}
\end{align*}
I still get a monoid only now $\cdot$ is the product and $1$ is $1:\mathbb{N}$.
% Before continuing on to categories it is 
% worth pausing to notice how different these two algebraic structures 
% really are.  In $\langle \mathbb{N},+,0\rangle$ every number $n:\mathbb{N}$ 
% is $f(1)$ for a $\{\cdot,1\}$-formula $f(x)$ in one variables, e.g. 
% $3=(1+1)+1=((x\cdot x)\cdot x)|_{x:=1}$ when we replace products by $+$.
% Meanwhile to write that same $n$ as a true product of natural numbers 
% would need access to all the primes, i.e.\ $6=2\cdot 3$ and $35=5\cdot 7$.

Now an \emph{category} is the same idea as a monoid only we accept
that sometimes products we are not everywhere defined.  

\subsection{Why would a product not be everywhere defined?}

Some products are too expensive to define everywhere.  Say what you 
will about the existence of every natural number, but there is no point 
in multiplying two random 1 million digit integers.  Even if it is possible 
it is an unforgivable waste of energy, time, and money.  That limit tends to 
break many would-be monoids.  As a pint-size example try the sum and product of
natural numbers $0,\ldots, 15$---a traditional engineering limit imposed 
on some of the first computers which got the name \emph{nibble} in the 1950's.Those numbers got to be written in Hexidecimals
\[
    H=\{0,\ldots,9,A,B,C,D,E,F\}.
\]
I.e. $C=12$.
The arithmetic operations the addition and multiplication tables would be these.
\begin{center}
    \begin{tikzpicture}
        % \foreach \x in \{0,...,15\} {

        % }
        \pgfmathsetmacro{\s}{0.5};
        \pgfmathsetmacro{\t}{0.25};
        \foreach \x in {0,...,15} {
            \node[scale=\s,fill=Violet!40] at (16*\t-\x*\t,-\x*\t) {$F$};
        }
        \foreach \x in {0,...,14} {
            \node[scale=\s,fill=NavyBlue!40] at (15*\t-\x*\t,-\x*\t) {$E$};
        }
        \foreach \x in {0,...,13} {
            \node[scale=\s,fill=SeaGreen!40] at (14*\t-\x*\t,-\x*\t) {$D$};
        }
        \foreach \x in {0,...,12} {
            \node[scale=\s,fill=Emerald!40] at (13*\t-\x*\t,-\x*\t) {$C$};
        }
        \foreach \x in {0,...,11} {
            \node[scale=\s,fill=green!40] at (12*\t-\x*\t,-\x*\t) {$B$};
        }
        \foreach \x in {0,...,10} {
            \node[scale=\s,fill=YellowGreen!40] at (11*\t-\x*\t,-\x*\t) {$A$};
        }
        \foreach \x in {0,...,9} {
            \node[scale=\s,fill=GreenYellow!40] at (10*\t-\x*\t,-\x*\t) {$9$};
        }
        \foreach \x in {0,...,8} {
            \node[scale=\s,fill=Yellow!40] at (9*\t-\x*\t,-\x*\t) {$8$};
        }
        \foreach \x in {0,...,7} {
            \node[scale=\s,fill=Goldenrod!40] at (8*\t-\x*\t,-\x*\t) {$7$};
        }
        \foreach \x in {0,...,6} {
            \node[scale=\s,fill=BurntOrange!40] at (7*\t-\x*\t,-\x*\t) {$6$};
        }
        \foreach \x in {0,...,5} {
            \node[scale=\s,fill=Dandelion] at (6*\t-\x*\t,-\x*\t) {$5$};
        }
        \foreach \x in {0,...,4} {
            \node[scale=\s,fill=Red!40] at (5*\t-\x*\t,-\x*\t) {$4$};
        }
        \foreach \x in {0,...,3} {
            \node[scale=\s,fill=BrickRed!40] at (4*\t-\x*\t,-\x*\t) {$3$};
        }
        \foreach \x in {0,...,2} {
            \node[scale=\s,fill=Maroon!50] at (3*\t-\x*\t,-\x*\t) {$2$};
        }
        \foreach \x in {0,1} {
            \node[scale=\s,fill=Maroon!60] at (2*\t-\x*\t,-\x*\t) {$1$};
        }
        \node[scale=\s,fill=Maroon!80] at (\t,0) {$0$};
    \end{tikzpicture}
    \hspace{1in}
    \begin{tikzpicture}
        % \foreach \x in \{0,...,15\} {

        % }
        \pgfmathsetmacro{\s}{0.5};
        \pgfmathsetmacro{\t}{0.25};
        \foreach \x in {0,...,15} {
            \node[scale=\s,fill=Maroon!80] at (\x*\t,0) {$0$};
            \node[scale=\s,fill=Maroon!80] at (0,-\x*\t) {$0$};
        }
        \node[scale=\s,fill=Maroon!60] at (\t,-\t) {$1$};

        \node[scale=\s,fill=Maroon!50] at (2*\t,-\t) {$2$};
        \node[scale=\s,fill=Maroon!50] at (\t,-2*\t) {$2$};

        \node[scale=\s,fill=BrickRed!40] at (3*\t,-\t) {$3$};
        \node[scale=\s,fill=BrickRed!40] at (\t,-3*\t) {$3$};

        \node[scale=\s,fill=Red!40] at (4*\t,-\t) {$4$};
        \node[scale=\s,fill=Red!40] at (2*\t,-2*\t) {$4$};
        \node[scale=\s,fill=Red!40] at (\t,-4*\t) {$4$};

        \node[scale=\s,fill=Dandelion] at (5*\t,-\t) {$5$};
        \node[scale=\s,fill=Dandelion] at (\t,-5*\t) {$5$};

        \node[scale=\s,fill=BurntOrange!40] at (6*\t,-\t) {$6$};
        \node[scale=\s,fill=BurntOrange!40] at (3*\t,-2*\t) {$6$};
        \node[scale=\s,fill=BurntOrange!40] at (2*\t,-3*\t) {$6$};
        \node[scale=\s,fill=BurntOrange!40] at (\t,-6*\t) {$6$};

        \node[scale=\s,fill=Goldenrod!40] at (7*\t,-\t) {$7$};
        \node[scale=\s,fill=Goldenrod!40] at (\t,-7*\t) {$7$};

        \node[scale=\s,fill=Yellow!40] at (8*\t,-1*\t) {$8$};
        \node[scale=\s,fill=Yellow!40] at (4*\t,-2*\t) {$8$};
        \node[scale=\s,fill=Yellow!40] at (2*\t,-4*\t) {$8$};
        \node[scale=\s,fill=Yellow!40] at (1*\t,-8*\t) {$8$};

        \node[scale=\s,fill=GreenYellow!40] at (9*\t,-1*\t) {$9$};
        \node[scale=\s,fill=GreenYellow!40] at (3*\t,-3*\t) {$9$};
        \node[scale=\s,fill=GreenYellow!40] at (1*\t,-9*\t) {$9$};

        \node[scale=\s,fill=YellowGreen!40] at (10*\t,-1*\t) {$A$};
        \node[scale=\s,fill=YellowGreen!40] at (5*\t,-2*\t) {$A$};
        \node[scale=\s,fill=YellowGreen!40] at (2*\t,-5*\t) {$A$};
        \node[scale=\s,fill=YellowGreen!40] at (1*\t,-10*\t) {$A$};

        \node[scale=\s,fill=green!40] at (11*\t,-1*\t) {$B$};
        \node[scale=\s,fill=green!40] at (1*\t,-11*\t) {$B$};

        \node[scale=\s,fill=Emerald!40] at (12*\t,-1*\t) {$C$};
        \node[scale=\s,fill=Emerald!40] at (6*\t,-2*\t) {$C$};
        \node[scale=\s,fill=Emerald!40] at (4*\t,-3*\t) {$C$};
        \node[scale=\s,fill=Emerald!40] at (3*\t,-4*\t) {$C$};
        \node[scale=\s,fill=Emerald!40] at (2*\t,-6*\t) {$C$};
        \node[scale=\s,fill=Emerald!40] at (1*\t,-12*\t) {$C$};

        \node[scale=\s,fill=SeaGreen!40] at (13*\t,-1*\t) {$D$};
        \node[scale=\s,fill=SeaGreen!40] at (1*\t,-13*\t) {$D$};

        \node[scale=\s,fill=NavyBlue!40] at (14*\t,-1*\t) {$E$};
        \node[scale=\s,fill=NavyBlue!40] at (7*\t,-2*\t) {$E$};
        \node[scale=\s,fill=NavyBlue!40] at (2*\t,-7*\t) {$E$};
        \node[scale=\s,fill=NavyBlue!40] at (1*\t,-14*\t) {$E$};

        \node[scale=\s,fill=Violet!40] at (15*\t,-1*\t) {$F$};
        \node[scale=\s,fill=Violet!40] at (5*\t,-3*\t) {$F$};
        \node[scale=\s,fill=Violet!40] at (3*\t,-5*\t) {$F$};
        \node[scale=\s,fill=Violet!40] at (1*\t,-15*\t) {$F$};
    \end{tikzpicture}
\end{center}
There are gaping holes in these two operations so $+$ and $\cdot$ 
are not even functions $A\times A\to A$ so they cannot be operations
in that technical sense.  


Now for some reality.  Computers actually do something when asked 
to add or multiply beyond the bounds.  Most output some sort of 
error code, something like \textsc{overflow}.  I will 
use $\bot$.  So in that model computers and I are extending $H$ to include the 
error outcome like this
\[
    H^? = H\sqcup\{\bot\} = \{0,\ldots,9,A,\ldots, F,\bot\}.
\]
So our addition and multiplication tables are now functions 
$+,\cdot :H \times H\to H^?$ or what are often called \emph{partial}
operations because of the possible error state.  This at least makes 
it clear that we have functions back.  What about our monoids?

This sort of 
failure does break the monoid conditions on addition and multiplication 
but only a mostly uninteresting way.  For example we could extend our 
sum and product by the \emph{absorption rule}
\begin{align*}
    n+\bot & = \bot = \bot + n 
    & 
    n\cdot \bot & = \bot = \bot \cdot n.
\end{align*}  
\begin{center}
    \begin{tikzpicture}
        % \foreach \x in \{0,...,15\} {

        % }
        \pgfmathsetmacro{\s}{0.5};
        \pgfmathsetmacro{\t}{0.25};
        \foreach \x in {1,...,17} {
            \foreach \y in {0,...,16} {
                \node[scale=\s,fill=black!40] at (\t*\x, -\t*\y) {$\bot$};
            }
        }
        \foreach \x in {0,...,15} {
            \node[scale=\s,fill=Violet!40] at (16*\t-\x*\t,-\x*\t) {$F$};
        }
        \foreach \x in {0,...,14} {
            \node[scale=\s,fill=NavyBlue!40] at (15*\t-\x*\t,-\x*\t) {$E$};
        }
        \foreach \x in {0,...,13} {
            \node[scale=\s,fill=SeaGreen!40] at (14*\t-\x*\t,-\x*\t) {$D$};
        }
        \foreach \x in {0,...,12} {
            \node[scale=\s,fill=Emerald!40] at (13*\t-\x*\t,-\x*\t) {$C$};
        }
        \foreach \x in {0,...,11} {
            \node[scale=\s,fill=green!40] at (12*\t-\x*\t,-\x*\t) {$B$};
        }
        \foreach \x in {0,...,10} {
            \node[scale=\s,fill=YellowGreen!40] at (11*\t-\x*\t,-\x*\t) {$A$};
        }
        \foreach \x in {0,...,9} {
            \node[scale=\s,fill=GreenYellow!40] at (10*\t-\x*\t,-\x*\t) {$9$};
        }
        \foreach \x in {0,...,8} {
            \node[scale=\s,fill=Yellow!40] at (9*\t-\x*\t,-\x*\t) {$8$};
        }
        \foreach \x in {0,...,7} {
            \node[scale=\s,fill=Goldenrod!40] at (8*\t-\x*\t,-\x*\t) {$7$};
        }
        \foreach \x in {0,...,6} {
            \node[scale=\s,fill=BurntOrange!40] at (7*\t-\x*\t,-\x*\t) {$6$};
        }
        \foreach \x in {0,...,5} {
            \node[scale=\s,fill=Dandelion] at (6*\t-\x*\t,-\x*\t) {$5$};
        }
        \foreach \x in {0,...,4} {
            \node[scale=\s,fill=Red!40] at (5*\t-\x*\t,-\x*\t) {$4$};
        }
        \foreach \x in {0,...,3} {
            \node[scale=\s,fill=BrickRed!40] at (4*\t-\x*\t,-\x*\t) {$3$};
        }
        \foreach \x in {0,...,2} {
            \node[scale=\s,fill=Maroon!50] at (3*\t-\x*\t,-\x*\t) {$2$};
        }
        \foreach \x in {0,1} {
            \node[scale=\s,fill=Maroon!60] at (2*\t-\x*\t,-\x*\t) {$1$};
        }
        \node[scale=\s,fill=Maroon!80] at (\t,0) {$0$};
    \end{tikzpicture}
    \hspace{1in}
    \begin{tikzpicture}
        % \foreach \x in \{0,...,15\} {

        % }
        \pgfmathsetmacro{\s}{0.5};
        \pgfmathsetmacro{\t}{0.25};
        \foreach \x in {0,...,16} {
            \foreach \y in {0,...,16} {
                \node[scale=\s,fill=black!40] at (\t*\x, -\t*\y) {$\bot$};
            }
        }
        \foreach \x in {0,...,15} {
            \node[scale=\s,fill=Maroon!80] at (\x*\t,0) {$0$};
            \node[scale=\s,fill=Maroon!80] at (0,-\x*\t) {$0$};
        }
        \node[scale=\s,fill=Maroon!60] at (\t,-\t) {$1$};

        \node[scale=\s,fill=Maroon!50] at (2*\t,-\t) {$2$};
        \node[scale=\s,fill=Maroon!50] at (\t,-2*\t) {$2$};

        \node[scale=\s,fill=BrickRed!40] at (3*\t,-\t) {$3$};
        \node[scale=\s,fill=BrickRed!40] at (\t,-3*\t) {$3$};

        \node[scale=\s,fill=Red!40] at (4*\t,-\t) {$4$};
        \node[scale=\s,fill=Red!40] at (2*\t,-2*\t) {$4$};
        \node[scale=\s,fill=Red!40] at (\t,-4*\t) {$4$};

        \node[scale=\s,fill=Dandelion] at (5*\t,-\t) {$5$};
        \node[scale=\s,fill=Dandelion] at (\t,-5*\t) {$5$};

        \node[scale=\s,fill=BurntOrange!40] at (6*\t,-\t) {$6$};
        \node[scale=\s,fill=BurntOrange!40] at (3*\t,-2*\t) {$6$};
        \node[scale=\s,fill=BurntOrange!40] at (2*\t,-3*\t) {$6$};
        \node[scale=\s,fill=BurntOrange!40] at (\t,-6*\t) {$6$};

        \node[scale=\s,fill=Goldenrod!40] at (7*\t,-\t) {$7$};
        \node[scale=\s,fill=Goldenrod!40] at (\t,-7*\t) {$7$};

        \node[scale=\s,fill=Yellow!40] at (8*\t,-1*\t) {$8$};
        \node[scale=\s,fill=Yellow!40] at (4*\t,-2*\t) {$8$};
        \node[scale=\s,fill=Yellow!40] at (2*\t,-4*\t) {$8$};
        \node[scale=\s,fill=Yellow!40] at (1*\t,-8*\t) {$8$};

        \node[scale=\s,fill=GreenYellow!40] at (9*\t,-1*\t) {$9$};
        \node[scale=\s,fill=GreenYellow!40] at (3*\t,-3*\t) {$9$};
        \node[scale=\s,fill=GreenYellow!40] at (1*\t,-9*\t) {$9$};

        \node[scale=\s,fill=YellowGreen!40] at (10*\t,-1*\t) {$A$};
        \node[scale=\s,fill=YellowGreen!40] at (5*\t,-2*\t) {$A$};
        \node[scale=\s,fill=YellowGreen!40] at (2*\t,-5*\t) {$A$};
        \node[scale=\s,fill=YellowGreen!40] at (1*\t,-10*\t) {$A$};

        \node[scale=\s,fill=green!40] at (11*\t,-1*\t) {$B$};
        \node[scale=\s,fill=green!40] at (1*\t,-11*\t) {$B$};

        \node[scale=\s,fill=Emerald!40] at (12*\t,-1*\t) {$C$};
        \node[scale=\s,fill=Emerald!40] at (6*\t,-2*\t) {$C$};
        \node[scale=\s,fill=Emerald!40] at (4*\t,-3*\t) {$C$};
        \node[scale=\s,fill=Emerald!40] at (3*\t,-4*\t) {$C$};
        \node[scale=\s,fill=Emerald!40] at (2*\t,-6*\t) {$C$};
        \node[scale=\s,fill=Emerald!40] at (1*\t,-12*\t) {$C$};

        \node[scale=\s,fill=SeaGreen!40] at (13*\t,-1*\t) {$D$};
        \node[scale=\s,fill=SeaGreen!40] at (1*\t,-13*\t) {$D$};

        \node[scale=\s,fill=NavyBlue!40] at (14*\t,-1*\t) {$E$};
        \node[scale=\s,fill=NavyBlue!40] at (7*\t,-2*\t) {$E$};
        \node[scale=\s,fill=NavyBlue!40] at (2*\t,-7*\t) {$E$};
        \node[scale=\s,fill=NavyBlue!40] at (1*\t,-14*\t) {$E$};

        \node[scale=\s,fill=Violet!40] at (15*\t,-1*\t) {$F$};
        \node[scale=\s,fill=Violet!40] at (5*\t,-3*\t) {$F$};
        \node[scale=\s,fill=Violet!40] at (3*\t,-5*\t) {$F$};
        \node[scale=\s,fill=Violet!40] at (1*\t,-15*\t) {$F$};
    \end{tikzpicture}
\end{center}
This adaptation restores the properties of a monoid because the addition and the
multiplication are now associative.

In general
$\mathbb{N}/_{k=k+1}$ allows for a similar approach for any $k$.   It is common
to use diagrams that depict how these data could be generated.
\begin{center}
    \begin{tikzpicture}[xscale=0.75]

        \foreach \x in {0,...,9} {
            \node[scale=0.75] (\x) at (\x,0) {$\x$};
            \draw[->] (\x) -- ++(0.75,0);
        }
        \foreach \x/\a in {10/A,11/B,12/C,13/D,14/E,15/F} {
            \node[scale=0.75,circle] (\x) at (\x,0) {$\a$};
            \draw[->] (\x) -- ++(0.75,0);
        }
        \node[scale=0.75] (t) at (16,0) {$\bot$};
        \draw (t) edge[->,out=45,in=-45, looseness=3] (t);
\end{tikzpicture}
\hspace{1in}
\begin{tikzpicture}[xscale=0.75]
    \node[scale=0.75] (1) at (0,0) {$1$};
    \node[scale=0.75] (2) at (1,0) {$2$};
    \node[scale=0.75] (4) at (4,0) {$4$};
    \node[scale=0.75] (8) at (8,0) {$8$};

    \node[scale=0.75] (3) at (3,1) {$3$};
    \node[scale=0.75] (9) at (9,2) {$9$};

    \node[scale=0.75] (6) at (6,1) {$6$};
    \node[scale=0.75] (10) at (10,-3) {$10$};
    \node[scale=0.75] (12) at (12,1) {$12$};
    \node[scale=0.75] (14) at (14,4) {$14$};
    \node[scale=0.75] (15) at (15,-3) {$15$};

    \node[scale=0.75] (5) at (5,-2) {$5$};
    \node[scale=0.75] (7) at (7,3) {$7$};
    \node[scale=0.75] (11) at (11,4) {$11$};
    \node[scale=0.75] (13) at (13,5) {$13$};
    \node[scale=0.75] (t) at (16,0) {$\bot$};
    \draw (t) edge[->,out=45,in=-45, looseness=3] (t);

    \draw[->] (1) -- (2);
    \draw[->] (1) -- (3);
    \draw (1) edge[->,bend right] (5);
    \draw (1) edge[->,bend left] (7);
    \draw (1) edge[->,bend left] (11);
    \draw (1) edge[->,bend left] (13);

    \draw[->] (2) -- (4);
    \draw[->] (2) -- (6);    
    \draw[->] (2) -- (10);
    \draw[->] (2) -- (14);

    \draw[->] (3) -- (6);
    \draw[->] (3) -- (9);
    \draw[->] (3) -- (15);
    \draw[->] (6) -- (12);

    \draw[->] (4) -- (8);
    \draw[->] (4) -- (12);

    \draw[->] (5) -- (10);
    \draw[->] (5) -- (15);
    \draw[->] (7) -- (14);

    \draw[->] (8) -- (t);
    \draw[->] (9) -- (t);
    \draw[->] (10) -- (t); %edge[bend left] (t);
    \draw[->] (11) edge[bend left] (t);
    \draw[->] (12) -- (t);
    \draw[->] (13) -- (t);
    \draw[->] (14) -- (t);
    \draw[->] (15) -- (t);
\end{tikzpicture}
\end{center}


The second way that monoid-type products could fail to be defined is when there 
was no intentional monoid behind it in the first place.  Suppose I give you instructions to walk.  Walk 2 block north, then 3 blocks west, then wait there.
That is a partial addition.  I am allow you to follow north walking by west walking.  There is no reason to believe from those instructions that you 
could combine the north with itself.


\begin{equation}
    \begin{array}{|c|ccc|}
        \hline 
        * & e & h & f\\
        \hline 
        e & e & \bot & \bot\\
        h & h & \bot & \bot \\
        f & \bot & h & f \\
        \hline
    \end{array}
\end{equation}



However if we do not include that case then 

So the addition of 
In this case 
the number $1:=S(0)$ is not the identity but rather it generates every term
$n:\mathbb{N}$ as a monoid.  ``Generates as a monoid'' is jargon to say that
$n=f(1)$ for some $\{\cdot,1\}$-formula $f(x)$.  So we say that $\langle
\mathbb{N},+,0\rangle$ is a \emph{cyclic monoid}. Another monoid on $\mathbb{N}$
uses multiplication $m\cdot n=0$ if $n=0$, or $m\cdot k+m$ if $n=S(k)$ and here
$1$ is the identity.  In that monoid every $n$ is the evaluation of a
$\{\cdot,1\}$-formula $f(x_2,x_3,x_5,x_7,\ldots)$ where the indicies run over
the primes and $n=f(2,3,5,7,\ldots)$ is to write $n$ as a product of its prime
factors.  So it is the operation more so than the type of data the dictates how
complicated the algebra really is.



As a consequence our identity $1$ may split into a 
collection $\one$ of many identities.  Since our products might not be defined 
we replace equals ($=$) with ``equal when both sides defined'' ($\asymp$)
and thus to write down a category will mean to describe a product 
with the following two rules.
\begin{align*}
    g\cdot (h\cdot k) & \asymp (g\cdot h)\cdot k 
    & 
    \one \cdot g & = \{g\} = g\cdot \one.
\end{align*}
We can be more precise about this because in a category we 
will insist that we are warned when a product will not exist 
so that we may plan around errors instead



The second way that natural numbers from a monoid is to use $e=1$ and the 
following common multiplication of natural numbers.
\begin{align*}
    m\cdot n & = \begin{cases} 0 & n=0 \\ m\cdot k +m & n=S(k).\end{cases}
\end{align*}
Here the numbers $n:\mathbb{N}$ are a product of primes so each 
$n$ is equal to $f(P)$ where $f(X)$ is a $\{\cdot,e\}$-formula over 
a countably infinite set $X$ of variables.  If this monoid were finitely generated 
then there would finitely many primes which is false.  So the negative is true
$\langle \mathbb{N},\cdot,1\rangle$ is not finitely generated.  

Notice in particular that the condition of finite and infinite generation is 
not a quality of cardinality of the sets here are the same.



and find these qualities.  The monoid of natural numbers under addition uses $0$ as 
$e$ and $+$ as $\cdot$ and is generated by $1=S(0)$
An \emph{abstract category} is a the same only the multiplication may not be 
partial and likewise the identity may split into may parts.

partial product that is associative and has an identity.



In the beginning there were functions and functions had composition
which was associative.  Then some thought that there should be types 
and functions were divided into types $f:X\to Y$ and composition became 
untenable with types.  So composition was declared restricted.  
No longer could all functions compose but only those $f:X\to Y$ 
and $g:A\to B$ where $A=Y$.  What would equal types mean?  No one 
was certain but those who insisted on this rule insisted they could 
continue even if that was not clear.  And so it became the rule across 
the land that composition of typed functions would obey typed composition.

Given two types $X$ and $Y$ and an untyped function $f$, we if 
for every $x:X$, $f(x):Y$ then we say that we may introduce a typed function 
also denoted by $f$ of type $X\to Y$.  Define 
\[
    Y^? = Y\sqcup \{\bot\}
\]
where we can think of $\bot$ as a symbol for an error.  Given an untyped 
function $f$ where $x:X$ yields $f(x):Y\sqcup\{\bot\}$, then we say $f$ 
introduces a \emph{partial typed function} and denoted $f:X\to Y^?$.


An \emph{abstract category} is a type $A$ equipped with a binary 
partial operation 

\subsection{Guarded operation}
How do you invert an integer $n$?  You write $1/n$.  That is of course 
unless $n=0$, at which point we do not bother to define an inverse.
So there is a guard $\lhd n=n$ as a function $\lhd:\mathbb{N}\to \mathbb{N}$ such that 
\begin{align*}
    n^{-1} = \text{ if }\lhd n=0 \text{ then } \bot \text{ else } 1/n.
\end{align*}
We could extend this guard to all rational numbers 
\begin{align*}
    \lhd \frac{m}{n} & = m \\
    \left(\frac{m}{n}\right)^{-1} & =
    \begin{cases}
        \bot & \lhd \frac{m}{n} = 0\\
        \frac{n}{m}\in \mathbb{Q} & \text{ else.}
    \end{cases}
\end{align*}
If we want we could think of a function $\Box^{-1}:\mathbb{Q}\to \mathbb{Q}^?$.
This is place where filling the hole with anything that behaves as a number would be to disrupt the intended algebra.  So it is a hole we wish to maintain.

Now to amend my earlier claim, a category is a monoid with \textbf{guarded} holes.  The operation that can have holes is the binary one and so the guard 
now needs to be on the left and the right. 
Now the partial axiom has chirality based on whether we think of $\cdot$ as 
to be read left-to-right ($\rhd$) or right-to-left ($\lhd$).
In order to abbreviate a digression into the topic of grammars and 
signatures, I will write $\Box$ for the position of an input to function.
\begin{gather}
    \tag{Product}
    \Box \cdot \Box : C\times C\to C^?\\
    \tag{Source}
    \lhd \Box :C\to C \qquad \Box\rhd: C\to C\\
    \tag{Target}
    \Box\lhd :C\to C \qquad \rhd\Box :C\to C
\end{gather}
Guards are always total functions.
\begin{align*}
    \tag{Left2Right}
    x\lhd = \lhd y & \Rightarrow x\cdot y \in C\\
    \tag{Right2Left}
    x\rhd = \rhd y & \Rightarrow x\cdot y \in C\\    
\end{align*}
Now the axioms.  In order to state the associative law 
we need to consider that $x,y,z\in C$ where 
each of $x\cdot y$, $y\cdot z$, $x\cdot (y\cdot z)$ and 
$(x\cdot y)\cdot z$ are defined.  We should like this occur 
whenever only the first two are defined and so we need some 
assumptions that could lead there.  One such choice made perhaps by 
Freyd is the following.
\begin{align*}
    \lhd(x\cdot y) & = \lhd(x \cdot \lhd y)
\end{align*}


\end{document}
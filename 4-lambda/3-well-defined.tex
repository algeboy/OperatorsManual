
\section{Operations are well-defined}
The most important aspect of functions to preserve is the reliability of evaluation.
\begin{align*}
    \tag{Well-defined}
    x=\acute{x} & \quad\Rightarrow\quad f(x)=f(\acute{x}).
\end{align*}
Mathematicians call this the ``well-defined'' property but it 
is a special case of a much older principle known as Leibniz' law 
and used actually in reverse, that is, it serves to define equality.
We will come to that later.

There are of course witty counter-examples.
\begin{align*}
    f\left(\frac{a}{b}\right)\defeq a+b.
\end{align*}
Then using $1/2=2/4$ we find that
\begin{align*}
    1/2 = 2/4 & \quad \Rightarrow\quad 3=1+2=f(1/2)=f(2/4)=2+4=6.
\end{align*}
In the exercises you can unwind how this is mostly the 
usual mistake of using nominal judgement ($\defeq$) 
being misinterpreted as operator abstraction ($\mapsto$).
There is no function here at all.

Yet, we ought not dismiss the potential for error hidden 
just below the surface.  After all, what makes us 
confident in functions like $f(x)=x+2$?  If pressed we 
base our judgement on $+$ being well-defined and that 
gets to the heart of whether we trust operators in algebra.

% The first answer is that we must all doubt some aspects 
% of algebraic operations.  Not only do we make mistakes when 
% calculating, we make mistakes when designing operators.  
% A negative sign wrong, a bug in the code, a missing 2.  
% But on such matters either it gets fixed or no one really needed 
% it to begin with.  Lets set that sort of error to the side.

% A real mathematical problem is that operators make new strings 
% from old, and this can make longer strings.  For example, 
% because an operator has no domain by rights it can be applied to 
% itself ``f(f)'', only it gets written as $self:x\mapsto xx$.  Let us 
% apply this to itself.
% \begin{align*}
%     self(n) & = (x\mapsto xx)(12345) & \leadsto (xx)[x\leftrightarrows 12345] = 1234512345.\\
% \end{align*}

% The answer comes in two parts.  First a bit of disappointment.
% There is no reason to expect an operation to come to an end.
% Not only do we write code with bugs, at a deeper level some 
% questions lead to more questions and the computation 
% evolves without bound.  Worse still, Turing taught us that 
% no program exists to detect if another program will halt.
% Thus operators in algebra suffer a fate the perhaps they may 
% never finish.  You may think this fate is reserved for esoteric 
% logical puzzles, but in examples are encountered routinely.

First we should address when to stop an evaluation.

\begin{definition}
    In string of the form $(x\mapsto M)N$ is called a 
    \emph{$\beta$-redex}, and $f(N)$, where
    for some variable $y$, $f:y\mapsto M$,
    is an \emph{$\alpha\beta\eta$-redex}.

    A string $F$ is \emph{$\beta$-reduced}, resp.
    \emph{$\alpha\beta\eta$-reduced}, if no substring is 
    a $\beta$-redex, resp. $\alpha\beta\eta$-redex.
\end{definition}
It is often enough to consider $\beta$-reduction since 
renaming variables and replacing named functions can be done 
as a first single pass on a string.  At this point we would 
very much like to write a theorem that says every string 
has a $\beta$-reduced form.  This, however, is impossible 
in the worst possible way.

\begin{theorem}[$\beta$-reduction is not an algorithm]
    There are unreduced strings where application of $\beta$-reduction 
    never halts.
\end{theorem}

\begin{definition}
    Fix strings $F,G,H$.
    \begin{itemize}
        \item If there are strings 
        $G_1,\ldots,G_{\ell}$ such that 
        \[ G=G_1 \leadsto_{\alpha\beta\eta} G_2 \leadsto_{\alpha\beta\eta}
        \cdots\leadsto_{\alpha\beta\eta} G_{\ell}=F\]
        then write $G\leadsto F$ or $G\leadsto_{\ell}F$.

        \item if $G\leadsto H$ and $F\leadsto H$ then write $G\betaeq F$.

    \end{itemize}
    
\end{definition}
In particular $G\betaeq F$ if after renaming variables and doing some reductions 
they agree.

\subsection{Confluence and Normal Forms}
\begin{theorem}[Church-Rosser]
    If $L\leadsto_{\beta} M$ and $L\leadsto N$ then there is $O$ such that 
    \[M\leadsto_{\beta} O\qquad N\leadsto_{\beta} O.\] 
    In particular, if $F\leadsto S,R$ where both $R$ and $S$ are reduced,
    then $R\betaeq S$ (that is equal up to possibly renaming a variable).
\end{theorem}
\begin{proof}[Sketch of proof]
Suppose that both $L\leadsto M$ and $L\leadsto N$ use a single $\beta$-reduction.
If they are the same reduction then $M=N$ and are done.  Otherwise 
the two reductions occur in different places in the string $L$.  So we might 
think that we could apply both reductions in parallel to arrive at a string $O$
for which $M\leadsto O$ and $N\leadsto O$, for example:
\begin{align*}
    L = \ldots (x\mapsto A)B\ldots (y\mapsto C)D\ldots 
\end{align*}
can reduce to 
\begin{align*}
    M & = \ldots A[x\leftrightarrows B]\ldots (y\mapsto C)D\ldots 
    &
    N & =  \ldots (x\mapsto A)B\ldots C[y\leftrightarrows D]\ldots 
\end{align*}
and then both of these to 
\begin{align*}
O =  \ldots A[x\leftrightarrows B]\ldots C[y\leftrightarrows D]\ldots 
\end{align*}
Unfortunately, if we make a reduction on terms that are nested 
the substitutions may evolve with a different number of total moves.
So instead we must take care to select sequences reductions in $M$ and $N$ carefully 
paired so that when applied in parallel they indeed arrive at the same term.  
The type of parallel reductions that do so were introduced by Tait (1965) and
Martin-L\"of (1971). Following this adaptation the proof follows from induction
on the reduction poset.
\begin{center}
\begin{tikzpicture}
    \node (L) at (0,0) {$L$};
    
    \node (M1) at (-1,-1) {$M_1$};
    \node (N1) at (1,-1.5) {$N_1$};

    \node (M2) at (-2,-2.5) {$M_2$};
    \node (O1) at (0,-2) {$O_1$};    
    \node (N2) at (2,-2) {$N_2$};

    \node (M3) at (-3,-3) {$M_3$};
    \node (O21) at (-1,-3.5) {$O_{21}$};    
    \node (O12) at (1,-3) {$O_{12}$};    
    \node (N3) at (3,-3) {$N_3$};

    \node (M) at (-4,-4) {$M$};        
    \node (N) at (4,-4) {$N$};        
    \node (O) at (0,-7) {$O$};

    \draw[-] (L) -- (M1) -- (M2) -- (M3);
    \draw[-] (L) -- (N1) -- (N2) -- (N3);
    \draw[-] (M1) -- (O1) -- (O12);
    \draw[-] (N1) -- (O1) -- (O21);
    \draw[-] (M2) -- (O21);
    \draw[-] (N2) -- (O12);
    \draw[dotted] (M3) -- (M);
    \draw[dotted] (N3) -- (N);
    \draw[dotted] (O12) -- (2,-4);
    \draw[dotted] (O21) -- (-2,-4);
    \draw[dotted] (M) -- (O);
    \draw[dotted] (N) -- (O);

\end{tikzpicture}
\end{center}


A complete proof can be found in Hindley-Seldin Appendix A.
\end{proof}

This is the first of many \emph{normal-form} theorems in algebra.
Normal forms are unique representations given after rewriting a formula.


\begin{definition}
    A operator is a sentence in $\lambda$-calculus.
    Evaluating an operator is to apply $\beta$-reductions.
\end{definition}

\begin{corollary}
    If a operator evaluates in finite time then whatever process it uses 
    gives the same answer.
\end{corollary}


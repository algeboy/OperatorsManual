\section{Functions}

% So what is the root cause of our paradox of the trapped variable?
% The answer are bound (local) verses free variables.  In fact to 
% really appreciate this we need to decide what a variable is.

% The solution to these problems that we pursue was introduced by 
% Alanzo Church in the 1930's.  It is known as $\lambda$-calculus.
% It is a primitive conception of variables that can be used 
% as the foundation of more profound concepts like Set Theory. 
% Because it is so primitive it also works as a programming language
% and most of today's programming languages are built on a 
% $\lambda$-calculus.  It is also more flexible than Set Theory 
% and gives us ways to describe operations on large objects, like 
% all sets, or all groups, and etcetera.

% \begin{definition}
%     Fix an alphabet of constants and an alphabet of variables.
% \end{definition}



Why did something so basic as evaluating a function fail?  It has to do with
variables coming in two forms: free and bound.  If you program you might think
of a global verses local variable.  Yet unlike programming it requires that we
think of variables as set aside symbols, not empty buckets into which we can
store data.  In particular once you truly understand a variable you will be 
annoyed when you see people write $x\defeq 5$ as no variable can be assigned 
a value.

We are about to define a calculus---a system of calculation and reasoning.
It is known today as $\lambda$-calculus and it was invented by Alanzo Church 
to explain how it is we can describe functions.  You may think this has been 
handled conveniently with sets: A function $f:A\to B$ is a subset $f$ of $A\times B$ where 
\[
    \forall a\in A\exists! b\in B.(a,b)\in f.
\]
It is a nice story, but if you look into the axioms of Set Theory they already
assume there are functions there to explain what sets are (look at the Axiom of
Specification or the Axiom of Replacement).  So functions logically have to exist
before sets.  In fact we already saw two functions $I(x)=x$ and $K_c(x)=c$ that
could be explained with literally no concept of domain or codomain.  Later
within algebra we need this freedom to define higher-order structures such as
functors, which often cannot be explained with sets either.

\subsection{Pure $\lambda$-calculus}

To start out we designate an unbounded alphabet whose members we call
``variables''. That is all it means to be a variable: you are a variable if you
are in the alphabet of variables.  This alphabet can by anything, often it
includes $x,y,z$ but the one condition is that this alphabet be unbounded.  You
might think of this an an infinite alphabet but that starts to depend on notions
of infinity from perhaps sets and becomes circular. Rather all we insist is that
we can always make a new variable different from the ones that are already
present.  We might as well number the variables $x_1,x_2,\ldots$ and get on with
real issues.

\subsection{Substitution of free variables}
Let $L, M$ and $N$ be strings over the alphabet of variables
and fix a variable $x$.  We are about to describe the rules for substitution 
of $x$ by a $N$ in the string $M$.  We use the notation $M[x\leftrightarrows N]$ 
to indicate that we are swapping $N$ for $x$.  Historic 
options include $N/x$ but this runs a risk when used in algebra circles.
The Walrus $\defeq$ is used to define the symbols on the left, what is 
known as \emph{assignment} or \emph{judgemental equality}.  Notice none of 
these assigns a value to variable, rather it assigns a value decorated by 
a prescribe substitute.  Think of Walrus $\defeq$ as naming one set of symbols 
by another.
\begin{description}
    \item[Free.match] $x[x\leftrightarrows N]\defeq N$.
    \item[Free.other] $M[x\leftrightarrows N]\defeq M$ if $M$ is in the variables alphabet (and 
    because we already will have intercepted the case $M=x$ in the above case we know $M\neq x$).
    
    \item[Free.recurse] $(LM)[x\leftrightarrows N]\defeq L[x\leftrightarrows N]M[x\leftrightarrow N]$
\end{description}
So far we have designated substitution in the obvious way: replace very 
symbol in a string by its replacement.  However these symbols do not as 
yet reach the level of functions because the are static, they are just 
apiece of data not a way to describe a change in data.

\subsection{Substitution of bound variables}
To define a function we need inputs and outputs, and that means to specify that
some variables are the ones we want to replace.  Often we denote this by given a
function a name, e.g.\ $I(x)=x$ gave the function the name $I$ and the proceeded
to identify its variables. Likewise $K_c(x)=c$ is a function of $x$, but not of
$c$.  Finally the names are not important.  We are just as informed to write
$x\mapsto x$ or $x\mapsto c$.  Church used the notation $\lambda x.x$ and
$\lambda x.c$ because it fits in with other notation such as $\forall x.P$ and
$\exists x.Q$ and so on.  In programs these are known as \emph{anonymous
functions} or as $\lambda$'s.   The variable set apart as the input is said to
be \emph{bound} to the function, it now has a specific role to play similar to
how $\forall x.P$  and $\exists x.Q$ bind $x$ into a special role as well.
Variables that do not have special roles are called \emph{free}.  Note that the
special role of $x$ in $x\mapsto M$ is limited to $M$.  We can say $M$ is the
\emph{scope}.  For this reason programs typically refer to bound variables a
\emph{local}.

Now let us describe how we substitute into functions
\begin{description}
    \item[Bound.match] $(x\mapsto M)[x\leftrightarrows N]\defeq (x\mapsto N)$
    \item[Bound.trapped]
    if $x$ is a free variable in $M$ and $y$ is a free variable in $N$ then 
    $(y\mapsto M)[x\leftrightarrows N]\defeq (y'\mapsto (M[y\leftrightarrows y'])[x\leftrightarrows N])$ where 
    $y'$ is a variable distinct from any in $M$ and $N$ (which exists 
    as there are unbounded numbers of variables);

    \item[Bound.other] $(y\mapsto M)[x\leftrightarrows N]\defeq (y\mapsto M[x\leftrightarrows N])$.
\end{description}

\subsection{Applied $\lambda$-calculus}
If all we allow in our reasoning is an alphabet of variables then we describe
the $\lambda$-calculus as \emph{pure}.  Meanwhile in real life we often are
willing to stake our understanding on several established concepts, for example
$0,1,2,3,\ldots$ or the number $\pi$, the color blue.  When we want to add this
to our study we create a \emph{separate} alphabet of constants.  A system with
constants is called an \emph{applied $\lambda$-calculus}.  This requires just one 
further substitution rule:
\begin{description}
    \item[Constant] $c[x\leftrightarrows N]\defeq c$ when $c$ is from an applied constant alphabet. 
\end{description}

It is the job of a programming language to implement a version 
of substitution that follows these rules.  Once done the notation will take 
on the usual character of the programming language but often the notation 
comes close.  Here are some popular variations to try.
\begin{center}
    \code{lambda x.x+2}
    \code{x => x+2}
    \code{x |-> x+2}
\end{center} 
In older languages you may need to look for this under the term \emph{anonymous function}
since these are functions without names.

\subsection{Reduction}
The purpose of binding a variable is to control substitution but it may seem 
with rules like $(x\mapsto M)[x\leftrightarrows N]\defeq (x\mapsto M)$ that 
we will never get to evaluate a function.  To the contrary we are simply forcing 
ourselves to explicitly declare evaluation as a step, a computation, which 
very well may require work and alter out data.  This is done with a \emph{reduction}
denoted by $\leadsto$.
\begin{description}
    \item[$\alpha$-reduction] given variables $x$ and $y$, with $y$ not free in $N$,
    \[(x\mapsto M)\equiv (y\mapsto M[x\leftrightarrow y]),\]
    which simply renames the variable $x$.

    \item[$\beta$-reduction]
    $(x\mapsto M)N\leadsto M[x\leftrightarrow N]$, which replaces all (free)
    instances of $x$ in $M$ with $N$.  This is the main tool, it is evaluation.
    
    \item[$\eta$-reduction]
    $(x\mapsto M)\leadsto M$, which unbinds the variable $x$.
\end{description}

The $\beta$-reduction rule is what finally gives meaning to ``evaluate'' a function.

\begin{theorem}[Church-Rosser]
    If a formula reduces in finite number of steps then all ways to reduce it give the same 
    final answer.
\end{theorem}


\begin{definition}
    A function is a sentence in $\lambda$-calculus.
    Evaluating a function is to apply $\beta$-reductions.
\end{definition}

\begin{corollary}
    If a function evaluates in finite time then whatever process it uses 
    gives the same answer.
\end{corollary}

\subsection{Combinators}
At some point you will expect your function might be used and if it is done 
by a microprocessor that will mean translate all these substitution into a 
sequence of basic operations, called \emph{gates} or simply \emph{combinators}.
Of course the substitution rules are few so we can encode each by a small 
number of combinators.  Thee in fact suffice, two which we have already meet, 
$I$ and $K$.  We add to this the final pure combinator 
\begin{align*}
    S_{fg}(x) & \defeq (f(x),g(f(x))).
\end{align*}
With combinations of $S$, $K$, and $I$ we can enact all operations of substitution
and therefore all of $\lambda$-calculus is a sequence of $SKI$'s, the ``Ski'' combinators.
In fact combinators do not even require variables to be defined, we just give their 
$\beta$-reductions and that is all we need.  Here they are.
\begin{align*}
    IM & \leadsto M\\
    KMN & \leadsto M\\
    SLMN & \leadsto LM(LN)
\end{align*}
In practice writing programs purely from SKI is tedious and inefficient.  We can 
instead rely on circuits that are compute equivalent values and take advantage 
of realistic bounds on the data.  For this reason, microprocessors and 
quantum computers alike add many more combinators to achieve useful tasks 
such as adding multiplying, and, or and more.  These are called \emph{applied combinators}.
With these concepts in mind we can lay out the layers we find.

\begin{itemize}
    \item Turing machine: universal computer in that it can be reprogrammed to 
    mimic all others.

    \item Combinators: a universal assembly language, in that it reduces all 
    programs into a sequence of primitive steps from a finite list of primitives.

    \item $\lambda$-calculus: a universal programming language, in that it can 
    encode any computable function.

\end{itemize}


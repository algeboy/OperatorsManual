
\section{Elimination}
Let us now try to eliminate a string.  Recall elimination is the term used 
when we consume the structure use to introduce the data.  One example could be to imitate 
our addition of natural numbers, only now we seek an analog to ``add''
strings.  Perhaps concatenation comes to mind.  Using our $[a,b,c]$ 
alphabet ths addition could look like this.
\begin{align*}
    s+t & \defeq \begin{cases}
        t & s=\epsilon\\
        a(w+t) & s= a(w:String)\\
        b(w+t) & s= b(w:String)\\
        c(w+t) & s= c(w:String)\\
    \end{cases}
\end{align*}
Thus,
\begin{align*}
    cab+bab & = c(ab+bab)+c(a(b+bab))=c(a(b(\epsilon+bab)))\\
        & = c(a(b(bab)))=cabbab.
\end{align*}
Once more we can do the same in code by matching the pattern.
Listing~\ref{lst:peano} shows what this might look like for 
strings fixed on the alphabet [a,b,c].  Start to imagine what 
changes you would make to work with other alphabets, and 
eventually interchangeable alphabets.
\begin{lstfloat}
\begin{center}
\begin{Fcode}[]
cat:String_abc->String_abc->String_abc
cat Empty t   = t
cat 'a' tail t = 'a' (cat tail t)
cat 'b' tail t = 'b' (cat tail t)
cat 'c' tail t = 'c' (cat tail t)
\end{Fcode}
\begin{Pcode}[]
def cat(s,t:String_abc):String_abc =
  match s with 
    Empty() => t
    A(tail)=> A(cat(tail,t))
    B(tail)=> B(cat(tail,t))
    C(tail)=> C(cat(tail,t))
\end{Pcode}
\end{center}
\caption{Concatenation of strings over the alphabet [a,b,c] is a variation on 
Peano's addition of natural numbers.}
\label{lst:peano-elim}
\end{lstfloat}
    
% This algorithm remains predictable because of what we assumed about 
% primitive recursion.  That is $m+0$ is defined and if $m+k$ is defined 
% then $m+S(k)\defeq S(m+k)$ is therefore defined.  This is what is 
% known as \emph{recursion}.  It hinges on the fact that the proper 
% \[    P(n) :\equiv m+n \text{ defined}
% \]
% we want is known at step $k$, we write $P(k)$, and so we can 
% use that knowledge to deduce the $m+S(k)$ is defined, in other wards $P(S(k))$ follows.
% \begin{gather*}
%     \begin{array}{rl}
%     P(0)& \\
%     P(k) & \Rightarrow P(k+1)\\
% \hline 
% \forall n:\mathbb{N} & P(n).
%     \end{array}
% \end{gather*}
% This is sometimes known as the principle of induction.  It is more of a program than 
% an assumption of fact.  When we have a $n$ we want to confirm $m+n$ is defined for,
% say $n=SS0$, then we simply apply the rules given repeatedly:
% \[
%     m+SS0=S(m+S0)= SS(m+0)=SS(m)
% \]
% By rule \code{S<Nat>} we get that $SS(m):\mathbb{N}$.  As expected it added two strokes to the 
% tally.   This process of consuming the information given to introduce $n:\mathbb{N}$ is known 
% as \emph{elimination}.   We will see several examples introduction 
% and elimination patterns. 

% If we stand back to observe the entire process, grammars allow us to introduce numbers, 
% such as $2\defeq SS0:\mathbb{N}$.  Then when we have $2:\mathbb{N}$ we can 
% use the data behind its introduction (here two successors and a $0$) as instructions 
% to follow when producing a new outcome, perhaps a new type of data.  This is known 
% as \emph{eliminating} the introduction.  We will see several examples introduction 
% and elimination patterns. 

% \subsection{Type rules}
% We are engaged in forming a type of data, natural numbers.  Some times 
% of data will depend on others, for example tuples $\mathbb{N}^k$ depend on 
% $k$.  So even in the form of types we need to pause and 


% If we break down the 
% steps we first had to give the type a name, 
% If we break this process down here are the summary points.  Note that we use Frege's 
% notation where symbols $p_1,\ldots,p_n$ that lead to symbols $q$ are separated by 
% $\vdash$, so $p_1,\ldots,p_n\vdash q$, which is also often written 
% \begin{align*}
%     \frac{p_1,\ldots,p_n}{q}\qquad 
%     \begin{array}{c}
%         p_1\\
%         \vdots \\
%         p_n\\
%     \hline 
%         q 
%     \end{array}
% \end{align*}
% Note that if nothing is required to lead to $q$ we can write $\vdash q$.
% The symbol $\vdash$ is known formally as \emph{entailment} and while it may 
% appear at first like an implication, it is here being used more in the role of 
% defining rules.
% \begin{gather}
%     \tag{$\form{\mathbb{N}}$}
%     \vdash \mathbb{N}:Type\\
%     \tag{$\intro{\mathbb{N}}$}
%     \vdash 0:\mathbb{N} 
%     \qquad \frac{k:\mathbb{N}}{S(k):\mathbb{N}}\\
%     \tag{$\elim{\mathbb{N}}$}
%     \begin{array}{rl}
%         n &:\mathbb{N}\\
%         P(k) &\vdash P(S(k))\\
%     \hline 
%         P(n)
%     \end{array}
% \end{gather}

% \subsection{Wrapping up errors}
% \begin{center}
% \begin{minipage}{0.35\textwidth}
% \begin{Fcode}
% data Maybe B = 
%     | Just (b:B)  
%     | None 
% \end{Fcode}
% \end{minipage}
% ~
% \begin{minipage}{0.64\textwidth}
% \begin{Pcode}
% class Option[B]
%     case Some(b:B) extends Option[B]
%     case None() extends Option[Any]
% sealed
% def wrap(f:A->B):f:A->Option[B] = 
%     a ->
% \end{Pcode}
% \end{minipage}
% \end{center}
    

\section*{Exercises}

\begin{enumerate}
    \item Write down a grammar to accept natural numbers as digits.  Remember $07$ is not proper substitute for $7$.
    \item Make a grammar to describe rational numbers.
    \item Make a grammar to describe decimal numbers, call this \code{<Reals>}.  
    \item Augment your \code{<Reals>} to include constants for $\pi$ and $e$.
    \item Give a full grammar for real polynomials including conventional short-hand 
    such as $x$ for $x^1$ and $ax$ instead of $(a)(x)$.
    
    \item Suppose that we want to able to add a tally mark to either the left (L) or right (R).
    Define a grammar that allows.  Explain why this is not a model for the natural numbers.
%     Consider a natural number inspired grammar where you can apply a successor
% $L$ on the left or $R$ on the right.
% \begin{center}
% \begin{Gcode}[]
% <LRNat> ::= 0
%             | L <LRNat>
%             | <LRNat> R
% \end{Gcode}
% \end{center}
% Describe what words are accepted by this grammar and draw its word graph.  Is this a model of the 
% natural numbers?
        
    \item Turn the following function into code in both functional and procedural dialects.
    \begin{align*}
        f(n) & = \begin{cases}
                    0 & n=0\\
                    S0 & n=S(k)
        \end{cases}
         =\begin{cases} 0 & n=0 \\ 1 & \text{else}\end{cases}.
    \end{align*}
    
    \item Make a concatenation function for the general \code{String} and 
    provide a demonstration in some pseudo-code.
    \item Mimic the String data type to make a list of integers (that is 
    switch from the alphabet to integers).

    \index{generics}
    \item Mimic the String data type to make a list of fixed by 
    unknown data of type $A$, call it \lstinline{List[A]}.\footnote{
    Alternatives include \lstinline{List a} and \lstinline{List<A>}. 
    Search for \emph{generics} in your programming language to learn more.
    }

\end{enumerate}
\index{nil}\index{cons}\index{list}
Historically \lstinline{Empty} for lists is called \lstinline{Nil} 
and \lstinline{Prepend} is called \lstinline{Cons} or simply denote $:$.

\section{Conext-free grammar}\label{sec:cfg}

The linguist Chompsky sparked a revolution in the study of language by detailing
grammar in a mathematical context.  To make a grammar there is one forgoing
assumption that we can take an alphabet of fixed symbols, traditionally denoted
$\Sigma$, and from that have some concept of words over that alphabet.  For
example $\Sigma\defeq\{a,b\}$ would mean that $aabaaabb$ is a string of type
$\Sigma^*$, the $*$ is known as the \emph{Kleene star},\index{Kleene star}.  We
write $aabaaabb:\Sigma^*$.  For convenience we let $\epsilon$ denote the empty
string.  More on this in a moment.


\subsection{Formal grammar}
A \emph{context-free formal grammar} consists of 
\begin{itemize}
    \item \index{tokens}\index{terminals} a (finite) alphabet $\Sigma$ 
    of symbols called \emph{terminal}(or tokens) from which all strings will be made.
    \item \index{non-terminal} a (finite) alphabet $N$ of non-terminal symbols
    used to name how tokens can be combined.
    \item \index{productions} productions $P$, a list of ordered pairs 
    $(a,w)$ where $a:N$ and $w:(\Sigma\sqcup N)^*$, to explain allowed 
    combinations.
    \item One or more non-terminals $S$ designated as the \emph{starting}
    symbols.
\end{itemize}
If there is a single start then the grammar is called \emph{homogeneous}
otherwise it is known as \emph{heterogeneous} (or \emph{multi-lingual} grammars).

The notation we have been using is comes from early computer
scientist who were exploring ways to explain programming languages. The notation
 is the Backus-Naur Form (BNF).\index{Backus-Naur}\index{BNF} In
particular we list the two alphabet $\Sigma$ and $N$ and then list the
productions $(a,w)$ are recorded as \code{<a> ::= w}.  Starting symbols are
denoted with \code{<<a>> ::= w}.  This hides the individual components from 
view but the form is still evident and the BNF notation is for most readers 
a suitably simpler conception.



\begin{remark}{Meta-language}
    The symbols \lstinline{::=}, \lstinline{<Token>} and \lstinline{|} in
    Backus-Naur Form (BNF) are meant to clarify how we write down grammar.
    Standing back and looking at BNF we see  it too is actually subject to its
    own gammar, admittedly basic and fixed in length.  But you can be forgiven
    for wondering if this is all circular reasoning.  When this occurs,
    mathematicians like to attach the word \emph{meta}, which means literally
    ``self-referential''. So BNF would be called a \emph{meta-language}.
    Sometimes self-referential can be turned into paradoxes (Russell's paradox,
    G\"odel's Incompleteness, Turing's Halting problem).  So you may worry.  But
    I suppose if you didn't believe in language, why would you be reading?\\
\end{remark}

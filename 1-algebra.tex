\chapter{What is algebra?}

Algebra is the study of equations, for the most part equations involving variables.
That is because applications have unknowns and if 
the shape of the equation can tell us anything about the 
options to solve it, we shall want to take advantage of this.
Witness how we solve $ax^2+bx+c=0$ in one strategy despite it covering 
an infinite number of equations.

You probably already know every color, shape, and pattern of 
equation you will ever need.  Compare these two equations
\begin{align*}
    x^2+y^2 & \equiv 0 \pmod{541} 
    & 
    \frac{\partial^2 f}{\partial x^2}+\frac{\partial^2 f}{\partial y^2} & =0.
\end{align*}
Can you stop yourself from seeing them as related?
These equations concern entirely different things.  On the left $0$ can 
equal 541.  On the right $0$ is functions on the $xy$-plane.
Yet, the similarities as equations shine through.  Why? Is it because $0$, 
$+$, and  powers of $2$ are general concepts, ``abstractions''.  This doesn't connect them to
polarizing art movements nor render the concept inapplicable.  Abstract here,
and everywhere, means to study by limited attributes.  That's how we do all
math and science.  So when we abstract the equation on the left we forget
about mod 541 and the precise meaning of these numbers.  On the right we forget
about functions and the notions of derivatives.  We are left with just $0$, 
$+$, and squares and where they sit.  We abstract both to a common equation
\[
    x^2+y^2=0.
\]
In fact, even the equality was an abstraction which could vary from context to context.
With such flexibility a small number of symbols and their grammar are enough to capture 
the huge variety of equations we encounter in real life.

Now since every symbol in an equation is a variable we have new powers 
to solve equations.  Look to the humble 
\[
    x^2+1=0.
\]
By our own view, right now this is nothing but variables, so it means nothing to
solve this.  But, drop this into a context such as decimal numbers $\mathbb{R}$
and the understanding is to replace $1\defeq 1.0$, $0\defeq 0.0$, $+$ is
substituted for addition of decimals, and square is by multiplying decimals.
Equality now means two equal decimals, or in practice two decimal numbers that are close enough 
to be considered equal.  The only remaining unknown is $x$, but as everyone 
knows, $0\leq x$ or $x<0$ so in both cases $-1<0\leq x^2$.
% Algebra even gives 
% us the tools be prove this: real numbers are totally ordered so $0\leq x$ or $x< 0$.
% Square both sides we find $0\leq x^2$ and $0<x^2$ in both cases.  So the defects in $x^2=-1$
% are that we have order.  Getting ride of order we may find a number to solve the equation.

The power of variable everything is that we are not stuck with the real numbers.
Let us replace everything with complex numbers $\mathbb{C}$. Substitute $0\defeq
0+0i$, $1=1+0i$, $(a+bi)+(c+di)\defeq (a+b)+(c+d)i$, and
$(a+bi)^2=(a^2-b^2)+2abi$.  Now we find $\pm i$ are the solutions. Solutions do
exist!  Since they exist we can return to a problem and ask if the solutions we 
found will do the job.  Maybe not, or maybe we should revisit our models and 
see these solutions as predicting the presence of previously unknown realities.

Why stop here? Quaternions have $\hat{\i}^2=\hat{\j}^2=\hat{k}^2=-1$;
so, at least 6 solutions to $x^2+1=0$.  Even more.  Try $(2\times 2)$-matrices, I
bet you can find infinitely many matrices $M$ where $M^2=-I_2$.  This is the method
of algebra: dream up new numbers that might be used to solve equations.  Alter
their properties, e.g.\ drop the order of real numbers and you can get complex
solutions.  Drop commutative multiplication of complex numbers and you might get
infinite solutions.  Learn enough by this process and we can begin to predict if
solutions are to be expected and fathom algorithms to find them when they do
exist. When the solutions become infinite we find ways to parameterize them with
smaller data such as a basis.

Don't forget equality is variable too!  Suppose we wanted to solve $x^{541}+x+1=0$
using only integers.  Replace equality 
with $\equiv$ modulo 2 and ask for a whole number $x$ that solves
\begin{align*}
    x^{541}+x+1\equiv 0\pmod{2}.
\end{align*}
All integers in this equality become either $0$ or $1$, but neither will solve 
this equation.  By varying equality we confirm there are no solutions.


\begin{quote}
    \textbf{The power of algebra is that every symbol 
    in an equation is a variable, especially the equals sign.}
\end{quote}

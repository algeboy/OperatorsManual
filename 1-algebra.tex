\chapter{What is algebra?}

Algebra is the study of equations, for the most part equations involving variables.
That is because applications have unknowns and if the
the shape of the equation can tell us anything about the 
options to solve it we shall want to take advantage of this.
Witness how we divide $ax^2+bx+c=0$ into solutions that are 
real or complex based on $b^2-4ac\geq 0$.  This is an infinite family 
of equations captured by 2 strategies, one when we accept complex numbers.

You probably already know every color, shape, and pattern of 
equation you will ever need.  Compare these two equations
\begin{align*}
    x^2+y^2 & \equiv 0 \pmod{541} 
    & 
    \frac{\partial^2 f}{\partial x^2}+\frac{\partial^2 f}{\partial y^2} & =0.
\end{align*}
These equations concern entirely unrelated contexts, for example $0$ on the
right is a whole number, whereas $0$ on the left is the function $0(x,y)=0.0$
Yet, the similarities as equations shine through.  This is because $0$, 
$+$, and  powers of $2$ are abstractions.  This doesn't connect them to
polarizing art movements nor render the concept inapplicable.  Abstract here,
and everywhere, means to study by limited attributes, which happens everywhere
in math and science.  So when we abstract the equation on the left we forget
about mod 541 and the precise meaning of these numbers.  On the right we forget
about functions and the notions of derivatives.  We are left with just $0$, 
$+$, and squares and where they sit.  We abstract both to a common equation
\[
    x^2+y^2=0.
\]
In fact, even the equality was an abstraction which could vary from context to context.
With such flexibility a small number of symbols and their grammar are enough to capture 
the huge variety of equations we encounter in real life.

Now since every symbol in an equation is a variable we have new powers 
to solve equations.  Look to the humble 
\[
    x^2+1=0.
\]
By our own view, right now this is nothing but variables, so it means nothing to
solve this.  But, drop this into a context such as decimal numbers $\mathbb{R}$
and the understanding is to replace $1\defeq 1.0$, $0\defeq 0.0$, $+$ is
substituted for addition of decimals, and square is by multiplying decimals.
Equality now means two equal decimals, or in practice two decimal numbers that are close enough 
to be considered equal.  The only remaining unknown is $x$, but as everyone 
knows, we wont find a solution as no square decimal is -1.

The power of variable everything is that we are not stuck with the real numbers.
Lets replace everything with complex numbers $\mathbb{C}$. Substitute $0\defeq
0+0i$, $1=1+0i$, $(a+bi)+(c+di)\defeq (a+b)+(c+d)i$, and
$(a+bi)^2=(a^2-b^2)+2abi$.  Now we find $\pm i$ are the solutions. Solutions do
exist!  Since they exist we can return to a problem and ask if the solutions we 
found will do the job.  Maybe not.  

Why stop here? Try quaternions, or $(2\times 2)$-matrices.  These each have
addition, 0 and squares.  Now we learn there can be infinitely many solutions to
that equation.  If complex solutions were not right perhaps one of these
infinitely many quaternions or matrices will do.  This is the method of algebra:
find all the values that can be substituted into an equation to see what makes
solutions possible and how they behave.  Learn enough by this process and we can
begin to predict if solutions are to be expected and fathom algorithms to find
them when they do exist. When the solutions become infinite we find ways to
parameterize them with smaller data such as a basis.

Don't forget equality is variable too!  Suppose we wanted to solve $x^{541}+x+1=0$
using only integers.  Replace equality 
with $\equiv$ modulo 2 and ask for a whole number $x$ that solves
\begin{align*}
    x^{541}+x+1\equiv 0\pmod{2}.
\end{align*}
All integers in this equality become either $0$ or $1$, but neither will solve 
this equation.  By varying equality we confirm there are no solutions.


\begin{quote}
    \textbf{The power of algebra is that every symbol 
    in an equation is a variable, especially the equals sign.}
\end{quote}

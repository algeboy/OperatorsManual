\chapter{Substitution}

Every symbol in an equation is variable, meaning it can be substituted.
You probably understand this rather well so you will make quick work 
of a couple examples
\begin{align*}
    I(x) & = x & 
    K_c(x) & = c.
\end{align*}
You may call $I$ the identity function and the $K$ constant functions, 
but with excessive mathematical training you are likely to overthink 
the situation.  These two formulas have no use for domains and codomains for 
instance.  We can safely use these even when there are no sets or mathematics 
around to make sense of deeper concepts.
Try some substitutions, I tried $I(3)=3$, $I(\clubsuit)=\clubsuit$.
I found $K_3(2)=3$ and $K_3(\clubsuit)=3$ as well.  I even tried 
$K_{\clubsuit}(2)$ and got $\clubsuit$.  I changed $x$ for $y$, 
$I(y)=y$, and $d$ for $c$, $K_d(x)=d$. Next I substited $x$ for $c$ and got
\begin{align*}
    K_x(x)=x=I(x).
\end{align*}
Wait, that cannot be true.  How is a constant function suddenly the 
identity function?  

It seem substitution is not so basic after all.
I just walked you into the paradox of the trapped variable.
As my philosophy friend and colleague Professor Dustin Tucker likes to say, 
``any system studied long enough will reveal a paradox.'' 
A paradox may sound bad, but it is not a contradiction.  Rather 
it is an indication that we are missing some constraints that 
avoid a contradiction.  How we avoid paradoxes is often non-unique 
and can be come the subject of controversy and intrigue.

The rules I use to avoid trapped variables where written down in a book by
Curry-Feys, but others likely knew them before hand.  They work by separating 
alphabets, one for constant symbols ``atoms'', and a disjoint alphabet for 
variables, which we assume to be unbounded (is that the same as infinite?).
Strings $M$ are lists from the two alphabets.  Then we substitute recursively 
as follows.
%% flow chart
\begin{figure}[!htbp]
    % \begin{tikzpicture}
    %     \node
    % \end{tikzpicture}
\end{figure}
\begin{enumerate}
    \item if $M$ is an atom
\end{enumerate}



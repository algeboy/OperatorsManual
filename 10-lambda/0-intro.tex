\begin{quote}
% As my philosophy  colleague Professor Dustin Tucker says, 
\emph{A system studied long enough reveals its paradoxes.}\\
~\hfill-- Dustin Tucker
    % \emph{The power of algebra is that every symbol in an equation is a 
    % variable...}
\end{quote}
Algebra's need operators, and operators---whatever they are---turn inputs into
outputs.  For some that sounds like a function, for others function is reserved
for a set-based notion.  Operators in algebra will be needed for things that
come before (e.g. $\wedge,\vee,\Rightarrow$) and after sets (functors and beyond). As far as
simplicity goes the following two example operators win on the grounds of
requiring nothing, not even a domain or codomain.
\begin{align*}
    \CombI(x) & \defeq x & 
    \CombK{c}(x) & \defeq c.
\end{align*}
You may call $\CombI$ the \emph{identity} operator and the $\CombK{}$ \emph{constant} operator,
a different one for each $c$.  You can put anything into these.
Try some substitutions, I tried $\CombI(3)=3$, $\CombI(\clubsuit)=\clubsuit$.
I found $\CombK{3}(2)=3$ and $\CombK{3}(\clubsuit)=3$ as well.  I tried 
$\CombK{\clubsuit}(2)$ and got $\clubsuit$.  I changed $x$ for $y$, 
$\CombI(y)=y$, and $d$ for $c$, $\CombK{d}(x)=d$. Next I substituted $x$ for $c$ and got
\begin{align*}
    \CombK{x}(x)=x=\CombI(x).
\end{align*}
Now we have a true problem: a constant operator should not equal 
an identity operator.  This is the \emph{paradox of the trapped (or captured) variable}.
\index{trapped variable}\index{captured variable|see{trapped variable}}

\index{paradox}\index{inconsistent}
Paradoxes (para = distinct + dox = opinion) are places where logical reasoning
leads to two seemingly opposing conclusions.  We could just as well call this an
inconsistency and declare the topic dead, but usually mathematics reserves the
word ``paradox'' for settings where we could avoid the inconsistency by revisiting
some fundamental notion and choosing a narrower interpretation.  True
inconsistency we reserve for cases where our only option is to throw out the
system.  When we just remove the means for one of two options to 
emerge we are picking a side before it can mature: pull the weeds leave the flowers.
Yet, no botanist can define a weed, a weed is just what you don't want to grow.
Fixing a paradox is doctrine, a belief.  So it 
is that solutions to paradoxes have schisms.  My doctrine is that of Church,
Alanzo Church.\Church

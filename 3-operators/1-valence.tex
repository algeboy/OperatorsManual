

\section{Fixed Valence Operators}
While programming languages have entertained the possibility of 
variable valence for operators (look into \emph{variadic} evaluation)
by far the most common approach in scientific abstraction of operators 
is to study those with a fixed valence.  Often by composing several 
operators in a row we obtain the same effect as  variadic 
operators.  

\index{bivalent!opeator}\index{binary operator|see{bivalent operator}}
Addition from here on out will be bivalent (binary), meaning 
requiring 2 inputs, and with infix grammar $\Box +\Box$.
Since we are evolving, we may as well permit multiplication as a bivalent operator
symbol, changing the signature to $\Box \cdot \Box$, i.e. $2\cdot 4$; or
$\Box\Box$, e.g. $xy$.   Avoid $\Box\times \Box$, we need that symbol elsewhere.
These days composition $\Box\circ\Box$ is written as multiplication; so, you can
use that symbol however you like.  Addition is held to high standards in algebra
(that it will evolve into linear algebra).  So when you are considering a binary
operation with few if any good properties, use a multiplication inspired
notation instead.   




\index{univalent!opeator}\index{unary operator|see{univalent operator}}
Valence 1, also called \emph{univalent} or \emph{unary}, operators include the negative sign $-\Box$ to create 
$-2$.  The transpose $A^{\dagger}$ of a matrix $A$ is a unary operator.  Programming languages add several others 
such as \lstinline{++i, --i} which are said to \emph{increment} 
or \emph{decrement} the counter i (change it by $\pm 1$).

Programs also exploit a trivalent (ternary) operator:
\begin{center}
    \lstinline[language=Sava]{if (...) then (...) else (...)}
\end{center}
The words, while helpful, are unimportant.  Some programming languages 
replace them with symbols emphasizing their ``opperatorness'' 
\begin{center}
    \lstinline[language=Sava]{_?_:_}
\end{center}
Here for example is division with remainder of positive integers
\begin{center}
\begin{lstlisting}[language=Sava,mathescape]
div(m,n)=(m>=n)?(div(m-n,n)+(1,0)):(0,m)
\end{lstlisting}
\end{center}


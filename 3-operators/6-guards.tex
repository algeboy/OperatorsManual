



\section{Operators that guard}
When we divide we avoid division by $0$ (otherwise $0=0\frac{x}{0}=x$ so that every number
would be $0$ and we have no use for that sort of number system).  
Yet the problem is more pronounced.  With integers we cannot divide by anything other 
than $\pm 1$, so few that we in general through away division with integers.  
Now what happens in most of algebra is a mix of a good number of things with which we can 
divide but also a good number which we cannot use.  This means we cannot affort list 
the exception one by one, we need something stronger.

Consider matrices.  We decide if we can divide by $M$ if $\det(M)$ is invertible.
Here $\det$ is an operator on matrices.  It is an operator that informs us about 
what types of algebra we can perform on $M$.  It is \emph{guarding} us from mis using 
inverses.

A similar situation occurs with composing functions $f$ and $g$.  We need 
two guards: $\dom f$ and $\codom g$ that convert $f$ and $g$ to some other data 
where we can ask is $\dom f=\codom g$ and if so we get to use $f\circ g$.
The more algebraic operators we encounter the more often they are not total, 
meaning they cannot be applied everywhere, and the responsible step to take 
then is to add in further operators to serve as guards for when we can use the 
desired operations.
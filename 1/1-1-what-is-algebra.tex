\documentclass{beamer}

% \usetheme{focus}
\input{../beamer-theme.sty}
\usepackage{tikz}

\title{What is algebra}
\subtitle{}
\author{CC-BY, James B. Wilson, 2023}
% \titlegraphic{\includegraphics[scale=1.25]{focus-logo.pdf}}
\institute{Department of Mathematics\\ Colorado State University}
\date{August 2023}

\begin{document}

\begin{frame}
    \maketitle
\end{frame}

\begin{frame}

    \begin{tbee}{What variables do you see?}
        \[
        2x^2+bxy+cy^2 = 0
        \]
    \end{tbee}
    \only<2->{Obviously $x, y$\\}
    \only<3->{Constants $b,c$ unspecified, 
    so also variable but somehow ``bound'' to context\\}
    \only<4->{$0$...could be $0$ integers, 
    is  $2=2+0i$ complex, 
    maybe $0=\begin{bmatrix}0 & 0 \\ 0 & 0 \end{bmatrix}$...
    more variables, with conditions.\\}
    \only<5->{$+$, $\times$? Those also vary with other numbers, maybe squaring 
    is a second derivative.\\}
    \only<6->{What's left, $=$? How can that vary?...oh maybe its mod 12.}
\end{frame}

\begin{frame}
    Max variability allows for wide ranging commonality.
    \begin{tbee}{Conic sections}
        \[
            x^2-y^2  =0
        \] 
    \end{tbee}

    \begin{tbee}{Wave propogation}
        \[
        \frac{\partial^2 f}{\partial x^2}-\frac{\partial^2 f}{\partial y^2} = 0
        \] 
    \end{tbee}

    \begin{tbee}{Factoring by quadratic sieve}
        \[
        m^2-n^2  \equiv 0 \pmod{65398}
        \]
    \end{tbee}
    
    
\end{frame}
\begin{frame}
    \begin{tbee}
        \textbf{The power of algebra is that every symbol 
        in an equation is a variable, especially the equals sign.}
    \end{tbee}
\end{frame}
\end{document}
As a start, it matters first to admit that functions do not need 
domains.  Surprised? Disagree?    Suppose in high school we asked 
a student to  explore
\[
    f(x) = \frac{\sqrt{e^x-1}}{x^2-x-6}.
\]
We might pass this to a graphing calculator, or try some inputs. Along the way
we see the calculator gives up on plotting values $x=(-\infty,0],3$.  Yet, all
we gave it was the formula without prior knowledge of these conditions.  
Be honest with yourself, wouldn't you have thought you could use some of 
those numbers until you tried it as well?  

This problem did not exist in the 1900's.  It is a recent out-growth of strong
personalities.  Luminaries like David Hilbert urged mathematics to unifty around
a single ``deductive system''.  Hilbert's functions no longer did anything.
Instead, they were reduced to pairing up the before and after photographs.  Like
stumbling onto a crime scene where you know what was taken but have no clues
about the robbers or how they pulled it off.  This crime-scene styled view of
functions was helped along by the homogenizing tendencies of the Bourbaki who
recorded the first true textbooks an ``all'' mathematics. Post World War II
college enrollment spikes then triggered monopolies for textbook publishers
crowding out the majority but disbursed approaches of Church, Kolmogorov,
Brower, and Heyting.  Some of us privately resonate however with the following mindful
lament.
\begin{quote}
    A function that doesn't function,\\
    should not be called a function.
    \hfill Thorsten Altenkirch
\end{quote}

It is too much of a fight to reclaim the 
name of ``function'' as whatever process you just observed without a domain 
or codomain.  Some call it a \emph{partial function} which reads like defeat.
A suitable retreat is to call this an \emph{operator} because 
it operates on data and perhaps changes it, which after all is what a function should have done.


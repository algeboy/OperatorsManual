\section{Homogeneous Operators}
The above examples suggest that addition is a stand in, a variable.  It is not however 
an ordinary variable.  We don't replace $+$ sign with a number.  
Addition depends on some inputs, the total 
number being its \emph{valence}.    It also depends on a grammar.
It may be \emph{infix} $2+3$ as we commonly 
use with arithmetic.
% , more specifically, if we intend to add just one type 
% $A$ of data we want to add then its infix addition grammar is
% \begin{center}
%     \begin{minipage}{0.4\textwidth}
% \begin{Gcode}[]
% <A> ::= <A> + <A> 
% \end{Gcode}
% \end{minipage}
% \end{center}
Because this has only one type of data, e.g. everything here was an integer, 
 we call this operator \emph{homogeneous} operator.
% The term \emph{operad} is also used in some circles.
To express the pattern symbolically we can resort 
to schematic method like the following.
\[
    \Box+\Box
\]
Addition may also be \emph{prefix} $+,2,3$ as in the recipe instructions,
or it could be \emph{postfix} $2,3,+$; as in the census illustration. 
In diagrams:
\begin{center}
% \begin{minipage}{0.4\textwidth}
% \centering
% \begin{Gcode}[]
% <A> ::= + <A> <A> 
% \end{Gcode}
$+ \Box \Box$
\hspace{1in}
% \end{minipage}
% \hfill
% \begin{minipage}{0.4\textwidth}
% \centering
% \begin{Gcode}[]
% <A> ::= <A> <A> +
% \end{Gcode}
$\Box \Box +$
% \end{minipage}    
\end{center}    

The variations in grammar are like any language 
where there could be a dialect that evolves the operator's grammar and lexicon.
 HP calculators were postfix for some time to match engineering requirements.
A program to add two lists could get away with the following linguistic drift:
\begin{center}
\begin{notebookin}
cat [3,1,4] [1,5,9]
[3,1,4] + [1,5,9]
\end{notebookin}
\begin{notebookout}[2]
[3,1,4,1,5,9]
[4,6,13]
\end{notebookout}
\end{center}
Using \texttt{cat} reminded us to concatenate and avoided confusion with the
later $+$ concept.  It was the better choice. Challenge yourself to see both as
addition and you will find addition everywhere. 

Addition  may be with two or more terms.  It also may be 
sensitive to order: ``add the food coloring to the vinegar, then add the baking
powder'' is a different to ``add the vinegar and baking powder then the
food coloring''.  To make a manageable theory, algebra needs fewer variations that 
build towards this complexity.

Since we are evolving, we may as well permit multiplication as a bivalent operator
symbol, changing the signature to $\Box \cdot \Box$, i.e. $2\cdot 4$; or
$\Box\Box$, e.g. $xy$.   Avoid $\Box\times \Box$, we need that symbol elsewhere.
These days composition $\Box\circ\Box$ is written as multiplication; so, you can
use that symbol however you like.  

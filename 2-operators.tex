\chapter{How to operate}

One day  $+$ means to add natural numbers, the next day 
polynomials, later matrices.  
You can even add colors ``Yellow=Blue+Green''. When you program 
you learn to add strings
\begin{center}
\begin{notebookin}
print "Algebra " + " is " + " computation"
\end{notebookin}
\begin{notebookout}
Algebra is computation
\end{notebookout}
\end{center}
The $+$ is in fact a variable stand in for what we call a \emph{binary operator}
or \emph{bi-valent operator}.  It takes in a pair, 
according to the grammar $\Box+\Box$.  The valence (the number of inputs) and the grammar 
of an operator comprise its  \emph{signature}.  

Unlike these notes, addition should only be used when it is grammatically 
correct e.g.\ \emph{infix} $2+3$ rather than \emph{prefix} $+,2,3$ or
\emph{postfix} $2,3,+$.  Think of this like any other language 
where there could be a dialect that evolves the operator's grammar and lexicon.
Famously HP calculators were postfix for some time to match engineering requirements.
It turns out humans will adapt to technology easier than technology adapting to humans.
A program to add two lists could get away with the following linguistic drift:
\begin{center}
\begin{notebookin}
cat [3,1,4] [1,5,9]
[3,1,4] + [1,5,9]
\end{notebookin}
\begin{notebookout}[2]
[3,1,4,1,5,9]
[4,6,13]
\end{notebookout}
\end{center}
Using \texttt{cat} reminded us to concatenate and avoided confusion with the later $+$ 
concept.  It was the better choice.
Challenge yourself to see both as addition and you will 
find addition everywhere. 

Since we are evolving, we may as well permit multiplication as a binary operator
symbol, changing the signature to $\Box \cdot \Box$, i.e. $2\cdot 4$; or
$\Box\Box$, e.g. $xy$.   Avoid $\Box\times \Box$, we need that symbol elsewhere.
These days composition $\Box\circ\Box$ is written as multiplication; so, you can
use that symbol however you like.  Addition is held to high standards in algebra
(that it will evolve into linear algebra).  So when you are considering a binary
operation with few if any good properties, use a multiplication inspired
notation instead.   


Valence 1, \emph{unary}, operators include the negative sign $-\Box$ to create 
$-2$.  The transpose of a matrix is a unary operator.  Programming languages add several others 
such as \lstinline{++i, --i} which are said to \emph{increment} 
or \emph{decrement} the counter i (change it by $\pm 1$).


We can combine binary and unary operators to make useful (and useless) ternary 
operators.  One of my favorite (but generally useless) ternary products multiplies $(2\times 3)$-matrices,
that is right, rectangles not squares.  The product goes like this:
\begin{align*}
    [A,B,C] & = AB^{\dagger}C
\end{align*}
where $B^{\dagger}$ is the transpose.  A more serious product comes up 
in symmetric matrices where we need what is called the \emph{Jordan Triple product}
\begin{align*}
    \{A,B,C\} & = \frac{1}{6}(ABC+CBA)
\end{align*}
This is part of an whole family of Jordan products including 
\begin{align*}
    A\bullet B & = \frac{1}{2}(AB+BA)\\
    \langle A_1,\ldots,A_{\ell}\rangle & = \frac{1}{\ell!}(A_1\cdots A_{\ell}+A_{\ell}\cdots A_1).
\end{align*}
Notice in all these case if $A_i=A_i^{\dagger}$ then $\langle A_1,\ldots,A_{\ell}\rangle=
\langle A_1,\ldots,A_{\ell}\rangle^{\dagger}$.  So these are products that explain how 
symmetric matrices behave, and if you know about quantum mechanics that means these 
are the products you can use on observable quantum events.\footnote{Pascual Jordan was a physicist inventing 
math for quantum mechanics, not be be confused with Camile Jordan of Jordan-Holder and Jordan Normal form fame.}

Stranger ternary products showup in places where we wish we had easier binary products 
to explain things.  For example in geometry at small dimensions (dimension 2) we do not always 
have coordinates to explain what makes some set of points into a line. So Marshall Hall 
decided why not make a line $\ell$ be described by a ternary operator
\[
    -\otimes-\oplus -
\]
This is just notation but put in a formula like $y=m\otimes x\oplus b$ and you start to imagine 
the result as an algebraic line.  With this ternary product he was able to associated every 
projective plane to coordinates in some ternary ring.  Once you have such a ternary ring you 
can go to work to see if it might actually decompose into two binary operations of multiplicaton 
and addition, e.g.\ by locating a ``one'' and a ``zero'' where $x=1\otimes x\oplus 0$.  Then 
you reverse the process and define $m\cdot x\defeq m\otimes x\oplus 0$ and $x+b\defeq 1\otimes x\oplus b$
to get a more familiar ring-like structure.

Programs also exploit a ternary (valence 3) operator:
\begin{center}
    \lstinline[language=Sava]{if (...) then (...) else (...)}
\end{center}
The words, while helpful, are unimportant.  Some programming languages 
replace them with symbols emphasizing their ``opperatorness'' 
\begin{center}
    \lstinline[language=Sava]{_?_:_}
\end{center}
Here for example is division with remainder of positive integers
\begin{center}
\begin{lstlisting}[language=Sava,mathescape]
div(m,n)=(m>=n)?(div(m-n,n)+(1,0)):(0,m)
\end{lstlisting}
\end{center}
Programs go further making operators with huge valence.
If you find that interesting look into \emph{variadic} operators to see how 
far this idea goes.


A number of subtle problems are 
mounting.  For example, we probably want a Boolean (true/false) to 
go in the the left-most spot of an if-then-else-, and we cannot compose 
just any two functions and get expected results.  We scale a vector on one side 
by not the other.  Grammar must be more than just $\Box\cdot \Box$.


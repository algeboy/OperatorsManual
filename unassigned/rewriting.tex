% 
\section{Basic Grammar}
Admittedly  $\Box+\Box$, $\Box\cdot \Box$, and $-\Box$ indicate where to place 
information but they do not clarify what can be placed in each spot. 
We can add some clarity by clarifying the grammar with more meaningful tags.
For example, 
\begin{center}
    \code{<Matrix> ::= <Matrix> + <Matrix>}
\end{center}
would clarify that for this $+$ the intension was to add two matrices and 
the result will be another.   If we want to be certain that the dimensions 
match we can add this to the grammar.
\begin{quote}
    \code{<Matrix(2,3)> ::= <Matrix(2,3)> + <Matrix(2,3)>}\\
    \code{<Matrix(2,4)> ::= <Matrix(2,3)>  <Matrix(3,4)>}
\end{quote}
This approach becomes somewhat tedious as it depends so visibly on 
constants what will change between applications.  Later we revisit 
this problem with a few better options.


There are two implicit assumptions in what 
we have written.  First, while this definition is recursive we only intend 
that we should place a $+$ between two matrices that already exist.  In this 
way the grammar looks only back in time eventually settling on some constant
matrices.  Otherwise we could end up with some sort of infinite loop that 
never draws to a close.  Recursion that looks back in time to a start point is 
called \emph{primitive recursion}.  The second unexplained assumption is what 
qualifies as a constant matrix, a base case, to start the process off. 
The zero matrix for example would be one option, as would the matrices 
$E_{ij}$ that have $0$ in all position except row $i$ and column $j$ where 
the number is $1$.  If we include rescaling as an option then through 
linear combinations we could specify any matrix in this way.

Later we shall be more formal with grammars but we close we a few more 
demonstrations.
\begin{center}
\begin{Gcode}
<List> ::= cat <List> <List>
<List> ::= <List> + <List>
<A or B> ::= if <Boolean> then <A> else <B>
\end{Gcode}
\end{center}
When we wish to indicate that symbols $x$ have met the requirement to be 
treated as a type say ``matrix'', or ``list'', or ``Boolean'' we 
write $x:Matrix$, $x:List$, $x:Boolean$ accordingly.  We are also 
lenient with the use of popular shorthand such as $\mathbb{N}$ for natural 
numbers.  So $n:\mathbb{N}$ indicates that $n$ is a natural number.
Here are some related demonstrations.
\begin{quote}
    \code{(cat [1,2,3] [4,5,6]):List}.\\
    \code{([1,2,3] + [4,5,6]):\text{List}}.\\
    $\displaystyle 
        \begin{bmatrix} 1 & 0 & 8 \\ 2 & 7 & -1\end{bmatrix}
    + \begin{bmatrix} -1 & 0 \\ 0 & 1 \end{bmatrix}:\mathbb{R}^{2\times 3}$.
\end{quote}


\section{Rewriting}
In summary we introduced what passes as function evaluation 
in our usual $f(x)\defeq \ldots$ form and saw that it was in fact 
a mirage.  In retrospect, substitution leaves the overall structure 
intact whereas functions are instead tasked with transforming data.
So the idea was doomed from the start.  Even so the conventions now 
are so common as to require correctness to yield to history.
Let me therefore declare that whenever in this text or generally 
elsewhere we see or engage the use a function
\[
    f(x)\defeq M
\]
or even in code as 
\begin{Pcode}[]
def f(x) = M
\end{Pcode}
it is understood to mean $f:x\mapsto M$ and furthermore:
\[
    f(c) \defeq M[x\leftrightarrows c].
\]

\subsection{Abstract Rewriting Systems}
To truly rewrite data we need 
conversions and reductions, or what collectively we call \emph{rewriting}.
\index{rewriting}

Rewriting 
occurs everywhere from the evaluation of operators like we explored her, 
to multiplying in groups, to calculating algebraic geometry properties through 
Gr\"obner-Shirshov bases.  The strategies of rewriting evolve with every context 
but most have an arc similar to the one we pursue in this chapter.  Start by 
considering the order in which you rewrite.  Make it a partial order or at 
least a directed graph.  Search for confluence: places where branches meet back
together.  When enough confluence exists look for a reason for finite rewriting 
to reach a normal form---a unique lowest point, subject to the order we introduce.
Finally, when normal form exist the work begins to get that normal form efficiently.
Above all notice how many possible points of failure there are in effective 
rewriting.  

\subsection{Normal Forms in general}\index{normal form}

\subsection{Word problems}\index{word problem}

\subsection{General situation}

As a general intuition treat rewriting as undecidable, but if decidable 
then it will be exponentially hard, but if it can be rewritten efficiently 
then it was never rewriting---it was linear algebra disguise.
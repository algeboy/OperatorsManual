
\section{Higher order Operators}
Lets consider an operator 

Another operator takes a pair $A$ and $B$ of sets or more general types 
of data and creates ordered pairs or disjoint unions.
\begin{gather}
    \tag{$\form{\times}$}
    \frac{A,B:Type}{A\times B:Type}\\
    \tag{$\intro{\times}$}
    \frac{a:A\quad b:B}{(a,b):A\times B}\\
    \tag{$\elim{\times}$}
    \frac{x:A\times B}{\pi_A(x):A}\qquad 
    \frac{x:A\times B}{\pi_B(x):B}\\
    \tag{$\comp{\times}$}
    \frac{a:A\quad b:B}{\pi_A(a,b)\defeq a}
    \qquad
    \frac{a:A\quad b:B}{\pi_B(a,b)\defeq b}
\end{gather}
In code you may encounter this as \lstinline{Pair[A,B]} or 
as \lstinline{(A,B)}.

\begin{lstlisting}[language=Hidris]
Either A B = Left (a:A) | Right (b:B)

apply (f:A->C, g:B->C, x:Either[A,B]) : C =
  match x with 
    Left a => f(a)
    Right b => g(b)
\end{lstlisting}
A special case of the disjoint union operator is to extend a type by 
one term:
\begin{align*}
    A^? \defeq A\sqcup \{*\}.
\end{align*}
It is said that $x:A^?$ is ``maybe an A'', or ``optionally A''.
In code you will find these written out this way.
\begin{lstlisting}[language=Hidris]
Maybe A = Just (a:A) | None
\end{lstlisting}
\begin{lstlisting}[language=Sava]
class Option[A]
    case Some(a:A) extends Option[A]
    case None extends Options[All]    
sealed  // You cannot add to these cases
\end{lstlisting}
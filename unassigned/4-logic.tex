\section{Logical operators}

We can summarize this accomplishment in two logical rules, but first allow 
me to digress to explain three crucial differences in the logic of going from 
one list of facts to another.

\subsection{Meta language}

In fact, the notation \code{<Name> ::= ...} is evolving a grammar of its own
in a format known as \emph{Backus-Naur Form (BNF)}.  In different notation the 
idea was already present in Panini's explanations of the sanskrit language.


\subsection{A closer look}
Perhaps the most convincing way to see that we are creating 
data is to run this process all the way down to what a machine 
must do and witness that somewhere the computer must somewhere 
as for storage space.  We will do this 
in stages as the bottom is a mighty long fall.

A first step might be to convert the cases into a sequence 
of two functions which both make appropriate changes to 
some internal value.  This might look like this:
\begin{center}
\begin{lstlisting}[language=Java]
class NaturalNumber {
    int value // Internal computer storage
    Zero() {value=0} // zero out the memory
    Next(k:NaturalNumber){value=k+1}
}
z = new Zero()  // make a 0
two = new Next(new Next(z)) // make a 2
\end{lstlisting}
\end{center}

Grouping this under a single data type `NaturalNumber'
is mostly just for us not to get confused.  When we 
zoom in closer we find that the computer splits 
this apart, making the storage space separate from the 
functions that use it.  Here is a lower level take
and a realization that indeed both introductions (constructors)
need to ask for some storage, they do not really know 
how much, the computer calculates that for them.
\begin{center}
\begin{lstlisting}[language=C]
struct Nat { int length, uint8[] digits }
Nat NaturalNumber_Zero() { 
    memory = malloc(Nat) // allocate memory
    memory.length = 1
    memory.value[0] = 0
    return memory 
}
Nat NaturalNumber_Next(k:Nat) {
    memory = malloc(Nat) // allocate memory
    if k.digits[k.length-1] != 0 then
        memory.length = length +1
    
    memory.digits = k.digits +1
    return memory 
}
\end{lstlisting}
\end{center}

If we zoom in even further we see this `malloc' command eventually 
translate into asking for actual memory in the computer.  Here is a
simplified take that assumes the length is fixed at 9 bytes, that is 
the maximum value reached is $2^{128}$, it would take all the computers 
in the world to tally from $0$ to $2^{128}$ in your lifetime so 
this is a safe limit.
\begin{center}
\begin{lstlisting}[language={ [x86masm]Assembler}]
%macro NaturalNumber_Zero 0
    mov param1, 0x0009      ; move 4 into first parameter
    call malloc            ; storage of size param1 (=9)
    mov store, 0x0000      ; mov 0 into the storage
    ret 

%macro NaturalNumber_Next 1
    int  temp
    mov  temp, param1     ; make a copy of first input
    mov param1, 0x0009      
    call malloc          ; storage of size 9
    mov  store, temp     ; move input to storage
    inc  store           ; increment storage 
    ret 
\end{lstlisting}
\end{center}

As this snippet reveals this program comes to a physical limit 
once we exceed the internal storage capabilities of integers in the
system.  We can fix this but the spirit of the programs above 
is to separate the need to understand the computer deeply and see the 
overall behavior.

\subsection{Operators that generate}
Incrementing is more a definition than it is a 
 case of adding 1.
A natural number is either $0$, or a successor 
$S(k)$ to 
another natural number $k$.  Often this is expressed as a 
definition where $\mid$ stands for separating cases, 
for example:
\begin{align*}
    \mathbb{N} \defeq 0 \mid S(k)
\end{align*}
So $0$ is ``zero'', and $1$ is just a symbol representing $S(0)$, 
$2\defeq S(S(0))$ and so on.  Replace $S$'s with tally marks (not to be 
confused with `$|$' used as an ``or'' above)
we recover childhood counting:
\begin{center}
    $0\defeq$ \underline{\hspace{5mm}}, 
    $1\defeq$ \StrokeOne,
    $2\defeq$ \StrokeTwo,
    $3\defeq$ \StrokeThree,
    $4\defeq$ \StrokeFour,
    $5\defeq$ \StrokeFive,...
\end{center}
The point is, the successors are not so much a function 
moving around the numbers we have, it actually is a producer 
of numbers. 

Perhaps because it is so primitive, this is an idea we 
can imitate to create more meaningful values, like a string 
of characters in an alphabet \lstinline{Char:=['a','b',...,'z']}.
\begin{lstlisting}[language=Hidris]
data String = Empty | Prepend( head:Char, tail:String) 
\end{lstlisting}
Writing \lstinline{head:Char} or \lstinline{tail:String} 
indicates that head must come from the alphabet we chose 
and tail must be some already produced string, possibly empty.
Some readers might relate to a different dialect of 
programming such as the following
\begin{lstlisting}[language=Sava]
class String
    case Empty extends String
    case Prepend( head:Char, tail:String) extends String
\end{lstlisting}
The head here caries around what we put in the list and the tail 
is what comes next in the list.  Observe the similarities:
\begin{align}
     2 & \defeq S(S(0)) \tag{$\mathbb{N}$}\\
 \text{\lstinline{"me"}} & \defeq \text{\lstinline{Prepend('m',Prepend('e',Empty))}}
\tag{String}
\end{align}
The left-hand sides are merely notation for what the data really is on the right.
Both the successor and the \lstinline{Prepend} are operators that generate 
new values.  So part of algebra is to generate new data; so, it is no wonder 
that it closely connections to computation.

\subsection*{Exercise}
\begin{enumerate}
    \item Mimic the String data type to make a list of integers (that is 
    switch from the alphabet to integers).

    \index{generics}
    \item Mimic the String data type to make a list of fixed by 
    unknown data of type $A$, call it \lstinline{List[A]}.\footnote{
    Alternatives include \lstinline{List a} and \lstinline{List<A>}. 
    Search for \emph{generics} in your programming language to learn more.
    }

\end{enumerate}
\index{nil}\index{cons}\index{list}
Historically \lstinline{Empty} for lists is called \lstinline{Nil} 
and \lstinline{Prepend} is called \lstinline{Cons}.

\subsection{Generated operators}
We can take this the idea of generating further, for example, using 
the unary operator, successor, prepend, etc., and have them generate 
binary operators.
\begin{align*}
    m+n & \defeq\left\{ 
    \begin{array}{ll}
        n & m = 0\\
        S(n+k) & m=S(k)
    \end{array}
    \right.
\end{align*}
So does $2+4=6$?  We can test this out.
\begin{align*}
    \text{\StrokeTwo}+\text{\StrokeFour} & = \text{\StrokeOne}~ \big(\text{\StrokeOne} +\text{\StrokeFour}\big)\\
    & = \text{\StrokeOne}~ \big( \text{\StrokeOne}~\big(\underline{\hspace{5mm}}+\StrokeFour\big)\big)\\
    & = \text{\StrokeOne}~ \big( \text{\StrokeOne}~\StrokeFour\big)\\
    & = \text{\StrokeOne}~ \text{\StrokeFive}.
\end{align*}
Try this for strings
\begin{align*}
    s+t & \defeq\left\{ 
    \begin{array}{ll}
        s & t = \texttt{Empty}\\
        \texttt{Prepend}(x,s+tail) & t=\texttt{Prepend}(x,tail)
    \end{array}
    \right.
\end{align*}
What is \lstinline{"awe"+"some"}?  

While no one will seriously add integers as a tally, 
knowing that it can be done establishes a pattern which can be exported to other context
with meaningful new structure.  Just notice $3+4=4+3$ but not so with adding strings.
These siblings have their own personalities, and it this might even help us recognize 
that while $3+4$ does equal $4+3$ it might be for somewhat subtle reasons.

What about multiplication? Isn't is just this:
\begin{align*}
    m\cdot n\defeq \overbrace{n+\cdots +n}^m.
\end{align*}
That is nice, but seems to leave us to figure out missing parenthesis or set aside 
time to prove they don't matter.  I am in the mood to continue making things rather 
than study them.  Lets repeat what we have done.
\begin{align*}
    m\cdot n & \defeq \left\{
        \begin{array}{ll}
            0 & m=0\\
            k\cdot n+n & m=S(k).
        \end{array}
    \right.
\end{align*}
It works, but check it out for yourself.  And for strings what might we get?
\begin{align*}
    s\cdot t & \defeq \left\{
        \begin{array}{ll}
            0 & s=\texttt{Empty}\\
            \texttt{Prepend}(x,k\cdot n)+n & s=\texttt{Prepend}(x,tail)
        \end{array}
    \right.
\end{align*}
What is "Mua"$\cdot$"Ha"?
%% Mua*Ha=MuaHaHaHa
You can carry on to make exponents and more.  If you do you may find 
Knuth arrow notation helpful on your journey, look it up.



\section{Free algebra}
There is an evolving pattern in how we described a grammar first for strings, 
then Boolean algebras, and now polynomials.  They each rely on some 
constants set aside at the start.  They also allow for a separate sort for
variables.  Add a final sort for operators and we can describe all the formulas 
we expect in such an algebra.

Now let us notice the pattern.  Our Boolean algebra and our polynomial algebra each 
came in certain form.
\begin{Gcode}[]
    <Formulas<Var>> ::= <Var>
            | <Constants>
            | <Operators>
\end{Gcode}
In fact we can separate operators into order based on how many terms they depend on,
the \emph{valence}, equivalently the degree of the eventual parse tree should that 
production be found in a string.  Letting constants qualify as null-valent operators 
then we have the structure:
\begin{Gcode}[]
    <Formulas<Var>> ::= <Var>
            | <nullvalent operators>
            | <univalent operators>
            | <bivalent operators>
            ...
\end{Gcode}
Or we may abstract in the coarser direction and observe simply the structure operators:
\begin{Gcode}[]
    <Formulas<Var>> ::= <Var>
                      | <Operators>
\end{Gcode}

\begin{definition}
    A \emph{signature} $\sigma$ of an algebra is an (unambiguous) context-free grammar 
    whose production rules $P_1,\ldots,P_n$ are augmented by variables $X$:
    \begin{Gcode}[]
        <$F_{\sigma}$<X>> ::= <X> | $P_1$ | $\ldots$ | $P_n$
    \end{Gcode}
    The \emph{free algebra} on $X$ is the language accepted by this grammar.
\end{definition}

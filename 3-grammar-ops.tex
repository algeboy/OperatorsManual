

% Each production is an allowed step in an induction.  If it branches then two or more 
% previously built items must be combined.  The atoms are what we call base-cases 
% in the induction parlance.  The role of grammar on induction becomes all the more 
% pronounced when we begin to give them distinguished roles.


\section{Multi-sorted grammars.} \index{sort}
To add depth to language we sort the allowed productions.  In natural languages 
this may be verbs and nouns.  In mathematics the typical  first sort are constants,
\lstinline{<Nat>} for instance.  The next sort are variables, e.g.\
$x,y,\ldots$. These may depend on constants, such as numbering variables
$x_1, x_2,\ldots$.
\begin{center}
    \lstinline{<Var>  ::= x | y | x_<Nat> | y_<Nat>}
\end{center}
A third sort might be operators, e.g.\ $+,-,\times,\div$.
Symbols like parenthesis, braces and other groupings are usually considered as part of 
the meta-language and used to clarify how to read formulae, but not specifically part of any formula.
Like ``set'', ``sort'' now has this technical meaning: one of these designations in that language.
So a sort consists of types.

\index{boolean algebra|(}
% To make things short lets drop down to the language of Boolean (true/false) algebra.
\medskip

\noindent\textbf{Example.}
For Boolean (true/false) algebra we have true `$\top$', false `$\bot$', 
and `$\wedge$', or `$\vee$', and not `$\neg$' with the following grammar
which includes parenthesis. See Listing~\ref{lst:bool}.
For example, $\top \vee (\neg(x_2\wedge y)\vee x_1):Bool$ but $Bool$ rejects $\bot\neg$.
\begin{lstfloat}[!hbtp]
\begin{lstlisting}[mathescape]
    <Var>  ::= x | y | x_<Nat> | y_<Nat>
    <Bool> ::= $\top$ 
             | $\bot$ 
             | <Var>
             | $\neg$ <Bool> 
             | <Bool> $\vee$ <Bool> 
             | <Bool> $\wedge$ <Bool>
             | (<Bool>)
\end{lstlisting}
\caption{A Boolean algebra grammar.}\label{lst:bool}
\end{lstfloat}

\index{first order logic}
What we have built is very nearly the \emph{free boolean algebra of countable rank}, or 
sometimes known as 1st order logic. The missing ingredient (relations) will come later.
What this demonstrates is how close to language algebra really is.  You start with some desired 
operators, some constants, and some variables and you build it formulas by induction.
If you insist on repeating the use of common symbols like $+,\times,-,\div$ and etc. 
you will end up with formulas that look like polynomials from high school.  It is really one language with 
different dialects.
\index{boolean algebra|)}

\subsection{Order of operations.}

There is something still missing.  Try reading $\neg(x_2\wedge y)\vee x_1$.  
Is this meant to negate $(x_2\wedge y)\vee x_1$ 
or is it to negate $(x_2\wedge y)$ and then or that with $x_1$? This grammar 
offers two associated parse trees.
\begin{center}
    \begin{tikzpicture}
        \node at (-3.5,0) {\begin{tikzpicture}
            \node (10) at (0,0) {$\neg(x_2\wedge y)\vee x_1:Bool$};
            \node (0n) at ( 0,-2) {$(x_2\wedge y)\vee x_1:Bool$};
            % \node (0d) at ( 2,-4) {0:Digit};
            \draw[thick] (0n) -- (10);
            % \node[below of=10,scale=0.75] {$\circ$};
            \node at (0.5,-1) {$\neg$};
            % \node[scale=0.75,text width=0.6in] at ( 1.5,-3) {Nat case 1};
        \end{tikzpicture}};
        \node at (3.5,0) {\begin{tikzpicture}
            \node (01) at (0,0) {$\neg(x_2\wedge y)\vee x_1:Bool$};
            \node (1) at (-2,-2) {$\neg(x_2\wedge y):Bool$};
            \node (0n) at ( 2,-2) {$x_1:Bool$};
            \draw[thick] (0n) -- (10) -- (1);
            \node[below of=01] {$\vee$};
        \end{tikzpicture}};
    \end{tikzpicture}
\end{center}

\emph{Order of operations} steps in to resolve the ambiguity.
One option is \emph{left-most outer-most (LeMOM)}, which---similar to English language,
reads from left to right and assumes we start to match the pattern from the 
outer most symbols.  So  $\neg(x_2\wedge y)\vee x_1$ in LeMOM means
$\neg$ is the symbol we need to parse first.  That requires us to 
build a \lstinline{<Bool>}.  So we move left and read 
$(x_2\wedge y)\vee x_1$.  Here the LeMOM is `(' which will need to match 
with \lstinline{<Bool>)}.  Reading further $x_2:Var$ which promotes to 
$x_2:Bool$, but this is not followed by `)' so it must continue left-ward.
Next is $\wedge$ which matches with $x_2:Bool\wedge y:Bool$
so that we have $x_2\wedge y:Bool$.  Then we hit `)' which matches 
our earlier `('.  Now we finally get $(x_2\vee y):Bool$ which finally matches with 
$\neg$\lstinline{<Bool>}.  We have parsed $\neg(x_2\wedge y):Bool$.  At this point 
we have parsed $\neg$ and so we continue with $\vee$ to match 
$\neg(x_2\wedge y)\vee x_1:Bool$ with the right-hand-side parse tree.


LeMOM parsing is favored in computer science because we can 
decide what things mean the moment we encounter them in the string.
Meanwhile high school PEMDAS can require us to scan back and forth 
to know what to do, for example $7\cdot (7+3)^2$ could start by taking 
apart the product but then had to scan to the end to identify the exponent,
then back-track to insert the addition.  
There are other orders of operations include right-most inner-most (RiMIM).
The point of different orders of operations is to give the write the ability 
to write expressive language, jargon which means that more meaning is packed 
into the same space of letters.

\begin{quote}
\textbf{A properly sorted grammar makes expressive language 
that still parses uniquely.}
\end{quote}

\noindent\textbf{Exercises}
\begin{enumerate}
    \item Write the grammar for decimal numbers.
    \item Write the grammar for a 4 function calculator with decimal numbers
    and $+,-,\times, \div$.
\end{enumerate}
\medskip

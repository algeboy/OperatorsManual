\section{Inventing solutions}
Now since every symbol in an equation is a variable we have new powers 
to solve equations.  Look to the humble 
\[
    x^2+1=0.
\]
By our own view, right now this is nothing but variables, so it means nothing to
solve this.  But, drop this into a context such as decimal numbers $\mathbb{R}$
and the understanding is to replace $1\defeq 1.0$, $0\defeq 0.0$, $+$ is
substituted for addition of decimals, and square is by multiplying decimals.
Equality now means two equal decimals, or in practice two decimal numbers that
are close enough to be considered equal.  The only remaining unknown is $x$, but
as everyone knows, $0\leq x$ or $x<0$ so in both cases $-1<0\leq x^2$.

The power of variable everything is that we are not stuck with the real numbers.
Let us replace everything with complex numbers $\mathbb{C}$. Substitute $0\defeq
0+0i$, $1=1+0i$, $(a+bi)+(c+di)\defeq (a+b)+(c+d)i$, and
$(a+bi)^2=(a^2-b^2)+2abi$.  Now we find $\pm i$ are the solutions. Solutions do
exist, they are imaginary, but every number is in our imagination. 
That solutions exists allows us to return to the story problem and ask:
maybe these new numbers hint at a process we are not yet noticing in our 
problem.

Why stop here? Quaternions have $\hat{\i}^2=\hat{\j}^2=\hat{k}^2=-1$;
so, at least 6 solutions to $x^2+1=0$.  Even more.  Try $(2\times 2)$-matrices, I
bet you can find infinitely many matrices $M$ where $M^2=-I_2$.  This is the method
of algebra: dream up new numbers that might be used to solve equations.  Alter
their properties, e.g.\ drop the order of real numbers and you can get complex
solutions.  Drop commutative multiplication of complex numbers and you might get
infinite solutions.  
% Learn enough by this process and we can begin to predict if
% solutions are to be expected and fathom algorithms to find them when they do
% exist. When the solutions become infinite we find ways to parameterize them with
% smaller data such as a basis.

Don't forget equality is variable too!  Suppose we wanted to solve $x^{541}+x+1=0$
using only integers.  Replace equality 
with $\equiv$ modulo 2 and ask for a whole number $x$ that solves
\begin{align*}
    x^{541}+x+1\equiv 0\pmod{2}.
\end{align*}
All integers in this equality become either $0$ or $1$, but neither will solve 
this equation.  By varying equality we confirm there are no solutions.

This is why so much of algebra today is concerned with making new numbers, the
operators, and searching out which equalities (congruences) can be used on new
numbers.  It is the front line of algebra, the first question that must be
concurred.  And it is  part of why so much research on algebra doesn't even
feature solving equations any more.  The goal is to have numbers ready for
whatever equations show up.

\begin{quote}
    \textbf{The first method of algebra is to invent numbers: their constants,
    their operations, \& their equations.}%  We call these ``algebras''.}
\end{quote}

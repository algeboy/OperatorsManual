
\section{Relating solutions}
Sometimes we cannot find a number system to solve an equation
in a meaningful way.  Sometimes we know solutions exist but are stymied 
when it comes to finding efficient algorithms to get them.  For centuries 
no one knew if we could trisect an angle by ruler and compass and no one 
could find a general formula for factoring 4th degree polynomials.  And other 
times we succeed far too well creating far too many solutions, 
maybe infinitely many, and we simply cannot expend the time to use them all.
This is when algebra turns its interest to relating solutions to one another.

The simplest example is solving systems of linear equations.  Often 
this is in an infinite solution set.  We dampen that by describing all 
solution with a basis, and prove that any two bases are interchangeable.
A harder example are the roots of a polynomial.  Surely we have come 
to expect that roots of quadratics like $x^2-2$ will be related, 
as $x=\pm \sqrt{2}$, and more generally $x^2+x+1$ gives 
$x=z,\bar{z}$ where $z=\frac{1}{2}+\frac{\sqrt{3}i}{2}$.  That process 
generalizes and gives away an inductive process to factor many polynomials.
Lagrange used that to finally  find a general formula for quartic polynomials.
Gauss used that idea to prove that the constructable numbers (numbers we can 
produce by ruler and compass) are limited to iterations with square roots.
That explain why so many regular polygons need more than a ruler and compass 
to be drawn.  Of course we can trisect angles with 
other tools it just means ruler and compass is not enough.

\index{Galois}
The mindset of investigating relationships between solutions even leads to
identifying obstacles to having solutions.  Abel found $x^5-x-1$ could not be
solved by radicals alone.  That is, a calculator that can correctly solve
quintic or higher degree polynomials will need buttons beyond the square-root,
cube-root and n-th root buttons.  Galois went further and described exactly what
polynomials need new methods to make roots.  In a sense, Abel and Galois
explained we need to go back and invent more numbers.

\begin{quote}
    \textbf{The third method of algebra is to relate solutions to one another.}
\end{quote}

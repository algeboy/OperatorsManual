\chapter{What is algebra?}

Algebra is the study of equations, for the most part equations involving variables.
That is because applications have unknowns and if 
the shape of the equation can tell us anything about the 
options to solve it, we shall want to take advantage of this.



You probably already know every color, shape, and pattern of 
equation you will ever need.  Surprised? I bet you 
can see a relationship between the following two equations.
\begin{align*}
    m^2-n^2 & \equiv 0 \pmod{1649} 
    & 
    \frac{\partial^2 f}{\partial x^2}-\frac{\partial^2 f}{\partial y^2} & =0.
\end{align*}
These equations concern entirely different topics.  The left is a problem 
used to factor integers.  The right gives us the behavior of waves.  
The left uses integers.  On the right $0$ is a function on the $xy$-plane.
And yet,  similarities shine through and we can accept both equations 
have something to do with this one:
\[
    x^2-y^2=0.
\]
Why does this work? It is because $0$, $+,-$, and squaring are general concepts,
``abstractions''.  

Abstract does not connect our question to polarizing art movements nor render
the concept inapplicable.  Abstract means to study by limited attributes.
That's how we do all math and science.  So when we abstract the equation on the
left we forget about mod 1649 and the precise meaning of these numbers.  On the
right we forget about functions and the notions of derivatives.  We are left
with just $0$, $-$, and squares and where they sit.  We abstract both to a
common equation. In fact, even the equality was an abstraction which could vary
from context to context. With such flexibility a small number of symbols and
their grammar are enough to capture the huge variety of equations we encounter
in real life.

\begin{quote}
    \textbf{The power of algebra is that every symbol 
    in an equation is a variable, especially the equals sign.}
\end{quote}



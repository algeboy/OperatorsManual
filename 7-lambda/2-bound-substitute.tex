\section{Bound substitution}\index{bound}
With everything said so far we seem no closer to unravelling the paradox 
of the trapped variable.  Following our rules for free substitution and 
Walrus naming, $\CombK{c}(x)\defeq c$ so 
\begin{align*}
    \tag{Nominal equality}
    (\CombK{c}(x))[c\leftrightarrows x] 
    &\quad =\quad c[c\leftrightarrows x]\\
    \tag{Free.match}
    & \quad =\quad x.
\end{align*}
The problem however was not in the symbols, but in what we interpreted them 
to mean.  Recall we tempted ourselves to treat both $I$ and $K$ as operators
and we knew identity operators are not constant operators.

To put a finger on the problem, when we use the Walrus we are giving 
a string a name.  So $f(x)\defeq M$ gives $M$ the name $f(x)$.  
We could just as well have called $M$ by the name $x(f)$ or $xfx$ or 
even ignored $x$ and written $James\defeq M$.  This lead to the substitution 
rules being applied based on nominal reasoning.  Yet, we all know that 
the role of placing $x$ in parentheses is more than naming our operator.
It is meant to identify the correct symbol to play the role of variable.


\subsection{Operator Abstraction}
\index{operator!abstraction}\index{operator!naming}
We need to narrow how we use strings so that substitution behaves 
like in should in operators.  (Recall ``abstraction'' means to restrict reason by rules.)
We signal this abstraction by by writing:
\begin{equation}
    \tag{Operator Abstraction}
    x  \mapsto M
\end{equation}
(Historically this is called function abstraction but we are 
attempting to reserve the word ``function'' for the set-based meaning)
To be clear $x$ here must be a variable, that is, from the variable 
alphabet.  $M$ can be any string even one not containing $x$.
Notice we did not  name the operator `$f$' or something related, 
this is in a real sense an \emph{anonymous operator}.  To 
name a operator we can use $f\defeq (x\mapsto M)$ 
but we almost always abbreviate this to $f:x\mapsto M$.

Our examples so far include:
\begin{align*}
    \CombI:x & \mapsto x, &
    \CombK{c}:x & \mapsto c, &
    \CombK{}: c & \mapsto (x\mapsto c).
\end{align*}



The variable set apart as the input is said to
be \emph{bound} to the operator, it now has a specific role to play similar to
how $\forall x.P$  and $\exists x.Q$ bind $x$ into a special roles in propositions.
Variables that do not have special roles are called \emph{free}.  
Historically Church used the notation $\lambda x.M$ to match those 
patterns and for this reason such functions today are called \emph{Churches}, 
only joking, they are called ``lambdas''. 

Note that the special role of $x$ in $x\mapsto M$ is limited to $M$.  We can say
$M$ is the \emph{scope}.   For this reason programs typically refer to bound
variables as \emph{local}.  Now let us extend substitution into lambdas.
\begin{definition}
    Assume an alphabet of variables with the property that it can always increase in number
    (e.g.\ $x_1, x_2,\ldots$).
Given strings $M,N$ and variables $x,y,\ldots$,
\begin{description}
    \item[Bound.match] $(x\mapsto M)[x\leftrightarrows N]\quad \defeq\quad (x\mapsto M)$
    \item[Bound.trapped]
    if $x$ is a free variable in $M$ and $y$ is a free variable in $N$ then 
    \[ 
        (y\mapsto M)[x\leftrightarrows N]
        \quad \defeq\quad 
        \bigl(z\mapsto (M[y\leftrightarrows z][x\leftrightarrows N])\bigr)
    \]
    where 
    $z$ is a variable distinct from any in $M$ and $N$ (which exists 
    as there are unbounded numbers of variables);

    \item[Bound.other] $(y\mapsto M)[x\leftrightarrows N]\quad \defeq\quad 
    (y\mapsto M[x\leftrightarrows N])$.
\end{description}
\end{definition}

\subsection{Operator Application}
The purpose of binding a variable is to control substitution but it may seem 
with rules like $(x\mapsto M)[x\leftrightarrows N]\defeq (x\mapsto M)$ lead 
us to never actually evaluate an operator.  To the contrary we are simply forcing 
ourselves to explicitly declare evaluation as a step, a computation, which 
very well may require work and alter our data.  This is done with a \emph{reduction}
denoted by $\leadsto$.
\begin{description}
    \item[$\alpha$-reduction] given variables $x$ and $y$, with $y$ not free in $N$,
    \[(x\mapsto M)\quad \leadsto_{\alpha}\quad (y\mapsto M[x\leftrightarrows y]),\]
    which simply renames the variable $x$.

    \item[$\beta$-reduction]
    $(x\mapsto M)N \quad \leadsto_{\beta} \quad M[x\leftrightarrows N]$, which replaces all (free)
    instances of $x$ in $M$ with $N$.  This is the main tool, it is \emph{evaluation},
    also called \emph{application}.
    
    \item[$\eta$-reduction]
    $(x\mapsto M)\quad \leadsto_{\eta} \quad M$, which unbinds the variable $x$.
\end{description}
For a combination of any three of these reductions we write $\leadsto$ unadorned.

Reverting temporarily to the notation $f(x)\defeq \ldots$, the $\alpha$-reduction rule
is what makes the following to operators ``the same''.
\begin{align*}
    f(x) &\defeq  \frac{\sqrt{e^x-1}}{x^2-x-6}
    & 
    f(t) &\defeq  \frac{\sqrt{e^t-1}}{t^2-t-6}
\end{align*}
More precisely
\begin{align*}
    f&:x\mapsto \frac{\sqrt{e^x-1}}{x^2-x-6}\\
    f & \leadsto_{\alpha} \biggl(t\mapsto \biggl(\frac{\sqrt{e^x-1}}{x^2-x-6}\biggr)[x\leftrightarrows t]\biggr)\\
      &  = \biggl(t\mapsto\frac{\sqrt{e^t-1}}{t^2-t-6}\biggr).
\end{align*}
While we might express this as ``replacing $x$ by $t$'', our reduction rules
now make it clear that we are also updating the role of the operator abstraction.


Likewise to evaluate $f(x)\defeq \ldots$ at $4$, commonly written as $f(4)$ will 
now strictly speaking mean
\begin{align*}
    f(4) & = \biggl(x\mapsto \frac{\sqrt{e^x-1}}{x^2-x-6}\biggr)4 \\
     & \leadsto_{\beta} \biggl(\frac{\sqrt{e^x-1}}{x^2-x-6}\biggr)[x\leftrightarrows 4]\\
    & =\frac{\sqrt{e^4-1}}{4^2-4-6}.
\end{align*}
Key to this reduction is that we end up removing the operator abstraction, i.e.
the relevant ``$\mapsto$'' has been eliminated in our process.  It is no longer
the same string with symbols swapped.  We have applied the operator and so it 
is no longer a operator of this $x$.

As with $\alpha$-reductions, the use of $\eta$-reductions is mostly cosmetic 
but is useful when shortening computations.  We could recover both from
$\beta$-reductions as 
\begin{align*}
    t\mapsto ((x\mapsto M)t) & \leadsto_{\beta} (t\mapsto M[x\leftrightarrows t]).\\
    (x\mapsto M)x & \leadsto_{\beta} M[x\leftrightarrows x]=M.
\end{align*}


These details
gets us out of the trapped variable paradox.
\begin{align*}
    \CombK{} &: c  \mapsto (x\mapsto c).\\
    \tag{Nominal equality}
    \CombK x & = (c\mapsto (x\mapsto c))x\\
    \tag{Application}
        & \leadsto_{\beta} (x\mapsto c)[c\leftrightarrows x]\\
    \tag{Bound.trapped}
        & = z\mapsto (c[x\leftrightarrows z][c\leftrightarrows x])\\
    \tag{Free.other}
        & = z\mapsto (c[c\leftrightarrows x])\\
    \tag{Free.match}
        & = z\mapsto x
\end{align*}
So in the end the constant $c$ was replaced by $x$ but at the same time 
the variable that was $x$ is now $z$ so $\CombK{x}:z\mapsto x$ remains constant.

\subsection{Applying reductions}

It is the job of a programming language to implement a version 
of substitution that follows these rules.  Once done the notation will take 
on the usual character of the programming language but often the notation 
comes close to the mathematical notations.  Here are some popular variations to try.
\begin{center}
    \code{lambda x.x+2}
    \hspace{1cm}
    \code{x => x+2}
    \hspace{1cm}
    \code{x |-> x+2}\\
    \code{func(x)=x+2}
    \hspace{1cm}
    \code{(x)-> {return x+2}}
\end{center} 


So our traditional $f(x)\defeq M$ notation would no be hinting at 
$f:x\mapsto M$ in our notation, and the $f$ here would be naming 
the specific example $x\mapsto M$, which is often helpful.  Yet 
we should not take this correspondence too far since we have already 
seen the ways in which substituting for $x$ in $f(x)\defeq M$ notation 
goes astray.

% The first rule Bound.match may at first seem odd.  Aren't we trying to place $x$?
% Yes but when we write $x\mapsto M$ we are declaring $x$ as a local variable.  
% It is completely meaningless what it is called outside the scope of $M$.
% It is the same thing we come to expect when we do things like this:


% In some situations the role of bound/local variables is further 
% restricted to roles of a decidedly special meaning, for example, as indices that 
% run through a range.
% \begin{align*}
%     \sum_{i=1}^{10} i^2 \qquad \prod_{i\in I}X_i 
% \end{align*}
% or in code 
% \begin{center}
% \begin{Pcode}[]
% def sum(ns)= {
%   x = 0
%   for n in ns 
%     x = x + n
%   x  
% }

% x = [2,3,4]
% sum(x)  // the x outside is not the x inside sum
% \end{Pcode}
% \end{center}



\subsection{Exercises}
\begin{enumerate}
    \item Let $f:x\mapsto x\sqrt{x^-9}$.  Evaluate $f(4)$.

    \item Suppose $f(a/b)=a+b$ as formally abstracted as $f:a/b\mapsto a+b$.  What would $f(1/2)$ be?
    What would $f(2/4)$ be? 

    \item Suppose $f(a/b)=a+b$ as formally abstracted as $f:a\mapsto (b\mapsto a+b)$.  What would $f(1/2)$ be?
    What would $f(2/4)$ be?  

\end{enumerate}

\chapter{Paradox of the Trapped Variable.}
Think back to substitution in $x(x+3)^c$.  Suppose we use 
the rules to substitute $x\defeq c$.
\begin{center}
    \begin{tikzpicture}[yscale=0.75]
        \node (f) at (0,0) {$x(x+3)^c$};
        \node[below of=f,scale=0.75] {$\times$};
        \node (x1) at (-1,-2) {$x$};
        \node (sqrt1) at (1,-2) {$(x+3)^{c}$}; 
        \node[below of=sqrt1,scale=0.75] {$\circ$};
        \node (su) at (0,-4) {$u^{c}$};
        \node (u) at (2,-4) {$u=x+3$};
        \node (x2) at (1,-6) {$x$};
        \node[below of=u,scale=0.75] {$+$};
        \node (three) at (3,-6) {$3$};
        % \node (x3) at (0,-8) {$x$};
        % \node (x4) at (2,-8) {$x$};
        % \node[below of=x2,scale=0.75] {$\times$};

        \draw[-] (f) -- (x1);
        \draw[-] (f) -- (sqrt1);
        \draw[-] (sqrt1) -- (su);
        \draw[-] (sqrt1) -- (u);
        \draw[-] (u) -- (x2);
        \draw[-] (u) -- (three);
        % \draw[-] (x2) -- (x3);
        % \draw[-] (x2) -- (x4);

    \end{tikzpicture}
\end{center}

One substitution got us into trouble, so lets slow down to see what happened.  Set:
\begin{align*}
    I(x) & = x & 
    K_c(x) & = c.
\end{align*}
You may call $I$ the identity function and the $K$ constant functions.
% but with excessive mathematical training you are likely to overthink 
% the situation.  These two formulas have no use for domains and codomains for 
% instance.  We can safely use these even when there are no sets or mathematics 
% around to make sense of deeper concepts.
Try some substitutions, I tried $I(3)=3$, $I(\clubsuit)=\clubsuit$.
I found $K_3(2)=3$ and $K_3(\clubsuit)=3$ as well.  I even tried 
$K_{\clubsuit}(2)$ and got $\clubsuit$.  I changed $x$ for $y$, 
$I(y)=y$, and $d$ for $c$, $K_d(x)=d$. Next I substituted $x$ for $c$ and got
\begin{align*}
    K_x(x)=x=I(x).
\end{align*}
Now we have a true problem: a constant function should not equal 
an identity function.\footnote{Functions in this sense are so primitive 
they have no domains and codomains.  You can put anything into these functions.}
This is the paradox of the trapped variable.

As my philosophy  colleague Professor Dustin Tucker says, 
``A system studied long enough reveals its paradoxes.'' 
Paradoxes (para = distinct + dox = opinion) are inconsistencies that you can avoid by revisiting the scope 
of your definitions.  Just withhold some options and you wont end up with 
two different options.  It is not a philosophically satisfying resolution,
which is why most paradox hacks lead to schisms. 

So what is the root cause of our paradox of the trapped variable?
The answer are bound (local) verses free variables. 

To avoid trapped variables I use the rules set out by Curry-Feys.
First you need a grammar.  Keeping to context free and learning by example

Why did something so basic fail?  It has to do with variables coming in 
two forms: free and bound.  If you program you might think of a global 
verses local variable.  First things first: variables?
First sort out the data.  One sort will be constants, atoms like an alphabet,
maybe digits, or a word or special symbol. A second sort will be called variables.
That its, variables are symbols from special alphabet we call variables.
This means that a variable can never equal a constant, statements like $x=2$ 
are in strict sense nonsense.  But hold off on that journey for a moment.
Now having all these alphabets we can form strings using the various letters.
We could define arithmetic using digits $0,\ldots, 9$, $+,-,\times,\div$ and some 
variables but lets simplify things to true/false which is long enough to 
explore the idea.  We need to specify a grammar of how allowed formulas can 
be made.  We basically separate the options by $\mid$ (reads as ``or'')
and use patterns to explain structures that are built up recursively.
So for Boolean (true/false) algebra we have true $\top$, false $\bot$, 
and $\wedge$, or $\vee$, and not $\neg$ language might be defined like this.
\newpage    
\begin{lstfloat}
\begin{lstlisting}[mathescape]
    <var>  ::= x | x_<int>
    <Bool> ::= $\top$ 
             | $\bot$ 
             | <var>
             | $\neg$ <Bool> 
             | <Bool> $\vee$ <Bool> 
             | <Bool> $\wedge$ <Bool>.
\end{lstlisting}
\end{lstfloat}

To get started lets return to our use of a grammar.  The diagram we had 
was a tree, what is known as a \emph{parse tree}.  It is the same thing you 
do when you diagram a sentence in grammar school, only with English you can 
sometimes get cycles.  That we got a tree is owed to the fact that the grammars 
for mathematics are basic and gentle, what Chompsky calls \emph{context-free} grammars.

This situation comes about because of two flavors of variables: free and bound,
also called local.  A variable can be bound in many ways, for example 
$\forall x$ binds $x$ to $\forall$, same with $\exists x$.  The binding tells 
us that even if we are using $x$ somewhere else, form this point till 
the end of the block we are simply recycling the name $x$, but its meaning 
is now controlled by the start of the binding.  The binding in the 
substitution examples above is hidden by notation but it is third form 
known as $\lambda$-binding, such as $x\mapsto x+2$ (historically 
$\lambda x.(x+2)$ which is where the name comes from).  This says that 
$x$'s role is to serve as the variable in describing a function.
In the constant function $c\mapsto K_c$ or rather $c\mapsto (x\mapsto c)$.
Likewise $\sqrt{n}{u}$ means $n\mapsto (u\mapsto \sqrt[n]{u})$
Now the point is that a local variable is just reusing a symbol it has 
no visibility outside.

Why did something so basic fail?  It has to do with variables coming in 
two forms: free and bound.  If you program you might think of a global 
verses local variable.  First things first: variables?
First sort out the data.  One sort will be constants, atoms like an alphabet,
maybe digits, or a word or special symbol. A second sort will be called variables.
That its, variables are symbols from special alphabet we call variables.
This means that a variable can never equal a constant, statements like $x=2$ 
are in strict sense nonsense.  But hold off on that journey for a moment.
Now having all these alphabets we can form strings using the various letters.
We could define arithmetic using digits $0,\ldots, 9$, $+,-,\times,\div$ and some 
variables but lets simplify things to true/false which is long enough to 
explore the idea.  We need to specify a grammar of how allowed formulas can 
be made.  We basically separate the options by $\mid$ (reads as ``or'')
and use patterns to explain structures that are built up recursively.
So for Boolean (true/false) algebra we have true $\top$, false $\bot$, 
and $\wedge$, or $\vee$, and not $\neg$ language might be defined like this.
\newpage    
\begin{lstfloat}
\begin{lstlisting}[mathescape]
    <var>  ::= x | x_<int>
    <Bool> ::= $\top$ 
             | $\bot$ 
             | <var>
             | $\neg$ <Bool> 
             | <Bool> $\vee$ <Bool> 
             | <Bool> $\wedge$ <Bool>.
\end{lstlisting}
\end{lstfloat}




Before leaving a word on \emph{sorts}.
Refine this as you like. For instance, one sort $a,b,c,\ldots,m,n,\dots, x,y,z,
x_1,x_2,\ldots$ for numbers, a sort $+,-,\times, [\ldots]_{\ell},\ldots$ for
operators, $\cong, \equiv, \ldots$ for equality.  You decide, it is a made up 
language.



  Formulas are strings over these alphabets.  Each of these has a 
a grammar and a handy notation is 
to separate options by $\mid$ which reads as \emph{or}.  For example, 
a true $\top$, false $\bot$, and $\wedge$, or $\vee$, and not $\neg$ language might be defined like this.
\newpage    
\begin{lstfloat}
\begin{lstlisting}[mathescape]
    <var>  ::= x | x_<int>
    <Bool> ::= $\top$ 
            | $\bot$ 
            | <var>
            | $\neg$ <Bool> 
            | <Bool> $\vee$ <Bool> 
            | <Bool> $\wedge$ <Bool>.
\end{lstlisting}
\end{lstfloat}

So $x$ is a boolean, as is $\top$ and $\neg x\wedge x_3$.  Math languages assume 
also the symbols for parenthesis.

of variables for numbers will be $m,n,\ldots,
x,y,\ldots$ whereas a separate sort of variable is used for operations, such as
$+,-,\times,\ldots$, and still another $A,B,C,\ldots$ for sets and so forth.
Sorts used in this way have formal meaning in logic and I mention that because 
you will come across it in examples of formal methods---the growing field 
blending the idea of proving theorems with proving programs, what we will 
need to make safe self-driving cars and video games that you can't cheat.

Starting with a string $M$ with symbols of various sorts, the task is to 
substitution variable $x$ in by another string $N$.  If there is only one 
variable around you may think of this as $M(N)$.   Since $M$ may have many 
variables let us be specific:
\begin{align*}
    M[x\defeq N]
\end{align*}
To see how to do this lets break the down th process of making $M$.
For example we might be in context of a simple calculator.  So our constants 
are the digits $0,1,\ldots,9$, and we have also $+,-,\times,\div$.
Variables we call $x,y,z$ and if we need more $x_n$ where is a sequence 
of digits will do.


We can start out small with two sorts of data.  One 
sort are atomic symbols, digits, an alphabet or some other meaning 
of a constant.  The second sort are variables.  Then we formula 
$M$ is either an atom $a$, a variable $x$, or the concatenation 
of two formulas $K$ and $L$.  There is a popular notation for this 
is to separate each case by the stroke $\mid$ which reads as ``or''.
\begin{lstlisting}[language=Hidris,mathescape]
    F[X] = 0 | 1 | ... | 9 | + | -
         | x
         | <F[X]> cat <F[X]>
\end{lstlisting}

So if our atoms are $\clubsuit,\heartsuit, \spadesuit,\diamondsuit$
and 
\begin{lstlisting}[language=Hidris,mathescape]
    F[X] = a:A 
         | x:X
         | K:F[X] cat L:F[X]
\end{lstlisting}

\begin{lstlisting}[language=Hidris,mathescape]

    M[x:=N] =
        match M with 
                a:A $\Rightarrow$ a
                y:X $\Rightarrow$ if x=y then N else y
            K cat L $\Rightarrow$ K[x:=N] cat L[x:=N]
             y $\mapsto$ L $\Rightarrow$ if x=y then y $\mapsto$ L 
                                         else if x free in L 
    \end{lstlisting}
    

\begin{lstlisting}[language=Sava,mathescape]
F[X] = a:A 
     | x:X
     | M:F[X] cat N:F[X]
     | x:X $\mapsto$ M:F[X]

M[x:=N] =
    match M with 
            a:A $\Rightarrow$ a
            y:X $\Rightarrow$ if x=y then N else y
        K cat L $\Rightarrow$ K[x:=N] cat L[x:=N]
         y $\mapsto$ L $\Rightarrow$ if x=y then y $\mapsto$ L 
                                     else if x free in L 
\end{lstlisting}



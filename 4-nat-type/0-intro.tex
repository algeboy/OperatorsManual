
There are a lot ways we may encounter number of which grammar is a powerful 
but variable method.  For instance we already know of these
\begin{quote}
    \code{<<Tally>>::= _ | I <Tally>}\\
    \code{<<Nat>>::= 0 | S <Nat>}\\
    \code{<<Peano>>::= 0 | <Peano>+}
\end{quote}
To look past these cosmetic differences we often turn to a 
notation that lists the before-after and ignores specific choices of 
grammar.
To summarize there are two rules that have emerged as the only means to introduce 
natural numbers. From nothing we get the natural number $0:\mathbb{N}$.
From any natural number $k:\mathbb{N}$ we obtain another $S(k):\mathbb{N}$.
It is customary to denoted all the hypothesis required for a deduction in a column or 
several columns above a horizontal line over the conclusion.  Hence,
\begin{gather}
    \tag{Introduction$_{\mathbb{N}}$}
    \overline{0:\mathbb{N}}
    \qquad
    \frac{k:\mathbb{N}}{S(k):\mathbb{N}}
\end{gather}
This notation, known as \emph{Gentzen's sequent calculus} will eventually
eclipse our use of grammars since it captures just what is necessary in terms of
data and is indifferent to small grammatical choices.

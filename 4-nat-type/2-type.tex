
\section{Natural Number Type}
We now take a slow walk through the data type of natural numbers 
in preparation for future harder forms of induction.  We 
no include a list of annotations.
\begin{itemize}
\item     $A:\Type$ to indicate that $A$ is a type.
\item $P:\Prop$ if $P$ is a type of data with at most one term.
\end{itemize}
For example, $(x=y):\Prop$ because it has most the term $\refl_x:(x=x)$.
It may in fact be uninhabited such as $2=3$.

The forgoing implicit requirement is that we have 
some context in which we can build our number.  Natural 
numbers being natural can be formed from no assumptions.
So our formation rule has an empty premise.
\begin{gather}
\tag{$\form{\mathbb{N}}$}
    \overline{\mathbb{N}:\Type}
\end{gather}
We already saw the introduction rules.  There are two 
and this is what makes the type \emph{inductive}, i.e. 
it will be used case-by-case.
\begin{gather}
\tag{$\intro{\mathbb{N}}^0$}
    \overline{0:\mathbb{N}}
\\
\tag{$\intro{\mathbb{N}}^S$}
    \frac{k:\mathbb{N}}{S(k):\mathbb{N}}    
\end{gather}
Elimination means to be given data annotated by a type 
and possibly other rules and apply it towards a goal.
Usually this eliminates the annotation and hence the name.
For natural numbers induction needs several auxiliary terms.
When a premise is implicit (you know it from context) we 
put it in parenthesis.
\begin{gather}
\tag{$\elim{\mathbb{N}}$}
    \begin{array}{crl}
    (P:\mathbb{N}\to \Type) && \\
        & n & :\mathbb{N}\\        
        & base & : P(0)\\
    (k:\mathbb{N})    & indHyp(k) &: P(k)\to P(S(k))\\
    \hline 
        & \text{ind}(n,base,indHyp) & : P(n)
    \end{array}
\end{gather}
Now we relate introductions to eliminations by composing 
the two and deciding upon the rule that relates them.
These are known as \emph{computation} or \emph{comprehension}
rules.
\begin{gather}
\tag{$\comp{\mathbb{N}}^0$}
    \begin{array}{crl}
        & 0 & :\mathbb{N}\\
        & base & : P(0)\\
    (k:\mathbb{N})    & indHyp(k) &: P(k)\to P(S(k))\\
    \hline 
        & \text{ind}(n,base,indHyp) & \defeq base
    \end{array}\\
\tag{$\comp{\mathbb{N}}^{S}$}
    \begin{array}{crl}
        & S(k) & :\mathbb{N}\\        
        & base & : P(0)\\
    (k:\mathbb{N})    & indHyp(k) &: P(k)\to P(S(k))\\
    \hline 
        & \text{ind}(S(k),base,indHyp) & \defeq  indHyp(k)
    \end{array}
\end{gather}

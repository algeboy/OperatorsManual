
The first lesson of algebra is that everything you see is a variable. you can improve how we study equations 

A function takes an input and use some process to determine an output,
under the premise that this process is repeatable.  So reading the 
time off a watch is not a function because time keeps changing, 
but converting time from hours to minutes is a function because 
every hour has 60 minutes.  From that philosophical understanding 
two functions are obvious:
\begin{align*}
    I(x) & = x & 
    K_c(x) & = c.
\end{align*}
We call the first an \emph{identity} function and the second is a \emph{constant}
function.  This notation is misleading, technically $I$ is a symbol
which denotes some unknown process such that when applied to data $\clubsuit$,
its output, often denoted $I(\clubsuit)$, is given input data unchanged.
So $I(\clubsuit)=\clubsuit$ is a judgement we can make about an identity 
function not a program.

There is a bit of implicit information in our use 
of parenthesis and what they mean.  Traditionally $I$ and $K$ are 
the functions.  An input $\clubsuit$ passed to a function $I$ 
yields an output denoted $I(\clubsuit)$.  In the case of the identity 
function $I$ does nothing to modify the input so we can judge 
$I(\clubsuit)=\clubsuit$.  Since this applies for any input we 
can abstract over the inputs by replacing their role with a variable 
$x$ and write $I(x)=x$  It is not a definition of $I$ so much as 
a consequence.  Of course we could use the outcome rule to conjure 
a perspective algorithm to perform the function.  When we do this 
we often insert some extra language like ``define I(x)=x'' or 
use the Walrus notation $I(x)\defeq x$.  In programs words like 
define are abbreviated.  For example, the following two programs satisfy
the identity function rule without following the same process.
\begin{lstlisting}
    def doNothing(x) = x 
    def doNothingUseful(x) = compute 200! then return x
\end{lstlisting}
Much of the feasibility of algebra comes down to appreciating 
different processes that achieve the same overall calculations.


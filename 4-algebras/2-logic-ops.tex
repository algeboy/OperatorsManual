
\section{Logical operators}


\subsection{Predicates, Propositions, and Judgement}
If we make a sentence that declares a fact then we have the option 
to ask if it s true.  ``2 is even ''--- judged as true, ``5 is even''---judged as false,
``every integer n is even.''---it might be true, though maybe some work to 
explain why.  Call such sentences (and more generally paragraphs, pages, or books 
as might be necessary to explain the claim) \emph{propositions}\index{proposition}.

Frege suggested a notation where a sentence $P$ would be judged by writing the
\emph{Urteilsstrich} (judgement bar) $|P$.  In mood to be very precise,
Frege further observed that sometimes we want to judge of the content of sentence 
and other times we just want to judge the structure of the argument as valid.
Consider the following argument $P$.
\begin{quote}
    All men are mortal; \\
    Socrates is a man;\\
    therefore Socrates is mortal.
\end{quote}
This is certainly true.  What about this variation?
\begin{quote}
    All Pok\'emon are immortal; \\
    Pikachu is a Pok\'emon;\\
    therefore Pikachu is immortal.
\end{quote}
On the one hand 
this is a valid argument structure, known as a syllogism,
and all we have done is change objects of the sentence.  
However, now we deal with fictional species (Pok\'emon) and 
there is not truth to the sentences in terms of meaning.
So we could write the judgement bar $|P$ to indicate the argument is valid,
but to specify that the meaning is not true we add some notation.
Frege elected to symbolize the Inhaltsstrich (content bar)
$-P$ to denote the content of a sentence instead of the sentence itself.  
Thus to judge the content of a proposition $P$ as true is to write $|-P$.
Thus as most of us are comfortable with proof structures and want to speak 
to the content, the ``turnstyle'' $\vdash P$ has become the fused combination 
of judging content which is nearly always what is meant by ``P is true''.





\begin{itemize}
    \item $|P$ means the structure of the argument in $P$ is a valid.
    \item $-P$ means the content of the sentence.
    \item $\vdash P$ means the content of $P$ can be proved.\footnote{A proof is
    a rooted tree with root $q$, leaves either from $P$ or axioms, and all
    internal nodes labeled by a logical operation.  In computer science these
    are known as \emph{fan-in circuits}.} 
    \item $\vDash P$ means the content of $P$ is true, though maybe without proof
    (example it may be a definition, an axiom, or in obscure cases 
    it could be a proposition which prevents itself from being contradicted 
    or proved.)

    \item $P_1,\ldots,P_n \vdash Q$ means when the content of $P_1,\ldots,P_n$ are valid 
    then the content of $Q$ can be proved.  This is also often written 
    $\Gamma \vdash Q$ where $\Gamma$ is stand in context, meaning all 
    assumptions assumed in the statement.
    \item $P_1,\ldots,P_n \vDash Q$ means when the content of $P_1,\ldots,P_n$ are valid 
    then the content of $Q$ also valid, though again perhaps without proof.
\end{itemize}

\begin{remark}[Towards Heyting Algebra]
    Although $\Rightarrow$ is called the ``implies'' operator, to write $p\Rightarrow q$ 
    does not say that ``$p$ implies $q$''.  In particular, $p\Rightarrow q$ 
    is a binary operation, similar to $p\wedge q$, $\not p$, or $2+3$.  None 
    of these is innately ``true'', though they can take on a value of true 
    they may also take on false.  We do not yet know.  However to write 
    $p\vdash q$ is to say that $p$ does imply $q$.  As $\Rightarrow$ is an 
    operator it can, and is, explored through algebra.  The algebra is 
    known today as Heyting algebra.

    The solution is to write $\vdash p\Rightarrow q$, $\vdash p\wedge q$, 
    and etc. as these all declare the operator's value is true.  Even so 
    a popular shorthand is that if preceeded by words like ``Theorem, Claim,
    Proposition'' then the $\vdash$ is dropped.
\end{remark}

% Even so there is a modern nuance.  When a sentence talks about itself in the 
% negative it can be make it impossible to prove while also making a declaration 
% of validity.  
% \begin{enumerate}
%     \item Russell: the set of all sets that do not contain themselves.
%     \item G\"odel: the proposition  that not all theorems have proofs.
%     \item Turing: the program that decides of other programs will not halt.
% \end{enumerate}

    

% If this is false then every theorem has 
% Mathematicians cannot prove some palway's lie'', if true, would mean you my reader
% (yes you are a mathematician) are lieing right now to yourself, even about 
% being a liar.  And do this claim could not be true


% Today we find an even more nuanced landscape where the meaning of judgement 
% falls into question.  There are things we simply assign as names or 
% definitions, such as ``$n$ is even if $\exists k.(n=2k)$''.  So 
% there $n$

Other sentences have the form that they become propositions after clarifying
some variable quantity: ``n is even'', for example cannot be true or false, but
either by replacing $n$ with a number quantifying with terms like $\exists$ or 
$\forall$, we arrive at a proposition.  This second class of sentence is a
\emph{predicate}.\index{predicate} In our telling, all propositions are also
predicates but not all predicates are propositions.

When given predicates we extend the notation 
\[
    P_1,\ldots,P_n \vdash Q 
\]
to now mean that given any substitution of the variables to make the $P_i$ true, 
then $Q$ is true.  Today these notations are also written as 
\[
    \frac{P_1,\ldots,P_n}{Q}\qquad \begin{array}{c} P_1\\ \vdots\\ P_n\\ \hline Q\end{array}.
\]

\subsection{Simple logic}
There are many systems of logic required for applications.  Algebra seems to be 
adaptable to most but it may help to have one that we consider as a foundation.

First, we will permit ourselves to add unnecessary premises.
\begin{gather}
    \tag{Weakening}
    \frac{
        \Gamma\vdash Q
    }{
        \Gamma, P \vdash Q 
    }
\end{gather}
Second we allow our selves to re-use a premise.
\begin{gather}
    \tag{Reuse}
    \frac{
        \Gamma, P,P\vdash Q
    }{
        \Gamma, P \vdash Q 
    }
\end{gather}
In linear logic for example this law is removed in favor of keeping 
track of the resources necessary to do logic.  For example, if $P$ 
is a piece of data it may be overwritten to produce the data for $Q$
and hence if we want to use the value twice we will need to copy it.

\textbf{And}
\begin{gather}
    \tag{$\intro{\wedge}$}
    \frac{\Gamma \vdash P\qquad \Upsilon \vdash Q}{\Gamma, \Upsilon \vdash P\wedge Q}\\
    \tag{$\elim{\wedge}$}
    \frac{\Gamma \vdash P\wedge Q}{\Gamma\vdash P}
    \qquad
    \frac{\Gamma \vdash P\wedge Q}{\Gamma\vdash Q}
\end{gather}

\textbf{Implies}
\begin{gather}
    \tag{$\intro{\Rightarrow}$}
    \frac{\Gamma, P \vdash Q}{\Gamma \vdash P\Rightarrow Q}\\
    \tag{$\elim{\Rightarrow}$}
    \frac{\Gamma \vdash P\Rightarrow Q, P}{\Gamma\vdash Q}
\end{gather}


\textbf{Falsity} this only has an elimination rule, it says that if we can prove 
something is false then we can prove anything.
\begin{gather}
    \tag{$\elim{\bot}$}
    \frac{\Gamma\vdash\bot}{\Gamma\vdash P}
\end{gather}


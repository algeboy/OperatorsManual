
\section{Logical operators}


\subsection{Predicates, Propositions, and Judgement}
If we make a sentence that declares a fact then we have the option 
to ask if it s true.  ``2 is even ''--- judged as true, ``5 is even''---judged as false,
``every integer n is even.''---it might be true, though maybe some work to 
explain why.  Call such sentences (and more generally paragraphs, pages, or books 
as might be necessary to explain the claim) \emph{propositions}\index{proposition}.

Frege suggested a notation where a sentence $P$ would be judged by writing the
\emph{Urteilsstrich} (judgement bar) $|P$.  In mood to be very precise,
Frege further observed that sometimes we want to judge of the content of sentence 
and other times we just want to judge the structure of the argument as valid.
Consider the following argument $P$.
\begin{quote}
    All men are mortal; \\
    Socrates is a man;\\
    therefore Socrates is mortal.
\end{quote}
This is a valid argument structure, known as a syllogism.
So we could write the judgement bar $|P$ to indicate the argument is valid.
However, in other circumstances we want to know if the content discussed by 
the sentence is valid, not merely the type of argument.  Here the reader 
will need to investigate if Socrates was a man and if all men are mortal.
This is the content.  Frege elected to symbolize the Inhaltsstrich (content bar)
$-P$ to denote the content of a sentence instead of the sentence itself.  
Thus to judge the content of a proposition $P$ as true is to write $|-P$.
Thus as most of us are comfortable with proof structures and want to speak 
to the content, the ``turnstyle'' $\vdash P$ has become the fused combination 
of judging content which is nearly always what is meant by ``P is true''.

Even so there is a modern nuance.  When a sentence talks about itself in the 
negative it can be make it impossible to prove while also making a declaration 
of validity.  
\begin{enumerate}
    \item Russell: the set of all sets that do not contain themselves.
    \item G\"odel: the proposition  that not all theorems have proofs.
    \item Turing: the program that decides of other programs will not halt.
\end{enumerate}

    

If this is false then every theorem has 
Mathematicians cannot prove some palway's lie'', if true, would mean you my reader
(yes you are a mathematician) are lieing right now to yourself, even about 
being a liar.  And do this claim could not be true


Today we find an even more nuanced landscape where the meaning of judgement 
falls into question.  There are things we simply assign as names or 
definitions, such as ``$n$ is even if $\exists k.(n=2k)$''.  So 
there $n$




\begin{itemize}
    \item $|P$ means the structure of the argument in $P$ is a valid.
    \item $-P$ means the content of the sentence.
    \item $\vdash P$ means the content of $P$ can be proved.
    \item $\vDash P$ means the content of $P$ is true, though maybe without proof
    (example it may be a definition, an axiom, or in obscure cases 
    it could be a proposition which prevents itself from being contradicted 
    or proved.)

    \item $P_1,\ldots,P_n \vdash Q$ means when the content of $P_1,\ldots,P_n$ are valid 
    then the content of $Q$ can be proved.  This is also often written 
    $\Gamma \vdash Q$ where $\Gamma$ is stand in context, meaning all 
    assumptions assumed in the statement.
    \item $P_1,\ldots,P_n \vDash Q$ means when the content of $P_1,\ldots,P_n$ are valid 
    then the content of $Q$ also valid, though again perhaps without proof.
\end{itemize}


Other sentences have the form that they become propositions after clarifying
some variable quantity: ``n is even'', for example cannot be true or false, but
either by replacing $n$ with a number quantifying with terms like $\exists$ or 
$\forall$, we arrive at a proposition.  This second class of sentence is a
\emph{predicate}.\index{predicate} In our telling, all propositions are also
predicates but not all predicates are propositions.

Suppose that $P$ is a list of predicates 
and $q$ is a single predicate.  Then there are at least three possible 
ways that we might think of going from $P$ to $q$.
\begin{description}
    \item[Derives] write $P\vdash q$ when there is a proof from $P$ to $q$.
    (Formally: a proof is a rooted tree with root $q$, leaves either from $P$ or axioms, 
    and all internal nodes labeled by a logical operation.  In computer science 
    these are known as \emph{fan-in circuits}.)
    
    \item[Entails] write $P\vDash q$ when for every assignment of variables that makes 
    $P$ true, that assignment of variables makes $q$ true.

    \item[Implies] $p\Rightarrow q$ is merely a binary operator that takes
    predicates $p$ and $q$ and outputs a new predicate denoted $p\Rightarrow q$.
    So $P\Rightarrow q$ can take on a value of $\top$ but we cannot say 
    that it is true, it is just a predicate that it may be true.
\end{description}
A mathematician is prone to confuse these three and with good reason.
The first leads to the second, and the second leads to the third \emph{being true},
in symbols
\begin{itemize}
    \item $(p\vdash q)\vdash (P\vdash q)$.
    \item $\displaystyle (p\vDash q)\vdash \left(\vdash \left(\bigwedge_{p\in P} p\right) \Rightarrow q\right)$.
\end{itemize}
Notice $\vdash X$ means it takes nothing to prove $X$, in other words,
this is a fancy way to infer that $X$ is true.  

\begin{remark}
    So is it strictly speaking meaningful to write something like the 
    following? 
    \begin{quote}
        \textbf{Theorem.} $n=2k\Rightarrow \gcd(k,n+2)=2$.
        \hfill{\color{BrickRed} (Meaningless)}
    \end{quote}
    On the one hand this is not a statement of fact any more than 
    to say ``$n$ is even.''  To write $p\Rightarrow q$ is to give 
    a predict, not to argue that it is true.  A proper statement 
    might be 
    \begin{quote}
        \textbf{Theorem.} $n=2k\vdash \gcd(k,n+2)=2$.
        \hfill{\color{BrickRed} (Formally Correct)}
    \end{quote}

    On the other hand, the label ``Theorem'' could be tasked with the 
    role of decorating the predicate with an air of validity, as if 
    to implicitly tack on the ending ``...is true'' in invisible red ink:
    \begin{quote}
        \textbf{Theorem.} ``$n=2k\Rightarrow \gcd(k,n+2)=2$'' {\color{BrickRed} is true.}
    \end{quote}
    Yet, proponents of such fix should ponder if they really want to say
    Theorem. ``$n=2k$ is even'' is true.

    \noindent\rule{\textwidth}{1pt}
    Perhaps a solution worth considering is that in mathematics we 
    despense with the notation we do not entirely understand.  We may write 
    \begin{quote}
        \textbf{Theorem.} If $P$  then $q$.
        \hfill{\color{BrickRed} (Acceptable)}
    \end{quote}
\end{remark}
Finally, be mindful that G\"odel proved that the converse is false.  There
examples where we can write $P\vDash q$ but not $P\vdash q$.   This is sometimes
misrepresented as saying ``there are theorems which are true but cannot be
proved.'' That however looses the subtlety behind his claim by using the
vagueness of the word ``theorem'' to simultaneously mean both $\vDash$ and
$\vdash$.  More accurate reading would be that it may be impossible to find
counter-examples where $P$ is true and $q$ is false. Notice I did not say
``there are no counter-examples'' just that there is no process to find a
counter-example.  As everyone knows failure find counter-examples is not itself
a proof of anything. This leads one simply to surmise either by decree of axioms 
that such a situation should now count as a new form of proof, or to accept that 
some situations like this exist.

If you want to truly explore $\Rightarrow$ as an operator explore Heyting Algebras.


There are great number of logical operators to observe as well.

\begin{gather}
    \tag{$\intro{\wedge}$}
    \frac{\Gamma \vdash P, \Upsilon \vdash Q}{\Gamma, \Upsilon \vdash P\wedge Q}\\
    \tag{$\elim{\wedge}$}
    \frac{\Gamma \vdash P\wedge Q}{P}
    \qquad
    \frac{\Gamma \vdash P\wedge Q}{Q}
\end{gather}
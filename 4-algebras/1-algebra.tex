
\section{Free Algebra}
Let us now look back at some of our first constructions.
A context-free grammar $\mathcal{G}$ defines for us a number 
useful notions.  One is the concept of a language 
$\mathcal{L}(\mathcal{G})$, the strings 
accepted by the grammar.  The other is the concept of a production 
rule, which is a non-terminal term \code{<term>::=....} the string on the 
right is from existing productions rules and terminal symbols.  So 
now we turn even this into algebra.  Each production rule in fact defines a 
language, so what we mean by the language of the grammar is that we have 
starting symbol, e.g. as signaled by \code{<<start>>::=...}.  

To start let us associate types to the languages.  
First every language is given the type of its production rule.
(Note that because every production rule determines a language 
it follows that each production rule names a type.)

\begin{example}
    Recall that \code{<Nat> ::= 0 | S(k:Nat)} is shorthand for 
\begin{Gcode}[]
<Nat> ::= 0
<Nat> ::= S(k:Nat)
\end{Gcode}
So the first production rule says \code{0:Nat}, which we may write also 
as $0:\mathbb{N}$.
The second production says that \code{S(k):Nat}, which again we abbreviate 
as $S(k):\mathbb{N}$.
Now lets use these production rules to introduce some functions.
\begin{gather*}
    \frac{
        \vdash 0:\mathbb{N}
    }{
        \text{func}\langle x\mapsto 0\rangle:\{\bot\}\to \mathbb{N}
    }
    \qquad 
    \text{null}\defeq \text{func}\langle x\mapsto 0\rangle
    \\
    \frac{
        n:\mathbb{N}\vdash S(n):\mathbb{N}
    }{
        \text{func}\langle x\mapsto S(x)\rangle:\mathbb{N}\to \mathbb{N}
    }
    \qquad 
    ++\defeq \text{func}\langle x\mapsto S(x)\rangle
\end{gather*}
we have named the functions for convenience and we can now write these 
in the usual way:
\begin{align*}
    \text{null}&:\{\bot\}\to \mathbb{N}
    & 
    \text{null}&:x\mapsto 0\\
    ++& :\mathbb{N}\to \mathbb{N}
    & 
    ++&:n\mapsto S(n)
\end{align*}
\end{example}


\begin{example}
\begin{center}
\begin{Gcode}[]
<String<a>> ::= 
<String<a>> ::= a <String<a>>
\end{Gcode}
\end{center}
This prescribe the language $a^*\defeq \{a^n\mid n:\mathbb{N}\}$ where 
$a^n=\overbrace{a\cdots a}^n$. So we say it the strings over $a$ type.
These production rules to introduce some functions.
\begin{gather*}
    \frac{
        w:a^*\vdash aw:a^*
    }{
        \text{func}<x\mapsto ax>:a^*\to a^*
    }
\end{gather*}
In the customary notation
\begin{align*}
    L_a &:a^*\to a^* 
    &
    L_a& : x\mapsto aw
\end{align*}
\end{example}


\begin{example}
Boolean algebra.
\begin{center}
\begin{Gcode}[]
<String<Char>> ::= 
<String<Char>> ::= <Char> <String<Char>>
\end{Gcode}
\end{center}
This prescribe the language $\Sigma^*$ of strings over the alphabet $\Sigma$.
These production rules to introduce some functions.
\begin{gather*}
    \frac{
        a:\Sigma, \quad w:\Sigma^*\vdash aw:\Sigma^*
    }{
        func<x\mapsto ax>:\Sigma^*\to \Sigma^*
    }
\end{gather*}
Thus we can define several left multiplication operators
\begin{align*}
    & (a:\Sigma)
    &
    L_a &:\Sigma^*\to \Sigma^* 
    &
    L_a &: x\mapsto ax
\end{align*}
\end{example}


\begin{example}
    Now lets try an alphabet $\Sigma$ (i.e. \code{<Char>}).
\begin{center}
\begin{Gcode}[]
<Boolean<X>> ::= $\top$
                | $\bot$
                | $\neg$ <Boolean<X>>
                | <Boolean<X>> $\wedge$ <Boolean<X>>
                | <Boolean<X>> $\vee$ <Boolean<X>>
\end{Gcode}
\end{center}
This prescribe the language $B\langle X\rangle$ of strings over the alphabet $\Sigma$.
    These production rules to introduce some functions.
    \begin{gather*}
        \frac{
            \phi:B\langle X\rangle \vdash \neg\phi:B\langle X\rangle
        }{
            \text{func}<x\mapsto \neg x>:B\langle X\rangle \to B\langle X\rangle
        }
    \end{gather*}
    which we can name perhaps
    \begin{gather*}
        \text{neg}:B\langle X\rangle \to B\langle X\rangle
        \qquad \text{neg}:\phi \mapsto \neg \phi
    \end{gather*}
    There are also two bivalent operators
    \begin{gather*}
        \frac{
            \phi,\psi:B\langle X\rangle \vdash \phi\wedge \psi:B\langle X\rangle
        }{
            \text{func}\langle (x,y)\mapsto x\wedge y\rangle:
             B\langle X\rangle\times B\langle X\rangle \to B\langle X\rangle
        }\\
        \frac{
            \phi,\psi:B\langle X\rangle \vdash \phi\wedge \psi:B\langle X\rangle
        }{
            \text{func}\langle (x,y)\mapsto x\vee y\rangle:
            B\langle X\rangle\times B\langle X\rangle \to B\langle X\rangle
        }
    \end{gather*}
    which we could recorder as to named functions
    \begin{gather*}
        \text{and}:B\langle X\rangle \to B\langle X\rangle
        \qquad \text{and}:(\phi,\psi) \mapsto \phi\wedge \psi\\
        \text{or}:B\langle X\rangle\times B\langle X\rangle \to B\langle X\rangle
        \qquad \text{or}:(\phi,\psi) \mapsto \phi\vee\psi
    \end{gather*}

\end{example}



\begin{theorem}
    For every context-free grammar $\sigma$ and every type $X$, 
    there is a $\sigma$-algebra on the language $F_{\sigma}\langle X\rangle$.
    In particular, for every signature there is an algebra.
\end{theorem}

\begin{theorem}
    Suppose $\sigma$ a homogeneous signature.
    If $A$ is a $\sigma$-algebra, $X$ is a type, and $a_*:A^X$ is a function 
    then evaluation is a homomorphism $\text{eval}_{a_*}(p(X))\defeq p(a_*)$ 
    from $F_{\sigma}\langle X\rangle\to A$.
\end{theorem}


\section{Heterogeneous}

Now lets try an heterogeneous signature.
\begin{center}
\begin{Gcode}[]
<Ring> ::= 0 
         | 1
         | - <Ring>
         | <Ring> + <Ring>
         | <Ring> * <Ring>
<<Ring>Module> ::= 0
           | - <Module>
           | <Module> + <Module>
           | <Ring> * <Module>
\end{Gcode}
\end{center}
Because we have not specified a start symbol this prescribes two languages.
The rings and the modules.  So when we add variables we can add it both.
\begin{center}
\begin{Gcode}[]
<Ring<X>> ::= x:X
            | 0 
            | 1
            | - <Ring<X>>
            | <Ring<X>> + <Ring<X>>
            | <Ring<X>> * <Ring<X>>
<<Ring>Module<Y>> ::= y:Y
            | 0
            | - <Module<X>>
            | <Module<Y>> + <Module>
            | <Ring<X>> * <Module<Y>>
\end{Gcode}
\end{center}
    
From these we craft operators, first for rings with variables $X$ we 
write the type as $R\langle X\rangle$ and observe the following homogeneous operators.
\begin{align*}
    0&:\Unit\to R\langle X\rangle \\
    1&:\Unit\to R\langle X\rangle \\    
    e& :X\to R\langle X\rangle\\
    -&: R\langle X\rangle^1\to R\langle X\rangle\\
    +&:R\langle X\rangle^2\to R\langle X\rangle\\
    *&:R\langle X\rangle^2 \to R\langle X\rangle
\end{align*}    
For the next module type we write ${_{R\langle X\rangle} M}\langle Y\rangle$ and observe the 
following heterogeneous operators.
\begin{align*}
    0&:Unit\to {_{R\langle X\rangle} M} \\
    e& :Y\to {_{R\langle X\rangle} M}\\
    -&: {_{R\langle X\rangle} M}\to  {_{R\langle X\rangle} M}\\
    +&:{_{R\langle X\rangle} M}\langle Y\rangle\times {_{R\langle X\rangle} M}\langle Y\rangle\to 
    {_{R\langle X\rangle} M}\langle Y\rangle\\
    *&:R\langle X\rangle\times {_{R\langle X\rangle} M}\langle Y\rangle\to 
    {_{R\langle X\rangle} M}\langle Y\rangle    
\end{align*}

A (strict) $\sigma$-algebra is a an assignment of 
types to each production rule called the \emph{base}
(or sometimes the conductor)
along with functions of the typed signatures matching the grammar.
A (weak) $\sigma$-algebra is a an assignment to an isomorphic grammar.
So for example we can rename the symbols or change the order of terms in the 
production rules but do so bijectively.  (Tree-rewriting.)  So for example 
we will be allowed to compare $(\mathbb{R},+)$ with $((0,\infty),\cdot)$
via the homomorphisms $e^x$ by treating these as weak algebras over a 
bivalent infix grammar.



Recall that a typed function has only a left-to-right order of inputs, not 
the nuance of fractional, exponential and etc. notation.  So we loose 
something when we push down to operators.  
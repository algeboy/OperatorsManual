\section{Context sensitive grammar}

Now that we have seen a grammar given by production rules we can explain 
in what way context-free grammars are ``context-free'', and how they 
come to parse as trees.  Consider the situation discussing square-roots 
of natural numbers.
\begin{center}
\begin{Gcode}[]
+<Root> ::= + sqrt(<Nat>)
-<Root> ::= - sqrt(<Nat>)
\end{Gcode}
\end{center}
In this case \code{sqrt{2}:<Root>}.  However, we do not know if this 
was meant as the positive or negative root.  That information 
was left to the surrounding context in which the term was parsed.  
This is a very useful situation as often in algebra we do want to 
speak within a context.  Yet because of the ambiguity if we 
introduce a root in this way and then seek to eliminate we 
cannot be certain of many possible paths got us here.

Meanwhile 
to make this context-free grammar we could use 
\begin{center}
\begin{Gcode}[]
<Root> ::= + sqrt(<Nat>)
         | - sqrt(<Nat>)
\end{Gcode}
\end{center}
Now \code{+sqrt(2):<Root>} and \code{-sqrt(2):<Root>} but 
this grammar rejects \code{sqrt(2)}.  The grammar is less forgiving 
but more precise in the content it now holds for us.

\section*{Exercises}

\begin{enumerate}
    \item Run the necessary steps of recursion to add $5+3$.
    
    \item Prove that for natural numbers $m+n=n+m$.
    
    \item Write me a thank you letter that I didn't ask you to prove $\ell+(m+n)=(\ell+n)+m$.
    
    \item Suppose that we want to able to add a tally mark to either the left (L) or right (R).
    Define a grammar that allows this.  Explain why this is not a model for the natural numbers.

    \item Define multiplication of natural numbers from the Peano grammar given.  Use it to multiply $2\cdot 3$
    step by step.
    
    \item A function $f$ on the natural numbers is a successor homomorphism if 
    $f(S(k))=S(f(k))$.  What are all the successor homomorphisms.  Which ones preserve also $0$?
        
    \item Turn the following function into code in both functional and procedural dialects.
    \begin{align*}
        f(n) & = \begin{cases}
                    0 & n=0\\
                    S0 & n=S(k)
        \end{cases}
         =\begin{cases} 0 & n=0 \\ 1 & \text{else}\end{cases}.
    \end{align*}
    
    
    \item Write a program that defines natural numbers, giving its usual operations of addition 
    and multiplication from their inductive definition.  It wont be fast, that is not the point,
    it is practicing to translate induction into programs.
\end{enumerate}
\chapter{Natural numbers as grammar}
\index{natural numbers|(}
Nothing typifies induction to a mathematician quite like 
the natural numbers.  So I just need to convince you this is 
also just grammar.  Then the two will start to become inseparable 
in your own mind.

Math in early history (early childhood) is counting.  Count 
pebbles or beads and give the patterns names
\begin{center}
    $0\defeq$ \underline{\hspace{5mm}}, 
    $1\defeq$ \StrokeOne,
    $2\defeq$ \StrokeTwo,
    $3\defeq$ \StrokeThree,
    $4\defeq$ \StrokeFour,
    $5\defeq$ \StrokeFive,...
\end{center}
One hypothesis for the symbol
``0'' is that it looks like the shape left by removing the last pebble from
a sand table leaving behind no pebbles.  Is this writing or is this math?

The Walrus $\defeq$ is notation for naming
also called ``assignment''.
\begin{quote}
    ``James $\defeq$ the author.''\\
    ``N is 4.''
\end{quote}
On the left of $\defeq$ should be an as yet unused string of symbols and on the right 
one already known to the context.  Once we have named $M\defeq N$, then 
everywhere we use $M$ it is understood that we intend it to be $N$.
In this way $M=N$ because they are used interchangeably.  We say that 
$M$ is \emph{judgementally} (or \emph{nominally}) equal to $N$, meaning that there is 
nothing to be decided, it is so by declaration.\index{judgemental equality}\index{nominal|see{judgemental}}
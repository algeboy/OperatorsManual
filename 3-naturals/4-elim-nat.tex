\section{Eliminating grammar}    
Now that we have created numbers we shall want to use them.
We can of course use them to count: clear the count with the operator $0$
and apply successors when we add something to the count.

\begin{remark}{}
    Some argue that $0$ does not ``belong in'' $\mathbb{N}$ because 
    we begin counting at $1$.  Others argue that Peano's postulates
    clarify that $0$ is a natural number and begin counting from there.\\

    Both perspectives miss the point.  Peano is telling us that $\mathbb{N}$
    is not a set of numbers at all.  Instead it is two operators:
    \begin{itemize}
        \item $0$ clears the slate, preparing for a new count.
        \item $S$ successor advances the count.
    \end{itemize}
    So indeed $0$ is required at the start \emph{and} the first count will be $1$.
    The set of numbers created is immaterial.  Its the operations we need for counting.
\end{remark}

An obvious improvement in counting is to make the process 
modular and parallel.  For example we can have two separate counts 
later combined by addition.
\begin{center}
    \begin{tabular}{c|ccccc}
    ``add'' & \StrokeOne & \StrokeTwo & \StrokeThree & \StrokeFour & \StrokeFive\\
    \hline 
    \StrokeOne & \StrokeTwo & \StrokeThree & \StrokeFour & \StrokeFive & \StrokeOne \StrokeFive\\
    \StrokeTwo & \StrokeThree & \StrokeFour & \StrokeFive & \StrokeOne \StrokeFive & \StrokeTwo \StrokeFive\\
    \StrokeThree & \StrokeFour & \StrokeFive & \StrokeOne \StrokeFive & \StrokeTwo \StrokeFive & \StrokeThree \StrokeFive \\
    \StrokeFour & \StrokeFive & \StrokeOne \StrokeFive & \StrokeTwo \StrokeFive & \StrokeThree \StrokeFive & \StrokeFour \StrokeFive\\
    \StrokeFive & \StrokeOne \StrokeFive & \StrokeTwo \StrokeFive & \StrokeThree \StrokeFive & \StrokeFour \StrokeFive & \StrokeFive \StrokeFive\\
    \end{tabular}
    \hspace{1cm}
    \begin{tabular}{|c|cccccc|}
        \hline 
        + & 0 & 1 & 2 & 3 & 4 & 5\\
        \hline 
        0 & 0 & 1 & 2 & 3 & 4 & 5 \\
        1 & 1 & 2 & 3 & 4 & 5 & 6\\
        2 & 2 & 3 & 4 & 5 & 6 & 7\\
        3 & 3 & 4 & 5 & 6 & 7 & 8\\
        4 & 4 & 5 & 6 & 7 & 8 & 9\\
        5 & 5 & 6 & 7 & 8 & 9 & 10\\
    \hline
    \end{tabular}
\end{center}
Beyond a table we need some formulas.  If we have two groups $m$ and $n$ and each is either nothing 
or a successor to something else, then that leaves us with just four cases 
to consider.
\begin{align*}
    \begin{array}{|c|cc|}
        \hline 
        + & 0 & S(k)\\
        \hline 
        0 & 0 & S(k) \\
        S(\ell) & S(\ell) & S(S(\ell+k))\\
        \hline
    \end{array}
\end{align*}
As the first column does not change $m$ we can simplify this down to 2 cases.
\begin{align*}
    m+n \defeq \begin{cases} m & n=0\\ S(m+k) & n=S(k)\end{cases}
\end{align*}
Lets see this out one some examples, recalling that $1=S0$, $2=S1$, $3=S2$ and so on.
\begin{align*}
    3+2 & = S(3+1)= SS(3+0) = SS3=5.\\
    7+3 & = S(7+2) = SS(7+1) = SSS(7+0)=SSS7=10.
\end{align*}
Notice something about this process is consuming, that is ``eliminating'',
our number $n$ in the sum $m+n$.  Because we are eliminating induction this 
is often called \emph{recursion}.
\begin{remark}{Induction and Recursion}\index{induction}\index{recursion}
    Induction is the process of introducing new terms built up 
    from operations on existing terms starting with a base case.
    Recursion is the process of taking apart a term by unwinding 
    the operations back to a base case.
\end{remark}


This process gives immediate significance to computation.
To use an inductive type in a program we simply run through the cases.
Programming languages allow for case-by-case analysis in numerous ways 
but most today offer some form of \emph{pattern matching} which is 
where we list the types of production rules and then attach the outcome 
for each.
\begin{lstfloat}
\begin{center}
\begin{minipage}{0.4\textwidth}
\begin{Fcode}[]
+:Nat->Nat->Nat
+ m  0    = m
+ m (S k) = + m k
\end{Fcode}
\end{minipage}
\hfill
\begin{minipage}{0.59\textwidth}
\begin{Pcode}[]
def add(m,n:Nat):Nat =
  match n with 
    Zero() => m
    Next(k)=> Next(add(m,k))
\end{Pcode}
\end{minipage}
\end{center}
\caption{Peano's addition of natural numbers programmed in two different languages.}
\label{lst:peano}
\end{lstfloat}


So in a sense our grammar's role is to guard that data given fits 
a pattern.  Once that has been accepted, when we encounter data 
of this grammar's type we can use that pattern to define functions 
that consume that data.  Historically this is known as \emph{eliminiation}
on account that it may eliminate the data given in the process of 
creating new data as output.  

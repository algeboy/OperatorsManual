

\section{Induction: Proof or Program?}
Your induction proofs of falling dominos are starting to fall into programs.
That's a currency of its own worth more investment.

Here's a second use of natural numbers which reduces them by 1 until we reach $0$.
\begin{align*}
    \text{dec}(n) & =\begin{cases}
        0 & n=0\\
        k & n=S(k)
    \end{cases}
\end{align*}
Your programming instructor might not want you to call that recursion, you didn't 
even bother to call $\text{dec}$ inside the function.  Yet it clearly 
depends on taking apart the way the input was built inductively.  So if 
consuming induction is recursion (it is) then this is recursion.

Maybe it is not computation I am after but learning a quality through proof.

\begin{proposition}
    For every natural number $n:\mathbb{N}$, $0+n=n$.
\end{proposition}
\begin{proof}
    By definition of $+$, $0+0\defeq 0$.
    Then for the inductive step assume 
    $0+k=k$.  Then for $n=S(k)$, 
    \begin{align*}
        \tag{Nominal Equality}
        0+n & = 0+S(k)\\
        \tag{def $+$}
        & \defeq S(0+k)\\
        \tag{Inductive Hypothesis}
        & =S(k)\\
        \tag{Nominal Equality}
        & =n.
\end{align*}
\end{proof}

\subsection{Programs as proofs}
Calling our function $\text{dec}{n}$ was suggestive that 
it decreased $n$ by one until it reached $0$.  What if I called 
it $f$ or \text{inc}? Its steps would not change suggesting that 
the quality of decrementing is not really one that belongs in 
the documentation of the function but rather is a quality of the 
operations themselves.  We might therefore actually capture a 
this as a proposition.
\begin{proposition}
\(
    (\text{dec}(0)=0) \wedge (\text{dec}(S(k))=k).
\)
\end{proposition}
\begin{proof}
    By cases, $\text{dec}(0)$ is defined as $0$ and $0=0$ by 
    the reflexive law.  In the other case $\text{dec}(S(k))=k$ 
    and again $k=k$ by reflexivity of $k$.
\end{proof}
The program became a proof, and in fact the two carry the 
same information, only the proof is more conversational.
You can imagine a programming language that lets you program 
by writing a proof and the algorithm is a by-product.  You would 
never need to document your algorithms, or run any tests as 
it would all be proved.  That's not a fiction, it already exists
but it is hard to program in this way which is part of why I am 
writing this book to train into you the patterns you will need 
to be successful when your time comes.


\subsection{Proofs as programs}
Perhaps it is not surprising that proofs are powerful enough to 
describe algorithms.  The surprise is that the reverse is also true.
Let revisit $0+n=n$.  I am sure you noticed that similar to how 
we defined $\text{dec}$ we dealt with cases.  
Perhaps there is not much difference between the too, that is, 
a proof by induction maybe could be treated as a form of recursion. So the 
proof is its own form of program.  To do this lets think of 
propositions as their own little grammar and the strings accepted 
by the grammar are what we call ``proofs''.   After all, you read 
my proof above in some language, maybe we can just make it a 
miniature language so that its more obvious what is happening. 

For example, maybe I want to claim $x=x$.  We know 
this is not so much proved as it is part of the definition of equality, 
an axiom.  So we can just make the axiom's name the accepted 
string of this grammar.  Here I made up the following 
multilingual grammar $EQ(x,y)$
\begin{center}
\begin{Gcode}
<<x=y>>
<<x=x>> ::= refl_x
\end{Gcode}
\end{center}
Note that $\refl_x$ is a frequent abbreviation for ``reflexive law for $x$''
and we replace $x$ with whatever object we are interested in comparing.
For example:
\[
    \refl_0:(0=0)\qquad \refl_{1}:(1=S(0))
    \qquad \refl_{2}:(1+1=2).
\]
The last two display a forgoing assumption.
Recall our grammar is $EQ(x,y)$ so $EQ(1,S(0))$ is just $EQ(S(0),S(0))$ 
because $1\defeq S(0)$.  So in a strict sense
the string $\refl_1$ \emph{is} $\refl_{S(0)}$ just as $1=S(0)$ \emph{is}
$S(0)=S(0)$. Hence, $\refl_{S(0)}$ is accepted by the 
rule \code{<S(0)=S(0)>}.  That is the entire point of the 
nominal equality Walrus.  Think through the same for $2$:
$1+1\defeq S(1+0)\defeq S(S(0))$ and $2\defeq S(S(0))$.  So 
\begin{align*}
    \refl_{S(S(0))} =\refl_2 & : (S(S(0))=S(S(0))) = (1+1=2).
\end{align*}


Now what about the odd looking tag \code{<<x=y>>}, did we forget 
to include a production rule for this?  No.  The point is that 
we can propose that $1=2$ just as much as we can propose that $1=1$.
It is just that only one of these has evidence we consider to be 
enough to prove the claim.  So by accepting $\refl_1:(1=1)$ we 
are declaring there is a proof.  If we have no strings of type $1=2$ 
then we cannot make the similar claim.

Now we can revisit the proof the zero-law $0+n=n$ as a program 
split into cases
\[
    zeroLaw(n) \defeq \begin{cases}
        \text{refl}_0:(0+0=0) & n=0\\
        \text{refl}_n:(0+n=n) & n\defeq S(k)
    \end{cases}
\]
where the proof actually computes the following change in grammar names
not the actual strings in the grammar
\begin{align*}
    (0+0=0) & \longrightarrow_{+\text{case}} 0=0\\
    ((0+n)=n) & \longrightarrow_{\text{nominal}}  (0+S(k)=S(k))\\
     & \longrightarrow_{+\text{case}} (S(0+k)=S(k))\\
     & \longrightarrow_{\text{indHyp}} (S(k)=S(k))
\end{align*}
Therefore whatever string $zeroLaw(n)$ outputs it has type $(0+n=n)$.
It is a program, but its output could be read as proving the claim because 
our grammar has no path to produce a string of type $x=y$ except when $x$ is $y$.



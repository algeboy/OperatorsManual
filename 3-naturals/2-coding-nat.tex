\section{Realizing grammar}

We do well to acknowledge inductive types of data as basic programs. Computers
understand this much. Listing~\ref{lst:peano} shows two programs you could run
today that implement Peano's idea. There are of course many differences.
Visibly, the left-hand side favors mathematically minded symbolic notation and
economizes even on parentheses in the spirit of ``$\sin x$'' notation. Meanwhile
the right-hand side favors a verbose imitation of natural language and prefers
the $\sin(x)$ notation.  Set the differences aside.

\begin{lstfloat}
\begin{center}
\begin{minipage}{0.37\textwidth}
\begin{Fcode}[]
induct Nat = Z 
    | S (k:Nat)

zero = Z
two = S (S zero)
\end{Fcode}
\end{minipage}
\hfill
\begin{minipage}{0.62\textwidth}
\begin{Pcode}[language=Sava]
enum Nsat
  case Zero() extends Nat
  case Next(k:Nat) extends Nat
sealed  // no more cases
zero = new Zero()
two = new Next(new Next(zero))
\end{Pcode}
\end{minipage}
\end{center}
\caption{Peano's natural numbers programmed in two different programming languages.}
\label{lst:peano}
\end{lstfloat}
    

Even without a deep understanding of these programs, one can make out the
contours of Paeno's definitions.  Both use a mix of keywords (in blue) to tell
our system to prepare a new type (or class) of data that will be called
\code{Nat}.  Then they instruct the system to accept exactly two ways to make
such data. It may be some initial state, \code{Z}, respectively \code{Zero},
that depends on nothing; otherwise, we must give data \code{k} of type
\code{Nat}, denoted \code{k:Nat}, which will then produce new data \code{S k},
respectively \code{Next(k)}.  In many systems the keyword \code{new} is 
used to help clue the reader into the fact that this is some data being now 
created.  It helps remind us that numbers do not  exists on their own, 
we have to expend resources (energy \& storage) to create them. 

